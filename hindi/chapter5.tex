
%chapter 5
\chapter{\texthindi{हास्य
}}
%\begin{multicols}[2]

\poemtitle{\texthindi{बाँके लाल का ढाबा
}}\begin{verse}[\versewidth]\texthindi{
बाँके लाल तानसेन के घराने से था। संगीत\\
उस की रग रग में बसा था।\\
उस ने बॉम्बे में गाने रिकॉर्डिंग स्टूडियो के\\
बाहर एक ढाबा खोला।\\
मशहूर बैकग्राउंड गाने वाले खाना इस ढाबे में\\
खाते थे। बाँके का हुक्म था की हर बात या\\
आर्डर गाना गा के होना चाहिये।\\
सब से पहले मन्ना डे आये\\
अरे बाँके भाई, गरम गरम नान खिला दे\\
बाँके—“गा के माँगो तो मिले गा”\\
तर्ज़ —ओ मेरी ज़ोहरा जबीं…\\
ओ मेरे बांके भय्या\\
दे दूँ गा दस रुपइया x2\\
दे दे गरम नान\\
भूख से निकली मेरी जान\\
मेरी जान\\
\\
प्रदीप\\
अरे बाँके आज मीठा खाने का मन कर रहा\\
है।\\
बाँके— “गा के माँगो प्रदीप जी”\\
तर्ज़ —देख तेरे संसार की हालत\\
दूध मलाई लड्डू पेड़े दे दो बाँके लाल x2\\
खाये हुए पूरा हुआ है साल\\
खाये हुए पूरा हुआ है साल\\
\\
तलत महमूद ने उबले अंडे माँगे थे। पर बाँके\\
किसी और को दे देता है। तलत शिकायत\\
करते हैं।\\
तर्ज़ —-जलते हैं जिस के लिये\\
लाया था मेरे लिये\\
अंडे तू ने उस को दिये\\
जल्दी से वापिस ला\\
वो तो थे मेरे लिये\\
लाया था मेरे लिये x2\\
\\
मुकेश शाकाहारी खाना माँगते हैं\\
तर्ज़ —मुझ को इस रात की तन्हाई में\\
अंडे मछली ना दो\\
मीट लहसन ना दो\\
और प्याज़ ना दो ओ\\
और प्याज़ ना दो ओ ओ\\
और प्याज़ ना दो\\
दाल भाजी दे दो\\
चाहे चावल दे दो\\
मगर प्याज़ ना दो ओ\\
मुझे प्याज़ ना दो ओ\\
ओ प्याज़ ना दो\\
आशा भोंसले को बहुत ज़्यादा भूख लगी है।\\
खूब सारा खाना माँगती है\\
तर्ज़ —दम मारो दम मिट जाये गम\\
गोभी शलगम, आलू हों दम\\
चिकन गरम, रोटी पूरी चावल नान\\
रोटी पूरी चावल नान\\
\\
हेमंत खाना खा लेते हैं। जेब में हाथ डालते हैं\\
पैसे देने के लिये। मगर जेब खाली है। बाँके\\
को बताते हैं क्या हुआ।\\
तर्ज़ —जाने वो कैसे लोग थे जिन के…\\
खाने के पैसे थे पॉकेट में\\
छोड़ा जब घर का द्वार\\
राह में चोर खड़े थे लूटा\\
उन्हों ने बारम्बार\\
खाने के पैसे थे पॉकेट में\\
छोड़ा जब घर का द्वार x2\\
\\
असित सेन आते हैं और कहते हैं “मुझे कुछ\\
नहीं खाना।\\
तर्ज़ —-एहसान तेरा हो गा मुझ पर\\
एहसान तेरा हो गा मुझ पर\\
ये लड्डू बर्फ़ी रहने दो\\
मुझे तोंद से नफ़रत हो गई है\\
मुझे डायट पर ही रहने दो\\
\\
बाग में जब हम सैर को निकलें\\
इन से नहीं लोग मुझ से पूछें\\
ये लड़का है या लड़की है\\
बाकी एक है या महीने दो\\
मुझे तोंद से नफ़रत हो गई है\\
मुझे डायट पर ही रहने दो\\
\\
मोहम्मद रफ़ी के आने पर बाँके बोला\\
तर्ज़—ए मेरे दिल कहीं और चल\\
सब्ज़ी मीट चावल चना दाल दही\\
आलू गोभी पराँठा ख़त्म हुआ\\
ढूँढ लो तुम कोई ढाबा नया x2\\
\\
रफ़ी साहिब बोलते हैं।\\
तर्ज़ — बाबुल की दुआएँ\\
दो रोटी मुझ को देता जा\\
थोड़ा प्याज़ अचार ही दे देना\\
मीठा जो अगर कुछ बच जाये\\
थोड़ा सा वो भी देना\\
(रोते हुए)\\
दो रोटी मुझ को देता जा\\
थोड़ा प्याज़ अचार ही दे देना\\
\\
ढाबा बंद करते हुए बाँके लाल गाता है\\
तर्ज़ — चल उड़ जा रे पंछी\\
उठ जाओ रे… गवियो ओ ओ ओ\\
उठ जाओ रे गवियो कि अब ये ढाबा बंद हुआ\\
जाओ अपने घर सारे अब ये ढाबा बंद हुआ\\
\\
फिर मिलें गे इसी जगह पर\\
खाना जहां तुम खाते हो\\
जो जी चाहे सब मिलता है\\
गाना जब तुम गाते हो\\
किसी चीज़ की कमी नहीं यहाँ\\
बाँके को तुम भाते हो\\
जल्दी वापिस आना यारो\\
रब से करता दुआ\\
उठ जाओ रे गवियो कि\\
अब ये ढाबा बंद हुआ\\
अब ये ढाबा बंद हुआ
}\end{verse}

\poemtitle{\texthindi{पोकर
}}\begin{verse}[\versewidth]\texthindi{
सच को झूठ, झूठ को जीत बनाना है\\
दूजे की दौलत को अपने घर लाना है\\
तो भैया पोकर सीखो\\
\\
किस्मत से पत्ते भारी हैं तो जीत गये\\
खुद को बुद्धिमान समझना है\\
तो भैया पोकर सीखो\\
\\
जीवन में हर कोई नहीं जीतता\\
इक पत्ते की कमी बना घर गिरता\\
तो भैया पोकर सीखो\\
\\
अपनी जीत दूजे की हार है होती\\
उसे रुला कर खुश होना है\\
तो भैया पोकर सीखो\\
\\
हल्के हों या भारी पत्ते किस्मत से हैं\\
जो जीवन देता उस में खुश रहना है\\
तो भैया पोकर सीखो\\
\\
जीत में खुश और हार में रोना\\
दो पल का दौर गुज़र जाये गा\\
नया पत्ता नई आशा ले आए गा\\
तो भैया पोकर सीखो\\
रब की कृपा से खेलना मिला\\
पोकर या जीवन में\\
अगर जीतना है दोनों में\\
खुश रहने के साधन अपनाओ\\
किस्मत को दोषी ना ठहराओ\\
तो भैया पोकर सीखो\\
भैया पोकर सीखो
}\end{verse}

\poemtitle{\texthindi{बीवी
}}\begin{verse}[\versewidth]\texthindi{
तर्ज़ —ज़िंदगी खाब है \\
\\
बीवी ऐसी चाहिए मेरी वो गुलाम हो\\
सुबह हो या शाम हो2\\
\\
पांच फुट तीन इंच हाइट हो\\
जींज़ जिस की टाइट हो\\
ज़ुल्फ़ काली घटा हो जैसे\\
रंग जिस का वाइट हो\\
चेहरे पे मुस्कान रहे सदा\\
मुझ से कभी ना फाइट हो\\
बीवी ऐसी चाहिए...\\
\\
बाहर का भी काम करे\\
सास ससुर की सेवा करे\\
घर चमकाए शीशे की माफ़िक\\
खाना नित नया बना करे\\
घर जब लौटूं गोल्फ खेल के\\
मालिश मेरी किया करे\\
बीवी ऐसी...\\
\\
एक दर्जन बच्चे दे दे\\
टीम करिकेट की घर में हो\\
घर में मेला शोर शराबा\\
कार्निवल भी घर में हो\\
सारी पलटन को ये सँभाले\\
हाथ मेरे बीयर ही हो\\
बीवी ऐसी चाहिए...\\
बीवी ऐसी चाहिए मेरी वो गुलाम हो\\
सुबह हो या शाम हो
}\end{verse}

\poemtitle{\texthindi{पति
}}\begin{verse}[\versewidth]\texthindi{
ऐसा पति दे भगवन\\
सुबह उठ छुए मेरे चरण\\
बिस्किट और चाय वो लाये\\
प्यार से वो मुझ को जगाये\\
आँख खोलो मेरे सजन\\
ऐसा पति…\\
\\
मुझे जगाने से पहले\\
बच्चों को वो तयार करे\\
स्वामी नाश्ता भी तयार करे\\
खर्चा चलाने की खातिर\\
सोलह घंटे काम भी करे\\
ऐसा पति दे भगवन…\\
\\
जितने भी पैसे कमाये\\
मेरे हाथ में ही थमाये\\
स्वामी पांच वीज़ा कार्ड दिलाये\\
शॉपिंग मुझे ले जाये\\
सोना और हीरे दिलाये\\
ऐसा पति दे…\\
\\
साड़ी मेरी प्रेस कर के\\
नेल पॉलिश मुझे लगाये\\
स्वामी जूते भी चमकाये\\
पार्टियों में सज के मैं जाऊँ\\
रोल्स रॉयस में बिठाये\\
ऐसा पति दे…\\
\\
आमिर खान जैसी स्माइल हो\\
देव आनंद जैसे बाल हों\\
स्वामी उस के गाल लाल लाल हों\\
अमिताभ जैसी ऊँचाई
}x\texthindi{2\\
सलमन खान जैसी चाल हो\\
ऐसा पति दे भगवन\\
\\
ऐसा पति दे भगवन\\
सुबह उठ छु ए मेरे चरण\\
बिस्किट और चाय वो लाये\\
प्यार से वो मुझ को जगाये\\
आँख खोलो मेरे सजन\\
ऐसा पति दे भगवन\\
}\end{verse}

\poemtitle{\texthindi{पति पत्नी की नोक झोंक
}}\begin{verse}[\versewidth]\texthindi{
पर्दा उठता है। सोफ़े पे सफ़ेद कोट पहने\\
डॉक्टर कुर्सी पे बैठा वाल स्ट्रीट जर्नल पढ़\\
रहा है। वाल स्ट्रीट शब्द लोगों की तरफ़ हैं।\\
मेज़ पे स्टेथोस्कोप है। साथ लक्ष्मी देवी की\\
फ़ोटो है।\\
पति—हे लक्ष्मी माँ, आज तो कृपा कर दे।\\
कितने दिन से स्टॉक नीचे ही नीचे जा रहे हैं।\\
इंटेल, बोइंग, सिस्को सब सो गये हैं। देवी उन\\
को जगाओ ना। प्लीज़! 2 दिन हो गये हैं व्रत\\
रखे हुए। कृपा कर दे मैया।\\
ओम ओम ओम…\\
\\
पत्नी स्टेज पे आती है। हाथ में एक महँगी\\
लगती साड़ी है। उस को कभी इधर से कभी\\
उधर से देखती है।\\
कहती है—-आ गये?\\
इतनी लेट?\\
फिर लग गये स्टॉक और बौंड के चक्कर\\
में!\\
(पति अखबार को बंद कर के उठता है।
)\\
पति —-अच्छा बंद कर दी अखबार। बोलो\\
क्या हाल है जनाब का?\\
पत्नी—तुम्हें याद है आज कौन सा दिन है?\\
पति सोच के बोलता है —आज, आज सैटरडे\\
है। हफ़्ते में छे दिन काम कर कर के थक गया हूँ।\\
पत्नी— नहीं, आज हमारी पच्चीसवीं साल\\
गिरह है।\\
(पति घबरा के उठता है और कहता है
)\\
अरे मैं तो भूल गया। माफ़ कर दो। मैं तुम्हारे\\
लिये कोई गिफ्ट भी नहीं लाया। आज तो\\
लेट हो गया हूँ। कल सुबह सब से पहला का\\
काम…\\
\\
पत्नी—शादी से पहले रोज़ कभी गुलाब के \\
फूल, कभी सिनेमा, कभी हैसियत से भी महँगे\\
तोहफ़े लाते थे। शादी हुई ये सब भूल गये।\\
मुझे मालूम था।\\
इस लिये मैं खुद ही तुम्हारी तरफ़ से इक\\
साड़ी ले आई हूँ।\\
देखो अच्छी है ना। इस में हीरे मोती जड़े हुए\\
हैं।\\
(पति हाथ लगाता है।
)\\
पति—ये तो बहुत सुंदर है। तुम पर बहुत सजे गी। बहुत महँगी तो नहीं?\\
पत्नी —एक डॉक्टर की बीवी हूँ। सस्ती पहनू \\
गी तो लोग सोचें गे तुम अच्छे डॉक्टर नहीं हो,\\
यही सोच कर बस पच्चीस हज़ार की ही लायी हूँ।\\
पति—पच्चीस हज़ार रुपए?\\
पत्नी—आप भी ना बुद्धु ही रहे। रहते \\
अमरीका में और खर्चा रुपयों में। सिर्फ़ \\
पचीस हज़ार डॉलर की है! मिसेज़ वर्मा तो \\
पचास हज़ार से कम साड़ी देखती भी नहीं। मैं \\
ने तो सिर्फ़ पच्चीस हज़ार ही खर्चे हैं !\\
(पति सिर पकड़ के कुर्सी पे बैठ जाता है।\\
पत्नी उस की तरफ़ बड़ती है। पति खड़ा होता\\
है। गुस्से में गाता है।
) \\
गाने की धुन—तुम रूठी रहो मैं मनाता रहूँ\\
मज़ा जीने का और भी आता है\\
\\
पति—तुम खरचती रहो मैं कमाता रहूँ\\
पैसा जैसे मुफ़त में आता है
}x\texthindi{2\\
पत्नी—तुम स्टॉक खरीदो लाखों बॉण्ड खरीदो x2 \\
मेरा साड़ी का लाना नहीं भाता है\\
इक साड़ी का लाना नहीं भाता है\\
तुम स्टॉक खरीदो ओ ओ ओ\\
\\
पति —मैनेज्ड केयर ने बहुत सताया\\
एच एम ओ ने ठेंगा दिखाया X2\\
नींद वकीलों ने है चुराई\\
मेडिकेयर ने भी साथ छुड़ाया\\
सूरज उगने से पहले चाँद छुपने के बाद\\
तेरा घर वाला घर वापिस आता है\\
तुम खरचती रहो ओ ओ ओ\\
\\
पत्नी—भूल सदा तुम को है होती\\
तुम करो काम सारा दिन मैं तो सोती x2\\
घर का चलाना कोई आसान काम नहीं\\
माँजूँ मैं बर्तन तेरे कपड़े मैं धोती\\
तेरी रोटी पकाऊँ तेरे बच्चे भी पालूँ x2\\
सारा दिन तो यूँ ही गुज़र जाता है\\
तुम स्टॉक खरीदो ओ ओ ओ\\
\\
पति—मुझे तो खबर ना थी ऐसा तेरा हाल है\\
पत्नी—दिन ढले मैं जानू क्यों तू बेहाल है\\
दोनों मिल के—\\
हाथ पकड़ लो मेरा सजनवा\\
थाम मुझे तू रखूँ तेरा ख्याल मैं\\
चाहे रो के गुज़ारें चाहे हंस के बिताएँ\\
समा जीवन का यूँ ही गुज़र जाता है
}x\texthindi{2\\
पत्नी—तुम कमाते रहो मैं खरचती रहूँ\\
मज़ा जीने का और भी आता है\\
पति—तुम खरचती रहो\\
पत्नी—तुम स्टाक खरीदो\\
तुम तुम तुम…\\
स्टेज से उँगलियाँ दिखाते चले जाते हैं
}\end{verse}

\poemtitle{\texthindi{पैसे के दो रूप
}}\begin{verse}[\versewidth]\texthindi{
हाय पैसा हाय पैसा\\
रोग लगा रे हाय कैसा\\
मात पिता को रुला दिया\\
देश भी अपना भुला दिया\\
हाय पैसा हाय पैसा\\
\\
वाह पैसा वाह पैसा\\
देखा नहीं कोई तुझ जैसा\\
ऊँचे महल बनाये तू\\
हीरे मोती दिलाये तू\\
वाह पैसा वाह पैसा\\
\\
आँख को अंधा कर दे ये\\
कान को बहरा कर दे ये\\
अक्ल पे लग जायें ताले\\
काम कराये ये काले\\
हाय पैसा हाय पैसा\\
\\
झुक झुक लोग सलाम करें\\
नेता गण मेरा पानी भरें\\
जग में नाम कराये ये\\
बिगड़े काम बनाये ये\\
वाह पैसा वाह पैसा\\
\\
भाई बहनों में वैर करा दे\\
अपनों को ये गैर बना दे\\
तोड़े यारों की यारी\\
माया का पलड़ा भारी\\
हाय पैसा हाय पैसा\\
\\
सब से महँगी गाड़ी लें\\
हीरों से जड़ी साड़ी लें\\
जो जी आये खरीदें हम\\
यारों को भी खरीदें हम\\
वाह पैसा वाह पैसा\\
\\
बच्चों का बचपन ना देखा\\
देखी पैसे की रेखा\\
सोलह घंटे काम किया\\
धन खातिर घर त्याग दिया\\
हाय पैसा हाय पैसा\\
\\
दुनियाँ की ये सैर करा दे\\
चाँद और तारे ज़मीं पे ला दे\\
पूरे कर दे सब अरमान\\
पैसा खुशियों की है खान\\
वाह पैसा वाह पैसा\\
\\
माया तो आनी जानी है\\
साथ नहीं ये जानी है\\
राम नाम धन क्यों भूला\\
असली रूप को क्यों भूला\\
हाय पैसा हाय पैसा\\
\\
राम मिले ना पैसे से\\
यार मिले ना पैसे से\\
प्यार नहीं जहाँ धन का राज\\
इस सच को पहचान ले आज x2\\
हाय पैसा हाय पैसा\\
हाय पैसा हाय पैसा\\
हाय पैसा हाय पैसा\\
\\
पैसे वाला हाथ मसलता है\\
कोई जवाब नहीं सूझता।\\
स्वामी जी के पाँव पड़ जाता है।\\
कहता है —आप धन्य हैं। आप ने सुख और\\
शांति का पथ दिखाया। उस के लिये कोटी\\
कोटी धन्यवाद। भगवान का लाख लाख\\
धन्यवाद।\\
सिया पति राम चन्द्र की जय\\
दोनों बोलते हैं\\
सिया पति राम चन्द्र की जय\\
( ये एक स्किट बन सकता है।\\
एक किरदार कमीज़ या कोट पर नकली\\
रुपयों के या किसी भी किस्म के नोट चिपका\\
लेता है। ऊपर वाली जेब में रंगीन रूमाल\\
निकला हुआ है, महँगा चश्मा पहना है जो\\
एक अंतरे के बाद उतार देता है। चेहरे और\\
चाल में अकड़ है।\\
दूसरा किरदार भगवे कपड़ों में है। खड़ांवें\\
पहनी हैं। चेहरे पे मुस्कान है।
)
}\end{verse}

\poemtitle{\texthindi{शराबी
}}\begin{verse}[\versewidth]\texthindi{
शराबी मुख व्हिस्की के लिए ही दिया\\
शराबी मुख व्हिस्की के लिए ही दिया\\
\\
इंडिया का जॉनी वॉकर बॉम्बे में बस्ता\\
इन का जॉनी ढूँढे स्कॉटलैंड का रस्ता\\
शाम ढले घूँट लगाया दोस्तों ने साथ दिया\\
शराबी मुख व्हिस्की के लिये ही दिया\\
\\
सिंगल माल्ट पर पैग तो डबल हो\\
इन की मस्ती हो बीवी को ट्रबल हो\\
जूते पड़े डांट भी खाया\\
फिर भी मुँह बोतल में दिया\\
शराबी मुख व्हिस्की के लिए ही दिया\\
\\
पहले पेग ने शेर बनाया\\
दूजे ने बनाया बन्दर\\
सूअर जैसी सूरत बन गयी\\
तीसरा गया जो अन्दर\\
कुंभकर्ण से रिश्ता बनाया\\
खर्राटों ने सोने ना दिया\\
शराबी मुख व्हिस्की के लिए ही दिया\\
शराबी मुख व्हिस्की के लिए ही दिया
}\end{verse}

\poemtitle{\texthindi{आधुनिक दिवाली
}}\begin{verse}[\versewidth]\texthindi{
न फूल न पूजा की थाली\\
न कोई लेता राम का नाम\\
फ़िल्मी गाने भजन बने\\
चरणामृत शराब का जाम\\
\\
न मंदिर की घंटी ना शंख की\\
सुनी आवाज़\\
टकराते छलकते ग्लास बने\\
गीतों के साज़\\
\\
चढ़ावा मूर्ति पर नहीं\\
तीन पत्ति खेल पे था\\
न माँगी रब से शांति\\
मन ऊँची ट्रेल पे था\\
\\
घर में अगरबत्ती ना काली धूप जली\\
उड़ते बादल दिखे जब सिगरेट जली\\
\\
न सुंदरकांड पढ़ें\\
न हनुमान चालीसा बोली\\
चुटकुले सुनाए कोने में\\
हंसती इक नटखट टोली\\
\\
पैर लड़खड़ाए होंठ थर्राए\\
मदिरा अपना रूप दिखाए\\
श्रद्धा से साष्टांग नहीं\\
मदिरा फ़र्श पे उसे लिटाए\\
\\
शोरगुल में बोलें सभी\\
सुने न दूजे की बात\\
हैपी दीवाली सब कहें\\
पूछे ना दिल के हालात\\
\\
सीधे चल के आए थे\\
दीवाली की खुशी मनाने\\
कुछ होश में कुछ मदहोश\\
चले अपनी कार चलाने\\
\\
दिये गायब बनी चीन में\\
लड़ियाँ घर चमकायें\\
बचपन की दिवाली\\
बच्चों को कैसे समझायें\\
\\
वाह मेरे भाइयो बहनो\\
क्या हाल किया है\\
दीवाली का निकाल दीवाला\\
उसे बेहाल किया है\\
\\
बाहरी चमक धमक बंद कर\\
अंदर के दीप जला\\
आधुनिक दीवाली छोड़\\
बचपन की दीवाली मना\\
आधुनिक दीवाली छोड़\\
बचपन की दीवाली मना\\
}\end{verse}

\poemtitle{\texthindi{जॉनी का सर दर्
}}\begin{verse}[\versewidth]\texthindi{
पर्दा खुलता है। स्टेज पे एक सोफ़ा है। उस\\
की बाईं ओर छोटे मेज़ पे टेलीफ़ोन है। \\
दूसरी ओर मेज़ पे लैंप है। बीवी जिस का \\
नाम मेरी है, लैंप को कपड़े से साफ़ कर रही \\
है। उस का पति जॉनी दाईं तरफ़ से स्टेज पे\\
आता है। माथे को पकड़े हुए जैसे सर में बहुत दर्द है।\\
मेरी को कहता है।\\
तर्ज़—चंपी तेल मालिश, सर जो तेरा चकराये\\
जॉनी—मेरी ओ मेरी\\
मेरी(
  झुंझला के 
 ) अब क्या है ?\\
जॉनी—सर दर्द से फटा जाये\\
और दिल मेरा घबराये\\
इस से पहले दम तोड़ूँ मैं\\
डॉक्टर जल्दी बुलाये\\
डॉक्टर जल्दी बुलाये\\
\\
(मेरी डॉक्टर को फ़ोन करती है
)
तर्ज़—मन डोले मेरा तन डोले\\
डॉक्टर आना जल्दी आना\\
मेरा घर वाला है बीमार रे ए\\
कौन बचाये तेरे बिना x2\\
डॉक्टर आता है। सफ़ेद कोट, गले में\\
स्टेथोस्कोप है\\
तर्ज़—दुनियाँ बनाने वाले क्या तेरे मन\\
कौन मरीज़ है बोलो\\
किस को मुझे है बचाना\\
दुखियों की पीड़ा मिटाना मुझे\\
दुखियों की पीड़ा मिटाना x2\\
जॉनी कहता है—\\
मेरा सर दर्द बहुत ही ज़्यादा बड़ गया है।\\
लगता है स्ट्रोक हो गयी है।\\
(गिरते हुए सोफ़े का सहारा ले कर फ़र्श पे\\
बैठ जाता है और कहता है
)\\
हम छोड़ चले हैं महफ़िल को\\
याद आये…\\
(फ़र्श पे आँखें साँस बंद किए गिर जाता है
)\\
मेरी कहती है—हाय हाय!\\
गाना तो ख़त्म कर जाते!\\
(जॉनी उठ जाता और गाता है
)\\
कभी तो मत रोना\\
( आँखें साँस बंद कर के फिर लेट जाता है
)\\
(डॉक्टर गाता है
)\\
तर्ज़—सावन का महीना पवन करे सोर\\
डॉक्टर—जॉनी का नंबर आया\\
रब ने खींची ढोर\\
मेरी—ढोर नहीं डोर\\
डॉक्टर—रब ने खींची ढोर\\
मेरी—अरे बाबा ढोर नहीं डोर, डोर!\\
डॉक्टर—हाँ, रब ने खींची डोर\\
मेरी तू हो जा मेरी\\
चल नाचें जैसे मोर\\
डॉक्टर और मेरी एक दूसरे का हाथ पकड़ते\\
हैं। इक दूसरे को देख कर मुसकाते हैं। दोनों\\
स्टेज से बाहर की तरफ़ गाते हुए जाते हैं\\
डॉक्टर—मेरी तू हो जा मेरी\\
चल नाचें जैसे मोर\\
मेरी—डॉक्टर तू हो जा मेरा चल नाचें जैसे मोर\\
(लाइट्स मद्धम हो जाती हैं
)\\
दोनों —मेरी तू हो जा मेरी\\
डॉक्टर तू हो जा मेरा\\
चल नाचें जैसे मोर
}\end{verse}

\poemtitle{\texthindi{जीवन का चक्कर
}}\begin{verse}[\versewidth]\texthindi{
तर्ज़—ये देश है वीर जवानों का अलबेलों का\\
मस्तानों का\\
बड़े शौक से शादी करते हैं\\
और जल्दी से बच्चे करते हैं\\
इन बच्चों का यारो, होय!\\
इन बच्चों का यारो क्या कहना\\
ये मम्मी पापा का गहना\\
\\
ओ, ओ, ओ ओय ...\\
बड़े जोश से हनी मून गए\\
पर वहाँ तो एक्सीडेंट हुए\\
गए दो थे तीन, होए!\\
गए दो थे तिन वापिस आये\\
दोनों के कठिन अब दिन आये\\
\\
नौ मास मम्मी को तड़पाया\\
पापा भी लगे अब घबराया\\
घर छोटा पैसे, होए!\\
घर छोटा पैसे कम हों गे\\
अब बाल सफ़ेद और कम हों गे\\
\\
सारी सारी रात ये रोते हैं\\
और सारा दिन ये सोते हैं\\
सोया मुखड़ा रब का, होए!\\
सोया मुखड़ा रब का रूप लगे\\
और जगें तो जिन्न और भूत लगे\\
\\
मुश्किल दिन रात की भूल गए\\
जब आँख खुली तो स्कूल गए\\
दो शब्द इंग्लिश के , होए!\\
दो शब्द इंग्लिश के क्या जाने\\
मम्मी पापा को मूरख माने\\
\\
दो पल में टीनेजर बन गए\\
हम प्यादे ये मेजर बन गए\\
अब अक्ल का ठेका, होए!\\
अब अक्ल का ठेका इन्हें मिला\\
हर बात पे अब ये करें गिला\\
\\
ओ, ओ, ओ ओय ... \\
पंछी की तरह घर में आयें\\
मौसम बदले ये उड़ जायें\\
दो दिन के ही, होए!\\
दो दिन के ही मेहमान हैं ये\\
सच मानो अपनी जान हैं ये\\
\\
चाहे जो भी करें जहाँ पर भी रहें\\
ये टुकड़े हमारे, होए!\\
ये टुकड़े हमारे दिल के हैं\\
सदा रहें हमारे दिल में हैं\\
(पलक झपकते ये टुकड़े ना जाने कब बड़े हो\\
गए और उन्हों ने क्या किया?
)\\
ओ, ओ, ओ ओय ... \\
बड़े शौक से शादी करते हैं\\
और जल्दी से बच्चे करते हैं\\
उन बच्चों का यारो क्या कहना\\
वो मम्मी पापा का गहना\\
जीवन का चक्कर फिर से शुरू हो गया और\\
चलता गया
}\end{verse}

\poemtitle{\texthindi{कोविड के दिन
}}\begin{verse}[\versewidth]\texthindi{
तर्ज़—जायें तो जायें कहां\\
एक आदमी रॉकिंग कुर्सी पे बैठा है। हाथों पे\\
ग्लव्स, मुँह पे मास्क लगाया हुआ है। मेज़\\
पर किताबें, टी वी का रिमोट, टेलीफोन और पानी का गिलास है।\\
एक ओर छोटे मेज़ पे लाइसोल वाइप्स का\\
डिब्बा, ग्लव्स का डब्बा, एलकोहल की\\
शीशी है।\\
घर के दरवाज़े के बाहर एक बड़ा साइन है जेल
}\end{verse}

\poemtitle{\texthindi{जाऊँ तो जाऊँ कहाँ
}}\begin{verse}[\versewidth]\texthindi{
(गाने की केरियोकी शुरू होती है।\\
आदमी गाता है
)\\
जाऊँ तो जाऊँ कहाँ\\
जाऊँ तो जाऊँ कहाँ\\
देखूँ जिधर कोरोना वहाँ\\
मुँह को ढकूँ धोऊँ हाथ\\
डरता है दिल छुपा है कहाँ\\
जाऊँ तो जाऊँ कहाँ\\
\\
कोई खाँसे आगे\\
कोई खाँसे आगे कोई छींके पीछे\\
बचना है मुश्किल\\
छुपे वायरस से\\
छे फुट का रखूँ फ़ासला\\
जाऊँ तो जाऊँ कहाँ\\
\\
घर जेल बन गया\\
घर जेल बन गया खत्म है आज़ादी\\
बंद हैं दुकाने हुई बर्बादी\\
नेतागण् लड़ रहे\\
जनता तबाह\\
जाऊँ तो जाऊँ कहाँ\\
\\
जाऊँ तो जाऊँ कहाँ\\
जाऊँ तो जाऊँ कहाँ\\
देखूँ जिधर कोरोना वहाँ\\
मुँह को ढकूँ धोऊँ हाथ\\
डरता है दिल छुपा है कहाँ\\
जाऊँ तो जाऊँ कहाँ
}\end{verse}

\poemtitle{\texthindi{महँगे प्याज़
}}\begin{verse}[\versewidth]\texthindi{
थका हारा दफ़्तट्रर से जब घर मैं आया\\
बीवी को बेहोश रसोई में पाया\\
सुन मेरी बोली आँख उस ने खोली\\
फ़रश पर पड़ा खाली थैला दिखाया\\
“मैं लेने गयी प्याज़ जब सब्ज़ी मंडी\\
सुन उस की कीमत हो गयी मैं ठंडी\\
रुपए सौ आधे किलो के उस ने माँगे\\
लगी काँपने मेरी नाज़ुक सी टाँगे\\
मुंह पे पसीना रुकी साँस मेरी\\
छाया अंधेरा ज्यों रात घनेरी\\
गिरती हुई मैं घर लौट आई\\
कमज़ोर कुर्सी पे बैठ ना पाई\\
अब तुम को मालूम बेहोशी का राज़\\
सोने से महँगे हैं नाचीज़ से प्याज़\\
सोने से महँगे हैं नाचीज़ से प्याज़\\
नाचीज़ से प्याज़ नाचीज़ से प्याज़
}\end{verse}

\poemtitle{\texthindi{फ़ोन
}}\begin{verse}[\versewidth]\texthindi{
बचाओ इक चोर मेरे घर घुस आया\\
बचाओ बचाओ\\
दुनियाँ भुलाई नींद गंवाई इसी से दिल\\
लगाया\\
रिश्ते नाते भूल गये अब अपना भी लगे पराया\\
चोर नन्हें चमकते फ़ोन का साया\\
\\
यही रकीब यही सौतन इसी ने चैन गंवाया\\
हंसते खेलते परिवार को इसी ने रुलाया\\
कैंसर का इलाज है इस ने सब को हराया\\
चोर नन्हें चमकते फ़ोन का साया\\
\\
नज़र इसी पे हरदम हाथों से सहलाया\\
बात करे ना कोई, साधन इसे बनाया\\
उँगलियाँ ज़बान बनी होंठ सिल गये\\
एक अजब नया युग आया\\
\\
इसी संग सोना इसी संग दिन बिताया\\
इक पल आँखों से ओझल दिल घबराया\\
निकालें घर से कैसे मालिक बन आया\\
\\
गायब पुराने अस्तर इस ने सब को खाया\\
कैमरा जी पी एस कैलकुलेटर बना पराया\\
छोटे से राजा ने हम सब को गुलाम बनाया\\
\\
इसी के तौर तरीकों को सब ने अपनाया\\
प्रेमी घुट के लिपटे नज़रें इस से मिलायें\\
झुकी आँखें नीचे मुगल राज लौट आया\\
चोर मेरे घर घुस आया\\
चोर चमकते फ़ोन का साया\\
बचाओ बचाओ
}\end{verse}

\poemtitle{\texthindi{स्टॉक्स
}}\begin{verse}[\versewidth]\texthindi{
तर्ज़—दुनियाँ बनाने वाले क्या तेरे मन में\\
समाई\\
स्टॉक्स खरीदने वाले क्या तेरे मन में समाया\\
सारा ही पैसा लुटाया तू ने\\
सारा ही पैसा लुटाया
}x\texthindi{2\\
\\
जिस पैसे ने दिए पाँव में छाले\\
पेट में अल्सर जान के लाले\\
दिल की नाड़ी जिस पैसे ने बंद की x2\\
उसी पैसे को बेदर्दी उछाले\\
मेहनत की पूँजी को\\
जुए में क्यों रे लगाया\\
सारा ही पैसा लुटाया\\
स्टॉक्स खरीदने वाले...\\
\\
जिस दिन स्टॉक्स के स्पेलिंग तू जाने\\
उसी दिन खुद को पीटर लिंच माने\\
किस्मत से दो चार पैसे बन जाएँ कहीं से x2\\
सब पार्टियों में गाये खुशियों के गाने\\
ये धन जिसे अपना समझे\\
जल्द ही हो गा ये पराया\\
सारा ही पैसा लुटाया\\
स्टॉक्स खरीदने...\\
\\
सारी सारी रात तोहे नींद नहीं आये\\
डूबे हुए स्टॉकों की याद सताए\\
गुज़रे हुए वापस नहीं आते x2\\
उन के लिए क्यों तू आँसू बहाए\\
म्युनिसिपल बौंड खरीदो\\
यही है डौली ने सिखाया\\
सारा ही पैसा लुटाया\\
स्टॉक्स खरीदने वाले\\
क्या तेरे मन में समाया\\
सारा ही पैसा लुटाया\\
तू ने सारा ही पैसा लुटाया
}\end{verse} 

\poemtitle{\texthindi{अमरीका के कुछ रंग
}}\begin{verse}[\versewidth]\texthindi{
(जहां भी रहो फूल और काँटे हों ही। अमेरिका में बहुत फूल मिले। उस के साथ कुछ काँटों का वर्णन इस गीत मैं है
)\\
\\
पत्नी—\\
कहाँ ले चले हो बता दो मुसाफ़िर\\
सितारों से आगे ये कैसा जहाँ है\\
पति —\\
सितारों के सपने हो सके ना अपने\\
पहुँचे अमरीका नसीबा बुरा है\\
\\
यहाँ मुंडू माली और नौकर तुम्हीं हो\\
धोबी तुम्हीं और शौफ़र तुम्हीं हो\\
दो पल बैठना मुमकिन नहीं है\\
अब क्यों रोये क्यों पछताये\\
मिलता वही जो ऊपर लिखा है\\
\\
यहाँ महल ऊँचे मगर दिल हैं नीचे\\
यहाँ पैसा पहले मगर प्यार पीछे\\
बिज़नेस जो दे जाये सब से बड़ा है\\
यारी खरीदो प्यार खरीदो\\
सब रिश्तों से डॉलर बड़ा है\\
\\
सुनते थे पेड़ों पे डॉलर हैं खिलते\\
जेबें हैं भारी दिल नहीं मिलते\\
सोने के पिंजरे में जान फँसी है\\
हीरे खरीदें स्टॉक्स खरीदें\\
चौबीस घंटे ये ही कथा है\\
\\
दिन रात देश में गोलियां बरसती\\
रंग भेद भाव की आँधी है चलती\\
स्कूल जेल बन गये बच्चे हैं कैदी\\
पड़ोसी ना जाने सभी अनजाने\\
घर मेरा मकबरा बन गया है\\
\\
सितारों के सपने हो सके ना अपने\\
पहुँचे अमरीका नसीब बुरा है\\
पहुँचे अमरीका नसीब बुरा है\\
\\
सत्तर्वाँ जन्मदिन\\
\\
कल का छोकरा बुड्ढा हुआ है\\
अपने को समझे जवान\\
भागे दो गज़ निकले जान\\
\\
मुँह पे ज़्यादा सिर पे कम हैं\\
झड़ती ज़ुल्फ़ों का मौसम है\\
कान सुने ना आँख ना देखे\\
जो अंग छुओ वो ही नरम है\\
आती जाती को ये छेड़ें\\
दिल इन का है जवान\\
भागे दो गज़ निकले जान\\
\\
देख तेरे चेहरे की लकीरें\\
हाथ शर्म से डूबा जाये\\
सर तो काला कर बैठे हो\\
बाल छाती के कै से छुपायें\\
फ़ेस लिफ्ट करा लो\\
छाती मुंडा लो\\
कोई ना सके पहचान\\
भागे दो गज़ निकले जान\\
\\
पिकी मदन को जो जन जाने\\
भाग्यशाली वो खुद को माने\\
इन के लब पे सदा तराने\\
दिल में क्या है कोई ना जाने\\
गम को छुपा के जग को हसायें\\
जगत में इन का नाम \\
भागे दो गज़ निकले जान\\
\\
कल का छोकरा बुड्ढा हुआ है\\
अपने को समझे जवान\\
भागे दो गज़ निकले जान\\
भागे दो गज़ निकले जान\\
(तर्ज़ —देख तेरे संसार की हालत\\
एक मित्र के लिये लिखी थी। नाम बदल के\\
किसी के लिए गाई जा सकती है।
)
}\end{verse}
%\end{multicols}