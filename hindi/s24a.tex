%Chapter 1
\chapter{\texthindi{क्रम}}

\poemtitle{\texthindi{आध्यात्मिक}}

\settowidth{\versewidth}{This is the average line and so is this}
\begin{verse}[\versewidth]\texthindi{
\texthindi
भगवान\\
स्रोत\\
सर्वव्यापी\\
मोक्ष\\
दो दिन का मेहमान\\
असली रूप\\
खुदी\\
नाशुक्रा\\
दो राहें\\
जीवन का मकसद
}
\end{verse}

\poemtitle{\texthindi{परिवार}}
\begin{verse}[\versewidth]\texthindi{
दो नम्बर मकान\\
पहला मिलन\\
मुहब्बत\\
नाचूँ खुशियों से\\
लगता है मैं घर आ गया हूँ\\
मैं कहाँ फँस गया हूँ\\
अपनी मिट्टी\\
हमारा बचपन\\
दिल करता है\\
व्यापारी की इज़्ज़त\\
माँ\\
उर्मिल की कहानी\\
प्रेम लूथरा श्रद्धांजली\\
एफ़ जी टी मर गई\\
जीवन\\
अब नहीं तो कब\\
एक फूल की कहानी\\
पुनर्जन्म\\
सुखी जीवन\\
सरसराहट\\
निंदा\\
कल\\
वक्त\\
वक्त या पैसा\\
खुशी अन्दर है\\
बाँट के चीज़\\
शब्द शक्ति\\
खोखली हंसी\\
ये वक्त जाने कहाँ चला गया
}
\end{verse}

\poemtitle{\texthindi{ज़िन्दगी}}
\begin{verse}[\versewidth]\texthindi{
बात\\
नये पंछी\\
रौशनी की इज़्ज़त\\
इन्सान की इज़्ज़त\\
चाँद\\
दोस्त\\
नकली दोस्त\\
अतीत के भूत\\
सवेरा\\
आग में सुलगना\\
कच्चे घड़े\\
बच्चों की मुस्कुराहट\\
नया जीवन\\
छे फ़ुट का फ़ासला\\
साथी\\
घर लुटवाना\\
औरत\\
काश हम मिले न होते\\
काश हम बिछड़े न होते\\
जीवन पथ\\
दर्दे दिल\\
गुस्सा\\
एक हाथ की ताली\\
मकड़ी जाल\\
राजनेता\\
पुनर्मिलन\\
नज़रिया\\
किस्मत के धनी\\
रामायण सारांश में\\
महाभारत सारांश में\\
सुनामी\\
इक रब के कई नाम\\
घड़ी
}
\end{verse}

\poemtitle{\texthindi{हास्य}}
\begin{verse}[\versewidth]\texthindi{
बाँके लाल का ढाबा\\
पोकर\\
बीवी\\
पति\\
पति पत्नी की नोक झोंक\\
पैसे के दो रूप\\
शराबी\\
आधुनिक दीवाली\\
जॉनी का सर दर्द\\
जीवन\\
कोविड के दिन, जाऊँ तो जाऊँ\\
महँगे प्याज़\\
फ़ोन\\
स्टॉक्स\\
अमेरिका के कुछ काँटे\\
सत्तर्वाँ जन्म दिन\\
बुढ़ापा बीमारी मौत\\
बड़ती उम्र\\
मेरी उम्र\\
खोई जवानी\\
दोस्तों की नई तस्वीरें\\
बुढ़ापे के रंग\\
आल्ज़ाइमर्ज़\\
जीवन का खेल\\
बहुत देर\\
मोम के पुतले\\
जन्म मरण\\
मौत\\
यादों के खँडरात\\
दुआ\\
जीवन और मौत\\
जीवन का अन्त\\
समय की धूल
}
\end{verse}

\chapter{\texthindi{आध्यात्मिक}}

\poemtitle{\texthindi{भगवान}}
\begin{verse}[\versewidth]\texthindi{
बिन बुलाए मेहमान घर में नहीं आते\\
मैं कब से तैयार तुम ही नहीं बुलाते\\
दिल से बुलाओ छुपे भगवान चले आते\\
दिल से बुलाओ छुपे भगवान चले आते\\
\\
तेरे न्योते के इंतज़ार में आँखें बिछाये\\
बार बार सोचूँ कब नींद से जग जाये\\
गहरी नींद में तुम ने कितने जन्म गंवाये\\
ना धन से ना हीरे मोती से मुझे सजाये\\
बैठा सच्चे प्यार लगन की आस लगाये\\
\\
तेरी शोहरत ताकत पैसा मुझ से मिलता\\
मेरे हुक्म बिना पत्ता भी ना हिलता\\
ना कर लोभ गुमान क्रोध ईर्षा शिकायत\\
जितना कर्मों ने कमाया उतना ही मिलता\\
\\
प्यार से खिला रूखी सूखी खा लूँ\\
पैरों में आ जाये उठा गले लगा लूँ\\
है अंश मेरा क्यों खुद से जुदा बना लूँ\\
अपनी मैं छोड़ दे, तुझे अपने में मिला लूँ\\
अपनी मैं छोड़ दे, तुझे अपने में मिला लूँ
}
\end{verse}

\poemtitle{\texthindi{स्रोत}}
\begin{verse}[\versewidth]\texthindi{
माँग जहाँ से सब कुछ आये\\
माटी से क्यों आस लगाये\\
बीज अक्षर तेरे बीच रमा है\\
जो सारा संसार चलाये\\
माँग जहां से...\\
\\
कोई जन धन से महान कहाये\\
कोई तन से बलवान कहाये\\
सुंदर काया शव कहलाये\\
जिस तन से श्री राम सिधाये\\
राम ही धन है, राम ही शक्ति 2\\
राम ही बेड़ा पार लगाये\\
माँग जहाँ से सब कुछ आये\\
माटी से क्यों आस लगाये\\
माँग जहां से...\\
\\
जीवन एक हवा का झोंका\\
आज उठा है कल ना रहे गा\\
ये मेरा है वो मेरा है\\
वो तो रहे गा तू ना रहेगा\\
राम सदा थे राम सदा हैं\\
जुग जुग चाहे बीत ही जायें\\
माँग जहाँ से सब कुछ आये\\
माटी से क्यों आस लगाये\\
बीज अक्षर तेरे बीच रमा है\\
जो सारा संसार चलाये\\
माँग जहां से…
}
\end{verse}

\poemtitle{ \texthindi{ सर्वव्यापी } }
\begin{verse}[\versewidth]\texthindi{
देखूँ जिधर मैं तुझ को पाऊँ\\
जब जब तेरा ध्यान लगाऊँ\\
देख तेरी लीला महिमा मैं गाऊँ\\
जोड़ के हाथों को सीस झुकाऊँ\\
\\
देखूँ जिधर मैं तुझ को पाऊँ\\
जब जब तेरा ध्यान लगाऊँ\\
\\
कोई तोहे राम कहे कोई हरि पुकारे\\
वाहे गुरु अल्लाह यीशु नाम तिहारे\\
जिस नाम से भी तुझ को पुकारूँ\\
पल में तेरा दर्शन पाऊँ\\
देखूं जिधर…\\
\\
दरस तेरा उस को मिल जाता\\
जिस पे कृपा हो जाये तेरी दाता\\
चरणों में तुम रख लो मुझ को\\
दर दुनिया मैं छोड़ के आऊँ\\
देखूँ जिधर...\\
\\
पाप की गठरी ढो कर आया\\
लाया वही जो मैं ने कमाया\\
दे दो सहारा ओ मेरे मालिक\\
मैली चादर धो कर जाऊँ\\
\\
देखूं जिधर मैं तुझ को पाऊँ\\
जब जब तेरा ध्यान लगाऊँ\\
देख तेरी लीला को\\
महिमा मैं गाऊँ\\
जोड़ के हाथों को सीस झुकाऊँ \\
जोड़ के हाथों को सीस झुकाऊँ
}
\end{verse}

\poemtitle{ \texthindi{ मोक्ष } }

\begin{verse}[\versewidth]\texthindi{ 
पलक झपकते जीवन बीता\\
पंछी को उड़ जाना है\\
कौन है अपना कौन पराया\\
दो दिन का ये ठिकाना है\\
पलक झपकते…\\
\\
बचपन बीता आई जवानी\\
फूल ही फूल थे रुत मस्तानी\\
काल खड़ा देखे राहें तेरी\\
छोटी सी है ये ज़िंदगानी\\
राम से मन का मेल मिला ले x2\\
तन तो यहीं रह जाना है\\
पलक झपकते…\\
\\
जिन से मोह ममता कर बैठे\\
वो ना कभी तुम्हारे थे\\
जन्म जन्म का जिन से नाता\\
उन को ही क्यों बिसारे थे\\
भगवन तुझ से दूर नहीं है x2\\
एक ही बार बुलाना है\\
पलक झपकते…\\
\\
झूठी काया झूठी माया\\
मृग तृष्णा में क्यों भरमाया\\
राम स्वरूप सुनहरा पंछी\\
तन की आँख से देख ना पाया\\
बाहर माटी में तू ढूँढे x2\\
मन के बीच खज़ाना है\\
पलक झपकते…\\
\\
काम क्रोध मद मोह और माया\\
हथकड़ियाँ बन जायें गे\\
मात पिता सुत बीवी भाई बहना\\
साथ ना तेरे जायें गे\\
सच्चे कर्म और नाम राम का x2\\
साथ ही तेरे जाना है\\
पलक झपकते…\\
\\
गुरु और गुर की महिमा जानो\\
राम का रूप हैं तुम पहचानो\\
गुरु कृपा देखो दीप जला कर\\
राह दिखाये ओ अनजानो\\
तन से पूजो मन से ध्याओ x2\\
आत्म लीन हो जाना है\\
पलक झपकते…\\
\\
गुरु ने राम से मेल कराया\\
राम का निश दिन ध्यान करो\\
राम नाम की नाव में चढ़ कर\\
भव सागर को पार करो\\
जन्म मरण का खेल मिटा कर x2\\
मोक्ष तुझे अब पाना है\\
पलक झपकते जीवन बीता\\
पंछी को उड़ जाना है\\
कौन है अपना कौन पराया\\
दो दिन का ये ठिकाना है\\
पलक झपकते…
}
\end{verse}

\poemtitle{ \texthindi{दो दिन का मेहमान
}} \begin{verse}[\versewidth]\texthindi{ 
दो दिन का मेहमान रे तू\\
खुद को अब पहचान रे तू x2\\
कल तू आया कल है जाना\\
काहे करे अभिमान रे तू\\
दो दिन का मेहमान रे तू\\
\\
जिस को तू ने घर है समझा\\
ये तो एक सराये है\\
सदा नहीं कोई रहता इस में\\
इक आये इक जाये है\\
इक आये इक जाये है\\
राम शरण में तुझ को जाना\\
वहीं लगा ले रे ध्यान तू\\
दो दिन का मेहमान रे तू\\
\\
कौड़ी कौड़ी माया जोड़ी\\
लाख कमाए फिर भी थोड़ी\\
इस पैसे के लालच ने\\
सब रिश्तों की कड़ी है तोड़ी\\
सब रिश्तों की कड़ी है तोड़ी\\
हाथ तो खाली जाना है\\
झूठी बनाए क्यों शान रे तू\\
दो दिन का मेहमान रे तू\\
\\
जिस माटी ने तुझे बनाया\\
उस में ही मिल जाना है\\
जब तक तू है इस दुनिया में\\
कर्म भला कर जाना है\\
कर्म भला कर जाना है\\
दुखियों का दुःख बाँट ले बन्दे\\
जन्म का कर कल्याण रे तू\\
दो दिन का मेहमान रे तू\\
\\
दो दिन का मेहमान रे तू\\
खुद को अब पहचान रे तू\\
कल तू आया, कल है जाना\\
काहे करे अभिमान रे तू\\
दो दिन का मेहमान रे तू
}
\end{verse}

\poemtitle{ \texthindi{असली रूप
}}\begin{verse}[\versewidth]\texthindi{ 
मन मंदिर में घोर अंधेरा\\
जोत जले हो जाये सवेरा\\
मूँद के आँखें ध्यान लगा लो\\
तन तेरा है राम का डेरा\\
मन मंदिर…\\
\\
मंदिर मस्जिद गुरुद्वारा मन\\
दर दर भटके क्यों प्राणी जन\\
श्वास की धारा में बह कर देखो x2\\
कण कण में राम जी का बसेरा\\
मन मंदिर में…\\
\\
सुषमणा खोलो कुंडलिनी जागे\\
तन मन लागें कच्चे धागे x2\\
प्राणायाम से योग मिला लो x2\\
उस से असली जो रूप है तेरा\\
\\
मन मंदिर में घोर अंधेरा\\
जोत जले हो जाये सवेरा\\
मूँद के आँखें ध्यान लगा लो\\
तन तेरा है राम का डेरा\\
तन तेरा है राम का डेरा
}
\end{verse}

\poemtitle{ \texthindi{खुदी
}}\begin{verse}[\versewidth]\texthindi{
खुदी को मार दो खुद मरने से पहले\\
फिर देखो जीने का मज़ा क्या है\\
अंदर झाँक के देखो रब का रूप\\
इधर उधर भटकने में रखा क्या है\\
\\
यही रब मुझ में जो छुपा तुझ में\\
अलग नाम देने का फ़ायदा क्या है\\
दीन धर्म मज़हब इंसानों की हैं देन\\
असल को खिताबों से लेना क्या है\\
\\
जिधर भी देखो उस की ही सृष्टि\\
समझो उस बीच रमा क्या है\\
उस की सोच से तेरी बहुत छोटी\\
होता वही जो उस की रज़ा है\\
\\
जीवन डोर उसे थमा दे जो सारा संसार\\
चलाए\\
वही बनाए वही चलाए फिर मिटा के नया\\
बनाए\\
इंसान को खुदी की ज़रूरत क्या है\\
खुदी को मार दो खुद मरने से पहले\\
फिर देखो जीने का मज़ा क्या है\\
फिर देखो जीने का मज़ा क्या है
}
\end{verse}

\poemtitle{ \texthindi{नशुकरा
}}\begin{verse}[\versewidth]\texthindi{
गिनती खत्म हो जाती है\\
जब तेरी मेहरबानियाँ गिनता हूँ \\
आँख झुक जाती है शर्म से\\
जब और भी मिन्नतें करता हूँ \\
\\
भूला भटका नाशुकरा लोभी\\
फिर से भिखारी बन जाता हूँ \\
भूल खिलौने तोहफ़े सेहत\\
खुशी के नये साधन अपनाता हूँ \\
\\
जो मिला मुझे मेरी मेहनत थी\\
जो ना दिया गिला तुझ से\\
अपनों से ऊँचों को देख जलूँ\\
भूला सभी जो मिला तुझ से\\
\\
जब देखूँ अंधे को, कुछ पल\\
आँखों पे गरुर आ जाता\\
देखूँ शव, इक हल्का एहसास\\
खुद ज़िंदा होने का आ जाता\\
\\
सोचूँ दुःख बीमारियाँ मौत\\
रब ने बनाये दूजों के लिए\\
मैं तो सदा रहूँ गा ज़िंदा\\
हस्पताल शमशान दूजों के लिए\\
\\
फिर इक दिन कैन्सर या\\
दिल का दौरा पड़ जाता\\
इक बुलबुला हूँ सागर में\\
साफ़ दिखाई पड़ जाता\\
\\
तब सोचूँ कितना दिया तू ने\\
जिसे मैं ने नज़र अन्दाज़ किया\\
भाई बहन साथी घर छोड़े\\
रब सेहत को भुला बस\\
पैसे शान का नशा पिया\\
\\
आधी बन्द आँखें बेहोशी में\\
अंत ख्याल मुझे आता\\
पर किस्मत वाला तेरी मेहर से\\
जल्द ज्ञान पा जाता\\
क्या ?\\
गिनती खत्म हो जाती है\\
जब तेरी मेहरबानियाँ गिनता\\
आँख झुक जाती है शर्म से\\
जब और भी मिन्नतें करता
}
\end{verse}

\poemtitle{\texthindi{दो राहें
}}\begin{verse}[\versewidth]\texthindi{
दो राहें तेरे मन को मिलीं थी\\
एक को क्यों तू ने छोड़ दिया\\
एक को क्यों तू ने छोड़ दिया\\
\\
भूल गया तुझे जिस ने बनाया\\
भटका जहाँ उस की है माया\\
माया मृग छाया है भोले x2\\
उस के पीछे क्यों दौड़ लिया\\
एक को क्यों तू ने छोड़ दिया...\\
\\
सूरज जिस से रौशनी पाये\\
जो सारा संसार चलाये\\
उस दीपक से उस शक्ति से\\
मुख को क्यों तू ने मोड़ लिया\\
एक को क्यों तू ने छोड़ दिया ...\\
\\
दो राहें तेरे मन को मिलीं थी\\
एक को क्यों तू ने छोड़ दिया\\
एक को क्यों तू ने छोड़ दिया
}
\end{verse}

\poemtitle{\texthindi{ जीवन का मकसद 
}}\begin{verse}[\versewidth]\texthindi{
तर्ज़ — तू गंगा की मौज\\
है सदियों से ये सवाल चलता ही आया x2\\
काहे कुदरत ने इंसान जहाँ क्यों बनाया\\
\\
गीता में अर्जुन ने कृष्ण से पूछा\\
कृष्ण से पूछा\\
कल युग में शिक्षक ने गुरुओं से पूछा\\
गुरुओं से पूछा\\
रब ने बनाया तुझे प्रेम खज़ाना\\
प्रेम खज़ाना\\
गम अपना भूल तुझे जग को हँसाना\\
हर कोई अपना है ना कोई पराया\\
काहे कुदरत ने…\\
\\
किस्मत तू लिखे हाथों से अपनी\\
हाथों से अपनी\\
मिलता वो ही फल जो बोये तेरी करनी\\
बोये तेरी करनी\\
मन छोड़ बुद्धि से काम जो ले तू\\
काम जो ले तू\\
रब तेरे संग है अकेला नहीं तू\\
सफ़र ज़िंदगी में वो तेरा ही साया\\
काहे कुदरत ने इंसान जहां क्यों बनाया\\
है सदियों से ये सवाल चलता ही आया x2\\
काहे कुदरत ने इंसान जहाँ क्यों बनाया
}
\end{verse}

% Page 55 of 518
\chapter{ \texthindi{परिवार} }

\poemtitle{ \texthindi{दो नंबर मकान } }
\begin{verse}[\versewidth]\texthindi{
आओ सब मिल गायें गाथा\\
दो नंबर मकान की\\
सन पचास में बोली लगा कर\\
लगा दी बाज़ी जान की\\
मात पिता की जै बोलो\\
मात पिता की जै\\
मात पिता की जै बोलो\\
मात पिता की जै\\
लाला जी को दस एकड़\\
ज़मीं मिली ईनाम में\\
कुंदन बेटा बने गा डॉक्टर\\
करम ज़मीं के काम में\\
कुदरत के रंग किस्मत पलटी\\
करम मिले श्री राम में\\
छोड़ डॉक्टरी के सपने\\
कुंदन खेती के काम में\\
ना शिकवा ना गिला था कोई\\
चेहरे पर मुस्कान थी\\
आओ सब मिल गायें गाथा\\
दो नंबर मकान की\\
मात पिता की जै बोलो\\
मात पिता की जै\\
\\
सरगोधे से चली ये जोड़ी\\
पहुँची खानेवाले में\\
पिता बाईस के माता जी थी\\
अभी सोलहवें साल में\\
पिता जी ने मारा छक्का\\
सब से पहली बॉल में\\
क्रिकेट टीम के कैप्टेन सूरज\\
पहुँचे पहले साल में\\
रेलवेज़ का अफ़सर हो गा\\
शान हिंदुस्तान की\\
आओ सब मिल गायें गाथा\\
दो नंबर मकान की\\
मात पिता की जै बोलो\\
मात पिता की जै\\

सुदेश मोहिन्दर लाँघ ना पाये\\
बचपन की दीवार को\\
प्रेम कांता कंचन विरिंदर\\
शोभा दें संसार को\\
कृशन गिंदी शोकी ने कर दिया\\
पूरा लंबी कतार को\\
मात पिता ने सींची क्यारी\\
दे कर अपने प्यार को\\
जीवन धारा बहती जाये\\
खबर ना पाकिस्तान की\\
आओ सब मिल गायें गाथा\\
दो नंबर मकान की\\
मात पिता की जै बोलो\\
मात पिता की जै\\
\\
खानेवाले में धूप की गर्मी\\
नफ़रत की थी आग जली\\
दूर दर्शी हिंदू जनता\\
सदियों के घर से भाग चली\\
पिता जी नारंग ठक्कर भाई\\
छुपा प्यारी हर दिल की कली\\
दूर सबाथु ठंडी छाँव में\\
परिवार की नाव चली\\
मई सैंतालिस जान बचा कर\\
ढूँढी जगह विश्राम की\\
आओ सब मिल गायें गाथा\\
दो नंबर मकान की\\
मात पिता की जै बोलो\\
मात पिता की जै\\
\\
जहाँ भी देखो लाशें थीं\\
हर तरफ़ खून की होली थी\\
हा हा कार था आग और धुआँ\\
मार पीट की टोली थी\\
सदियों से जो भाई बहन थे\\
अब नफ़रत की बोली थी\\
पानीपत घर छीन लिया\\
जो जगह थी मुसलमान की\\
\\
आओ सब मिल गायें गाथा\\
दो नंबर मकान की\\
मात पिता की जै बोलो\\
मात पिता की जै\\
\\
जिधर भी देखो टैंट लगे थे\\
हर कोई घर की आस में\\
रेल लाइन के पार था प्यारा\\
इक घर खुले आकाश में\\
दो नंबर पर नज़र पड़ी\\
पिता जी की तलाश में\\
बीवी बच्चे यहीं पलें\\
हरियाली और प्रकाश में\\
मां ने ना की पैसा ना पल्ले\\
बोली दी मकान की\\
आओ सब मिल गायें गाथा\\
दो नंबर मकान की\\
मात पिता की जै बोलो\\
मात पिता की जै\\
आओ सब मिल गायें गाथा\\
दो नंबर मकान की\\
मात पिता की जै बोलो\\
मात पिता की जै\\
\\
(यह कविता मैं माता जी और पिता जी को\\
अर्पित करता हूँ। हम भारत के उस हिस्से में थे\\
जो अब पाकिस्तान में है।\\
तर्ज़—आओ बच्चो तुम्हें दिखायें झांकी हिंदुस्तान)
}
\end{verse}

\poemtitle{ \texthindi{पहला मिलन}}
\begin{verse}[\versewidth]\texthindi{
याद है जब हम पहली बार मिले थे\\
थामा पहली बार नर्म काँपता हाथ\\
उँगलियों ने चेहरे से बाल हटाए\\
आँख झुकी “आप बोलती बहुत\\
अच्छा हैं।”\\
\\
बिजली जिस्म में फैली होंठ थर्राए\\
छोटी छोटी बातों पे रूठ जाया करते\\
रात की नींद दिन का चैन गवाया करते\\
कई सुंदर सपने दिन में बनाया करते\\
हवा में रंग बिरंगे महल सजाया करते\\
हाथों पे तरह तरह तेरा नाम लिख\\
अपने नाम से जोड़ा करते\\
ना रिश्तों का बोझ ना पीछे का गम\\
खुश आपस में गिले ना करते\\
\\
उत्सुकता थी दिल में घबराहट काफ़ी थी\\
याद है जब हम पहली बार मिले थे\\
याद है जब हम पहली बार मिले थे
}
\end{verse}

\poemtitle{\texthindi{मुहब्बत}}
\begin{verse}[\versewidth]\texthindi{
मुहब्बत मानो शब्दों में लायी नहीं जाती\\
हकीकत जो ज़बान से समझाई नहीं जाती\\
फूल की खुशबू हवाओं में रम जाती\\
हल्की मुस्कान दिल को है भाती\\
दिलों की बात चेहरे पे लाई नहीं जाती\\
मुहब्बत…\\
\\
झुकी आँखें होंठ काँपते दिल का राज़ बताते\\
गाल गुलाबी माथे पसीना खुद से वो शर्माते\\
अपनो से क्या परदा बात छुपाई नहीं जाती\\
मुहब्बत…\\
\\
ना पैसे की इसे चाहत ना ढूँढे कोई बड़ा नाम\\
बंगला ना शोहरत इसे बस दिल से है काम\\
ये रब की मेहर, दौलत से कमायी नहीं जाती\\
मुहब्बत…\\
\\
दिल की बात दिल जाने कहने से क्या लेना\\
नज़र नीची ने कह डाला होंठों को सी लेना\\
रूह बात करे रूह से मुँह से बताई नहीं जाती \\
मुहब्बत मानो शब्दों में लायी नहीं जाती\\
हकीकत ऐसी ज़ुबान से समझाई नहीं जाती
}
\end{verse}


\poemtitle{\texthindi{नाचूँ खुशियों से}}
\begin{verse}[\versewidth]\texthindi{
नाचूँ मैं खुशियों से रात दिन\\
मुझे मेरा प्यार मिला\\
यार मिला दिल दार मिला\\
नाचूँ मैं खुशियों से रात दिन\\
\\
जब से मैं ने होश सम्भाला\\
तुम्हें ही चाहा तुम्हें ही माँगा\\
राह में ठोकर जब मोहे लागी\\
बाँह पकड़ कर तू ने सम्भाला\\
जो भी मैं ने माँगा रब से\\
उस से ज़्यादा मिला\\
नाचूँ मैं …\\
\\
दिल में उमंगें लब पे तराने\\
सपनों ने ली है अंगड़ाई\\
फूल खिले हैं इस बगिया में\\
देखूँ जिधर बहार है छाई\\
कलियों के अब दिन आये हैं\\
करूँ क्या रुत से गिला\\
नाचूँ मैं खुशियाँ से रात दिन\\
मुझे मेरा प्यार मिला\\
यार मिला दिल दार मिला\\
नाचूँ मैं खुशियों से रात दिन
}
\end{verse}

% page 79 of working version.docx
\poemtitle{\texthindi{लगता है मैं घर आ गया हूँ
}}\begin{verse}[\versewidth]\texthindi{
ये कविता मैं भारत को, अपनी मातृ भूमि को\\
अर्पण करता हूँ। जो भारत छोड़ आये हैं,\\
आओ घर चलते हैं।\\
\\
खुदगर्ज़ी से खुशहाली पाने देश था छोड़ा\\
मात पिता भाई बहनों से नाता था तोड़ा\\
उन सुनहरी यादों को ताज़ा कर लेता हूँ\\
कदम जहाज़ से बाहर जब रखता हूँ\\
लगता है मैं घर आ गया हूँ\\
\\
इमीग्रेशन क्लर्क में देखूं बाप की परछाई\\
सर ओढ़े आँचल में माँ लौट के वापस आई\\
सड़क पे खेलते बच्चों बीच खुद को ढूंढूँ\\
शोर गुल में बचपन के खोये यारों को ढूंढूँ\\
बाहर निकलते ऐसे नज़ारे देखता हूँ\\
लगता है मैं घर आ गया हूँ\\
\\
पड़ोसी को मिलने का न्योता ना चाहिए\\
हमारे घर आये हो, चाय तो पी के जाइये\\
दो रोटी और बना लें गे, खाना यहीं खाइये\\
ऐसी प्यार भरी बातें सुनता हूँ\\
लगता है मैं घर आ गया हूँ\\
\\
जहां बड़ों कि इज़्ज़त अभी भी होती\\
अंत समय अकेले रहने नहीं देती\\
जहां बच्चे बढ़े बूढ़ों को कंधा देते\\
सर झुका पैरों को छू दुआ हैं लेते\\
ऐसी पुरानी रीतें देखता हूँ\\
लगता है मैं घर आ गया हूँ\\
\\
पड़ोसी जब चाहे दरवाज़ा खटका सकता है\\
मिलने को खास वक्त ज़रूरी ना समझता है\\
फ़ासला उन के अपने घर का मिट जाता है\\
ऐसे बड़े परिवार को साथ साथ देखता हूँ\\
लगता है मैं घर आ गया हूँ\\
\\
बड़े आदमी ताऊ और चाचा औरत\\
 मासी कहलाती है\\
हर बच्चा बच्ची अपनी ही बेटा \\
बेटी कहलाती है\\
जहां रिश्तों का मिट जाता है फ़र्क \\
अपने पराये में\\
 घुल मिल प्यार से सब का रहना देखता हूँ\\
लगता है मैं घर आ गया हूँ\\
\\
रेल गाड़ी में तेज़ “चाय गरम” की पुकार\\
पोटली से निकलें परांठे आम का अचार\\
मुँह में पानी आ जाता है, माँग लूँ?\\
दिल में आये विचार\\
“आप भी दो बुर्की ले लो”\\
अनजान हमसफ़र कहता है\\
दो रोटी सफ़र में खाता हूँ\\
लगता है मैं घर आ गया हूँ\\
\\
लाउड स्पीकर सुबह सुबह रब के गीत सुनाये\\
राम, वाहे गुरु, अल्लाह की ऊँची महिमा गाये\\
कोयल की मधुर आवाज़ सोये सपनों से जगाये\\
सुरीली आवाज़ों में माँ बाप से जफ्फी लेता हूँ\\
लगता है मैं घर आ गया हूँ\\
\\
दीवाली में जगमग देश हुआ होली में बना सतरंग\\
लोहड़ी में सुंदर मुंद्रिये राखी भाई बहिन के\\
संग\\
नवरात्रे, कंजकें, दसहरा हर मौके पे होता\\
सत्संग\\
जब ऐसे अपने अनेक त्यौहार देखता हूँ\\
लगता है मैं घर आ गया हूँ\\
\\
वो उड़ती पतंगें, पैंचे लड़ाना\\
दीवारों छतों पर दीयों का सजाना\\
गुल्ली डंडा, पिट्ठू, कंचों की आवाज़\\
कौओं की कैं कैं का शोर मचाना\\
नल से खींच ठंडे पानी में\\
ठिठुर के नहाना\\
राह चलते ऐसे भूले नज़ारे देखता हूँ\\
लगता है मैं घर आ गाया हूँ\\
\\
हवा महके जगाये भूली बिसरी यादें\\
धूल में अपनी मिट्टी की खुशबू\\
माँ बाप की फरियादें\\
खस खस सी सुगंध पहली बारिश की\\
ठंडी हवा\\
पानी से पेड़ घर की दीवारें धुलते देखता हूँ\\
लगता है मैं घर आ गया हूँ\\
\\
माँ बाप दादा नानी की ज़िंदगी दोहरायी जाती\\
भाई बहनों की दौड़ धूप कहानी सुनायी जाती\\
ज़िंदगी की ऊँच नीच, हालातें बताई जाती\\
बचपन से आज की खुली किताब देखता हूँ\\
लगता है मैं घर आ गया हूँ\\
\\
और फिर \\
बिछड़ते वक्त कैदी आंसुओं का छुपाना\\
अनकहे फिर ना मिलने के ख्यालों का आना\\
वो हाथों का पकड़ना फिर न छुड़ाना\\
लम्बी प्यार भारी जुदाई, कंधे सहलाना\\
टपकते सुर्ख आंसुओं में सोचता हूँ\\
लगता है मैं घर छोड़े जा रहा हूँ\\
\\
करता हूँ खुद से वादा, जल्द दोहराऊँ गा\\
लगता है मैं घर आ गया हूँ\\
\\
सुनसान हैं गलियाँ ना बन्दों की आवाज़\\
अजनबी चेहरे भाषा अलग यहाँ के साज़\\
पड़ोसी पड़ोसी को ना जाने\\
अपनों को भी न पहचाने\\
सालों से साथ है इन का\\
फिर भी लगते हैं अनजाने\\
\\
बिन वजह रोज़ गोलियों का चलना\\
मासूम बच्चों बड़ों का बन्दूकों से मरना\\
ऐसी जगह से हर साल वापिस लौटता हूँ\\
तो लगता है मैं घर आ गया हूँ\\
मातृ भूमि में आ गया हूँ\\
पितृ भूमि में समा गया हूँ\\
मैं अपने ही घर वापस आ गया हूँ\\
लगता है मैं घर आ गया हूँ\\
लगता है मैं घर आ गया हूँ
}
\end{verse}

\poemtitle{\texthindi{मैं कहाँ फँस गया हूँ 
}}\begin{verse}[\versewidth]\texthindi{
जब में ने कविता-“लगता है मैं घर आ गया\\
हूँ” लिखी तो मेरे भाई, प्रेम, ने कहा “भाई, \\
पढ़ के मज़ा आ गया और बहुत अच्छा लगा। \\
तुम ने ऐसी चीज़ें भी देखी, झेली हों गी जो\\
तुम्हें परेशान करती हैं। उस को मध्य नज़र\\
रखते हुए कुछ लिखो”\\
\\
लगभग बीस साल पहले ये कविता लिखी\\
थी। कई चीज़ें अभी भी लागू हैं। अब तो\\
भारत कई तरह से इतना बदल गया है की ये\\
शायद अब नहीं लिख पाता। ये परिवर्तन देख\\
कर बहुत खुशी और गर्व होता है।\\
\\
डेंगू टाइफाइड और मलेरिया\\
मखी मछरों का है राज\\
दूध में ज़्यादा नल में कम पानी\\
कूड़ा गलियों का सरताज\\
खुली नालियां, हवा में बदबू\\
पुरानी गलियाँ वैसी ही आज\\
बचपन की ऐसी निशानियाँ\\
देखता हूँ\\
सोचता हूँ मैं कहाँ फँस गया हूँ\\
\\
सड़कों पर वही भीड़ भड़का\\
ट्रकों का जहां राज है पक्का\\
सब को शिक्षा देता पिछवाड़ा\\
बुरी नज़र वाले तेरा मुँह काला\\
माँ का आशीर्वाद जय जय माता\\
डिप्पर एट नाईट ओ के टाटा\\
ट्रकों से लटकते गन्ने खींचें बच्चे\\
लगता है मैं फिर बचपन में आ गया हूँ\\
\\
मेट्रो मैं चड़ूँ जेबों को बचाऊं\\
खाने को देखूं, खाऊं ना खाऊं\\
पेट खराब हो गा ज़रूर\\
डॉक्टरों के चक्कर में न फँस जाऊं\\
हस्पताल बने पैसे की मशीने\\
बैंक बैलेंस खाली ना कर जाऊं\\
रोज़ बचाव के तरीके ढूंढता हूँ\\
सोचता हूँ मैं कहाँ फँस गया हूँ\\
\\
बदन कांपता है यहाँ की सर्दी से\\
फेफड़े बंद हुए धूएँ और गर्दी से\\
चोरी डकैती बलात्कारी\\
दिल दहल गया आवारागर्दी से\\
अरे यारों, शिकायत करें किस से\\
डर लगता यहाँ खाकी वर्दी से\\
ऎसे दुःख भरे हालात देखता हूँ\\
सोचता हूँ मैं कहाँ फँस गया हूँ\\
\\
कुर्सी खातिर आया राम गया राम हैं अभी\\
नाम बदले उन के पर काली हरकतें हैं अभी\\
देश में छाया अन्धकार रौशन इन का घर\\
कानून आम आदमी पर ना इन्हें कोई डर\\
नये चेहरों पे राजनीति पुरानी देखता हूँ\\
सोचता हूँ मैं कहाँ फँस गया हूँ\\
\\
हर काम के लिए जानकारी या रिश्वत लाओ\\
यहाँ का दस्तूर खुद खाओ दूजों को खिलाओ \\
मंत्री से ले चपरासी का रास्ता पैसे का\\
पर नारे ज़ोर से लगाते दुराचार हटाओ!\\
ये सालों पुरानी तरकीबें देखता हूँ\\
सोचता हूँ मैं कहाँ फँस गया हूँ\\
\\
चलते हुए देखता हूँ सामने तो\\
थूक कुत्तों की देन पे फिसलता\\
देखूं जो नीचे कार से जा टकरता\\
घर से बाहर जब मैं निकलता\\
बचाऊं खुद को सामने या नीचे से\\
सोचता हूँ कहाँ फँस गया हूँ\\
\\
लिखा है 'गधा पेशाब कर रहा है' \\
मगर आदमी खड़ा है\\
बिन वजह भौंकता कुत्तों का झुण्ड \\
निकल पड़ा है\\
सड़क पे उलटी तरफ कार स्कूटर चल पड़ा है\\
लाल बत्ती में ड्राईवर बेधड़क निकल पड़ा है\\
ऎसे अजब नज़ारे देखता हूँ\\
सोचता हूँ कहाँ फँस गया हूँ\\
\\
पक्की टिकट थी ट्रेन और प्लेन की\\
उसे भी उन्हों ने रद्द कर डाला\\
बहाने बनायें पर सच तो ये था\\
मिनिस्टर या वी आई पी आने वाला\\
सफ़र बन जाता है सफ्फ़र जहां\\
डाँटना दुतकारना झेलता हूँ\\
सोचता हूँ कहाँ फँस गया हूँ\\
\\
इंडिया का सफ़र बहुत लम्बा लगता\\
टी एस ए, थ्रोमबोसिस से डर लगता\\
जेट लैग सात दिन इधर भी उधर भी\\
दो हफ़्ते का सफ़र चार का बनता\\
ऎसे कष्ट भरे दिन देखता हूँ\\
सोचता है कहाँ फँस गया हूँ\\
\\
अब गिनता दिन घर वापस जाने के\\
बिन सोचे हरी सलाद खाने के\\
दोहते दोहतियों को गोद बिठाने के\\
उत्तर हो या दक्षिण घर अपना मन भाये\\
परिंदे छोड़ पुराना घोंसला नया बसाये\\
हर जगह फूल और कांटे अपने अपने\\
राह जो चुनी वहीं अपने सपने सजाये\\
ऎसे ख्यालों में डूबा जहाज़ में बैठता हूँ\\
इक घर छोड़ मैं दूजे घर जा रहा हूँ\\
\\
वो भी मेरा ये भी मेरा जहाँ भी जाता हूँ\\
लगता है मैं घर आ गया हूँ
}
\end{verse}

\poemtitle{\texthindi{अपनी मिट्टी }}
\begin{verse}[\versewidth]\texthindi{
गुज़रे साल पचास छोड़े अपना देश\\
मिट्टी उस की अभी भी अपनी लगती है\\
मिट्टी उस की अभी भी अपनी लगती है\\
\\
पहले लोग कहते बेटा भाई अब अंकल\\
जिस नाम से मुझे पुकारें\\
ज़ुबान उन की मीठी लगती है\\
\\
हवा नज़ारे रस्में लोग लगें अपने\\
जैसे कभी ना बिछड़े थे\\
कोयल की धुन मीठी\\
कुत्ते की भौं भौं भी अच्छी लगती है\\
\\
छुपी यादें खोल आँखें लें अंगड़ाई\\
पेड़ की छाया गर्म लू से बचाती\\
पैसा एक ना पल्ले न थे हम गरीब\\
प्यार भरी भरपूर ज़िंदगी\\
हर कमी को पूरा करती है\\
गुज़रे साल पचास छोड़े अपना देश\\
मिट्टी उस की अभी भी अपनी लगती है\\
मिट्टी उस की अभी भी अपनी लगती है
}
\end{verse}

\poemtitle{\texthindi{हमारा बचपन
}}\begin{verse}[\versewidth]\texthindi{
हम आठ, साइकल एक\\
भरपूर थी हमारी खुशी\\
इक निक्कर कमीज़\\
चप्पल का जोड़ा खज़ाना था\\
हर जश्न मनाते धूम धाम से\\
प्यार भर देता था खुशी\\
\\
माँ बाप मुसकाते चुपके पीते ज़हर\\
शहद हमें पिलाते थे\\
खुद रह कर भूखा\\
मक्खन लदे पराँठे हमें खिलाते थे\\
इक बादशाही ज़िंदगी से\\
इक दिन में बने खानाबदोश\\
ना जाने कै से हंस के\\
राज गद्दी पे हमें बिठाते थे\\
\\
पेड़ों पे आम अमरूद नहीं\\
मीठा अमृत मिलता था\\
तंदूर से आग नहीं, नर्म सेक\\
दिल को सुकून मिलता था\\
\\
पैसा एक ना पल्ले\\
घर शीश महल दिखता था\\
खुशियों के फव्वारे गूंजते\\
बेफ़िक्र सुख चैन मिलता था\\
\\
नाम पानीपत पर अक्सर नल में पानी नहीं था\\
दो हाथ पम्प थे कसरत कोई गिला नहीं था\\
कभी आयी कभी गई \\
बिजली खेले आँख मिचौली\\
हाथ के पंखे, मोम बत्ती\\
कमियों का पता नहीं था\\
\\
कभी गुल्ली डंडा पिठ्ठु\\
कभी क्रिकेट की थी बारी\\
कँचे लुक्कन छुप्पी झूला\\
गुलेल से पथरी मारी\\
\\
पढ़ाई क्या चीज़ है\\
उस बारे कम सोचा था\\
अभी है बचपन खेलो कूदो\\
पढ़ने लिखने को उम्र है सारी\\
\\
हवा में पतंगें फल फूल ज़मीं पे\\
भर देते रंगीन नज़ारा\\
ना परवाह दूजों पास है क्या\\
घड़ा रहता भरा हमारा\\
आँगन दिन में खेल मैदान\\
मच्छरदानी में तारों नीचे\\
सोने का कमरा हमारा\\
\\
बचपन के अनमोल दौर की\\
तस्वीरें जब मन में खोलूँ\\
न गम न ज़्यादा सपने\\
बस वर्तमान ही काफ़ी था\\
\\
स्वामी जी शकुन्तला दर्शी\\
माँ का आशीर्वाद बरसता है\\
ऐसा सुंदर सुहाना बचपन\\
किस्मत वालों को मिलता है\\
\\
जैसे हवा में खुशबू, तालाब में\\
रंगीन कमल खिलता है\\
ऐसा सुंदर बचपन\\
किस्मत वालों को मिलता है\\
ऐसा सुंदर बचपन\\
किस्मत वालों को मिलता है
}
\end{verse}

\poemtitle{\texthindi{दिल करता है}}
\begin{verse}[\versewidth]\texthindi{
दिल करता है उड़ कर आऊँ\\
चंद घड़ियाँ तेरे साथ बिताऊँ\\
प्यार की गंगा निस दिन बहती\\
आ के अपनी प्यास बुझाऊँ\\
दिल करता है…\\
\\
जो बिछड़े हैं काश वो होते\\
प्यार के फूलों की माला पिरोते\\
मात पिता भाई बहनों को\\
दिल से कैसे भुलाऊँ मैं\\
दिल करता है…\\
\\
बचपन की यादें दोहरायें\\
भूले बिसरे गीत सुनायें\\
उन यादों को दिल से लगा कर\\
सालों साल बिताऊँ मैं\\
दिल करता है…\\
\\
खुशी की चादर में गम छुपाये\\
हर कोई अपना बोझ उठाये\\
एक अकेला थक जाये गा\\
आ के हाथ बटाऊँ मैं\\
दिल करता है…\\
\\
जन्म मरण तक साथ है अपना\\
चार दिनों का है ये सपना\\
सपनों को रंगों से भर दूँ\\
खुशियों के फूल चढ़ाऊँ मैं\\
दिल करता है…\\
\\
दिल करता है उड़ कर आऊँ\\
चंद घड़ियाँ तेरे साथ बिताऊँ\\
प्यार की गंगा निस दिन बहती\\
आ के अपनी प्यास बुझाऊँ\\
दिल करता है
}
\end{verse}

\poemtitle{\texthindi{व्यापारी की इज़्ज़त
}}\begin{verse}[\versewidth]\texthindi{
सामान बेच रही हूँ मैं इज़्ज़त तो नहीं\\
झुक रही हूँ ना समझना गैरत ही नहीं\\
\\
गरीब शायद पैसे से शराफ़त से नहीं\\
करती मेहनत पैसे खातिर चोरी नहीं\\
\\
घर बच्चों की परवरिश करना है धर्म\\
लोगों की गाली सुनने का शौक नहीं\\
प्यार से बात करो पैसे से ना तोलो मुझे\\
मुस्कान देने से दौलत घटती तो नहीं\\
\\
आवाज़ ऊँची कर इंसान बड़ा नहीं बनता\\
हल्का सर्द झोंका देता सुकून, तूफ़ान नहीं\\
\\
ना देखो मुझे शक की निगाह से\\
थमाओ हाथ में, फेंको नहीं पैसे\\
काम करती हूँ भिखारी तो नहीं\\
\\
सामान बेच रही हूँ मैं इज़्ज़त तो नहीं\\
झुक रही हूँ ना समझना गैरत ही नहीं\\
\\
माँ\\
सन्देश आया तेरे घर से\\
माँ की आँखें तेरी राह को तरसे\\
पिछले सावन वो बोली थी\\
अर्थी निकले गी अब इस घर से\\
सन्देश आया…\\
\\
तन से अपना दूध पिलाया\\
भूखे रह कर तुझे खिलाया\\
अपने मन की चाह मिटा कर\\
तेरा सपना सार कराया\\
शिकवा ज़बान पर कभी ना लाई\\
प्यार सदा आँखों से बरसे\\
सन्देश आया...\\
\\
मन ही मन वो घबराती थी\\
जल्द बुढ़ापा आये गा\\
बेटा डॉक्टर बन जाने पर\\
वक्त पे काम वो आये गा\\
उस के सपने टूट गये जब\\
पाँव निकाला तू ने घर से\\
सन्देश…\\
\\
माया खातिर जाल बिछाया\\
जाल में अपना आप गंवाया\\
मात पिता को छोड़ा तू ने\\
यादों पर भी पड़ गया साया\\
यादों के वो महल हैं खाली\\
महल निवासी निकले घर से\\
\\
सन्देश आया तेरे घर से\\
माँ की आँखें तेरी राह को तरसे\\
पिछले सावन वो बोली थी\\
अर्थी निकले गी अब इस घर से\\
सन्देश आया…\\
\\
उर्मिल की कहानी\\
\\
सुनो छोटी सी लड़की की लम्बी कहानी\\
सारी दुनिया से न्यारी प्यारी सी नानी\\
सुनो छोटी सी लड़की की कहानी\\
\\
राम पिता थे और सरला थी माता\\
छोटी सी गुड़िया के नंदी हैं भापा\\
नाना नानी से सुख प्यार बहुत पाया\\
आँख जब खुली न देखा बाप का साया\\
दिन सात बाद मिला उन्हें गंगा का पानी\\
सुनो…\\
\\
गुड़ियों से खेला और वायलिन बजाया\\
छोड़ पाकिस्तान लुधियाना घर बनाया\\
चीनी घर में थोड़ी पर बाँट इस ने खाई\\
यौवन में आई तो सूरज से की सगाई\\
नौ साल बाद पिया पानीपत का पानी\\
सुनो…\\
\\
चाँद सा चेहरा और आँखें हैं तारों सी\\
लौ से चमके डार्लिंग सूरज प्यारे की\\
पहले पहुंची निशी फिर आरती घर आई\\
साथ साथ करती थी बी एड की पढ़ाई\\
सिर पे ना ताज था सूरज बुलाये रानी\\
सुनो…\\
\\
मोम जैसा दिल चट्टान जैसा सर है\\
बाल धो के निकली तेल से वो तर है\\
आम चूसे निम्बू का आचार मन भाये\\
तीस नंबर घर में मेहमान सदा आये\\
इस शोर गुल में इनकी बीती जवानी\\
सुनो…\\
\\
छोड़ जोधपुर को दिल्ली घर बनाया\\
रेलवे कालोनी फिर आनंद विहार बसाया\\
ब्रिज इन की सौतन बैडमिंटन से प्यार था\\
बच्चों के ऊंचे नंबर दिलाने का विचार था\\
मॉडर्न स्कूल में वो बन गई मास्टरानी\\
सुनो…\\
\\
फिर क्या हुआ आरती?\\
\\
पार्किन्सन सिर सढ़िया का दिल इन पे आया\\
सुरमिल की ताकत ने दूर तक भगाया\\
जिस्म थक जाये अन्दर से ताकतवर है\\
परिवार का प्यार सेवा दवा का असर है\\
अंत हुए कष्ट बीमारी करे खत्म ज़िन्दगानी\\
सुनो छोटी सी लड़की की लम्बी कहानी\\
सारी दुनिया से न्यारी प्यारी सी नानी\\
सुनो छोटी सी लड़की की कहानी\\
(उर्मिल की श्रद्धांजली आरती की सहायता से
लिखी गयी है)
}
\end{verse}

\poemtitle{\texthindi{प्रेम लूथरा श्रद्धांजलि
}}\begin{verse}[\versewidth]\texthindi{
बूढ़ी हड्डियां कमज़ोर हुईं\\
जोड़ भी अब जुड़ने लगे\\
थक गया दिल धड़कते धड़कते\\
सांस भी अब रुकने लगे\\
\\
दौड़ धूप से झुलसा कोमल बदन\\
जिस्म कमज़ोर उठते नहीं कदम\\
जीने का कोई मकसद ना देखूँ\\
अंदर बाहर मैं थक गया हूँ\\
\\
कैंसर ने घर मुझ में बनाया\\
चोर के माफ़िक वो घुस आया\\
लाख दवा और दुआएँ करवाईं\\
फिर भी उस से जीत ना पाया\\
\\
एक म्यान में दो तलवारें रह ना पाएं\\
दुश्मन इक दूजे को सह ना पाएं\\
मैदाने जंग में हम ने की बहुत लड़ाई\\
उस कातिल से हम जीत ना पाए\\
\\
अब दिल करता है मैं सो जाऊँ\\
ऐसी नींद कि उठ नहीं पाऊँ\\
अपने मरने का डर नहीं लगता\\
बोझा अपनों पे ना बन जाऊँ\\
\\
अपने जाने का गम नहीं मुझे\\
डरता हूँ सोच तेरा चुप के रोना\\
बुरा शगुन तुझे कहेगी दुनिया\\
अकेले ज़िंदगी के बोझ का ढोना\\
\\
थोड़ा और जीने को मन करता\\
अपनों से दिल कभी नहीं भरता\\
कुछ और पल तेरा दामन ना छूटे\\
घुट गले लगाने को दिल करता\\
\\
काश उन संग वक्त बिताया होता\\
कल मिल लें गे जल्दी क्या है ?\\
काश ऎसा ख्याल आया ना होता\\
ज़िंदगी को और गले लगाया होता\\
\\
काश जिन्हें दुःख दिया उन्हें सताया ना होता\\
ना चिंता ना मुसीबत में घबराया होता\\
अपनों को इज़हारे मोहब्बत कराया होता\\
दुनिया को प्यार से और सहलाया होता\\
\\
प्यार लेन देन थी मेरी पहचान\\
दो चार दिन के संगी साथी\\
जैसे आये और गये मेहमान\\
अब फूलों की सेज बना हूँ\\
कल तस्वीर दीवारों की पहचान\\
\\
दिन होते लम्बे पर ज़िंदगी छोटी\\
इक इक पल गिनते बीता\\
उम्र दो पल में है खोती\\
\\
कल की है बात बचपन था जवानी थी\\
आज मैं ने दुनिया से रवानी की\\
\\
कुछ मीठी यादें कुछ शिकवे गिले\\
चंद दिन मेरी बातें हों गी\\
फिर इक कागज़ पे या किसी के दिल में\\
मेरे नाम की यादें हों गी\\
\\
ऐसे ख्यालों में डूबा खोया रहता\\
पूछने पर ज़बान से कह नहीं पाता\\
मेरे ख्याल मेरे संग ही जाने दो\\
जो पराये हैं वो ना समझें मेरी बात\\
जो समझें उन्हें आँखों से कह जाता\\
\\
कुछ तो अच्छा किया हो गा\\
जब प्यार हर तरफ़ देखता हूँ\\
जो दिया था दूजों को\\
अब वापिस लौटते देखता हूँ\\
कुछ आते करने आखरी नमस्ते\\
कुछ को अपने संग मरते देखता हूँ\\
\\
आंसुओं से आटा गूंधना\\
फिर रोटी का जल जाना\\
एक के लिए बनाना\\
फिर चुप बैठ अकेले खाना\\
\\
आँखें नम हो जाती हैं\\
तेरी अकेली ज़िंदगी सोचता हूँ\\
इस सोच से दरवाज़ा मौत का\\
बंद होने से रोकता हूँ\\
\\
पर करूं क्या आज तक\\
कोई जीत ना पाया काल से\\
पचपन साल मिलीं थी खुशियां\\
मुसकाना उसी ख्याल से\\
\\
मेरे हिस्से का खाना मेरा प्यार भी दुगने लुटाना\\
कल हमसफ़र, अब तुम में मैं हूँ\\
गुज़रे तू जिस भी हाल से\\
\\
जिस्म ना सही, रूह सदा तेरे साथ है\\
रहना सुखी हर हाल में तेरे सर पे मेरा हाथ है\\
जितने दिन मिले तुम्हें हँसते हुए बिताना तुम\\
यही दुआ मेरी राम से यही मेरी फ़रियाद है\\
\\
अपनी इनिंग अब पूरी कर ली मैं ने\\
जितनी लिखी उतनी रन बना लीं मैं ने\\
खुशियां बांटने से स्कोर की गिनती होती\\
फिर तो कई सेंचरी लगा दी मैं ने\\
\\
कोई तो क्लीन बाउल्ड या कौट आउट हुआ\\
\\
शुक्र है माँ बाप राम और राम शरणम् का\\
शशि अशित ज्योती दिशा तनुज सब का\\
मेरे संगी थे साथी थे इस खूबसूरत मेल में\\
सब को आउट होना है ज़िंदगी के खेल में\\
\\
अलविदा मैं सब को करता हूँ प्रेम से\\
ये मेरा आखरी खत है तुम्हारे प्रेम से\\
\\
फ़ॉण्डली प्रेम
}
\end{verse}

\poemtitle{\texthindi{एफ़ जी टी मर गई
}}\begin{verse}[\versewidth]\texthindi{
कोई खबर ना ज़िक्र है उस का\\
लगता है वो शायद मर गयी है\\
चर्चे सुनते चार दिन उस के\\
ज़िंदाबाद ज़ोर शोर के नारे\\
लूथरा खानदान इतिहास पन्नों में\\
शायद उस का ज़िक्र तो हो गा\\
एक या दूजी पीढ़ी पढ़ेगी मुस्कुराते\\
मीठी यादें खोने का फ़िक्र तो हो गा\\
अब मुलाकात होती चमकते फ़ोन पे\\
झुकी आँखें उँगलियों से\\
कौन करे तकलीफ़\\
घर काम छोड़ सफ़र करने से\\
अब वट्स ऐप रहे ज़िंदाबाद\\
हिंग लगे ना फटकड़ी\\
मुलाकात हो जाए\\
अब तो शब्द भी सिम्बल बने \\
करनी पड़ती कम बात \\
\\
भूले सुख जफ्फी हाथ सहलाने में\\
मुस्करा आँख से आँख मिलाने में\\
भूल गए मिल खेलना हँसना हँसाना\\
खुशियाँ बाँटना कंधे सहलाना\\
सुख दुख में आँ सुओं का बहाना\\
 \\
वक्त रुकता नहीं ज़माना बदल जाता \\
कुछ अच्छा बचा ज़्यादा खो जाता\\
अपने अपने में सब मग्न\\
परिवार का टीला धूल हो जाता\\
\\
डूब गई एफ़ जी टी वक्त के अंधेरे में\\
भूल गए गीत दिल करता है उड़ कर आऊँ\\
झूले पे चंद घड़ियाँ तेरे साथ बिताऊँ\\
दो नम्बर गाथा बोसा राम की मिठाई लाऊँ\\
\\
हम बेखुदी में तुम को पुकारे का गाना\\
दिन रात खेलना मिल जुल खाना\\
आम का पेड़ क्रिकेट ताश खिलाना\\
बिन वजह हँसी के फव्वारे लुटाना\\
\\
देखूँ उन यादों की खान में\\
सुनहरी मंच के कई खिलाड़ी\\
अलविदा कह रुला कर चले गए\\
कुछ जीवन की दौड़ धूप सह रहे\\
\\
जिन्हें ये तोहफ़ा मिला था मंच का\\
चार पीढ़ी कायम हैं मिलो मिलाओ\\
उन को इस अमृत का रस चखवाओ\\
\\
वक्त ने तब्दील किया जीवन हमारा\\
मंज़ूर है स्वीकार करना धर्म हमारा\\
\\
मौत किसी की हो खासकर अपनों की\\
इक आँसू तो आ जाता है\\
जब एफ़ जी टी का ख्याल आता है\\
चलचित्र का नज़ारा सामने आता है\\
पलकों से टपकता इक आँसू तो आ जाता है\\
पलकों से टपकता इक आँसू तो आ जाता है
}
\end{verse}

% page 160

\chapter{\texthindi{जीवन}}

\poemtitle{\texthindi{अब नहीं तो कब}}
\begin{verse}[\versewidth]\texthindi{
जीवन की रफ़्तार एक ही सब के लिए\\
ना चले ये तेज़ ना रुके किसी के लिए\\
छोटा हो या बड़ा राजा या रंक इंसान\\
पहुँचें एक मंज़िल जो कहलाये शमशान\\
\\
खुले हाथ से रेत जीवन से दिन निकलें\\
उम्र है छोटी लाखों सपने हैं दिल में\\
रंग से भर साकार सभी को कर लो\\
कल का सूरज आये या ना आये\\
सपना अधूरा रातों का रह ना जाये\\
जो करना है आज अभी ही कर लो\\
अब नहीं तो कब ?\\
\\
शोहरत पैसे के लालच ने बीवी बच्चे भुलाए\\
हर पल दुनियाँ में भटका नाम काम कमाए\\
खड़ा बुढ़ापा अगले चौराहे पे आस लगाए\\
बच्चे जल्द घर छोड़ अपनी राहों पे पड़ जायें\\
उन से खेलो हँसो हसाओ और दिल से सोचो\\
अब नहीं तो कब ?\\
\\
इस से पहले बीमारियाँ घर में बस जायें\\
जोड़ जुड़ें साँसें रुकें मुँह से निकले हाय\\
फूलों को दो पानी उन्हें सूखने से पहले\\
लोहे को बचा लो ज़ंग लगने से पहले\\
गुज़रा वक्त ना वापस आया कभी\\
जिस्म को संवार लो ढलने से पहले\\
अब नहीं तो कब ?\\
\\
जितनी उमंगें हैं दिल में अब पूरी कर लो\\
खुला आसमान देता तोहफ़े झोली भर लो\\
किस्मत से कुछ वक्त बचा है\\
ना दुःख दे जो करना अभी ही कर लो\\
सभी दोस्त भाइयों का यही है कहना\\
अरे भाई रुको, सुनो, जागो, गौर करो\\
अब नहीं तो कब ?\\
अब नहीं तो कब ?
}
\end{verse}

\poemtitle{\texthindi{एक फूल की कहानी}}
\begin{verse}[\versewidth]\texthindi{
उस की ज़ुबानी\\
\\
खुली हवा में मस्ती से मैं \\
इठलाता लहराता था\\
महक उभरती शोख लबों से \\
गीत मधुर मैं गाता था\\
\\
साथ थे मेरे संगी साथी \\
रंग बिरंगे और निराले\\
भँवरे तितली पीते रस उन का \\
जो छलकते किस्मत वाले\\
\\
ले के चुम्बन एक फूल का\\
 दूजे पर वो जाते थे\\
सात सुरों से मुझे रिझाने \\
\\
दूजे से सुंदर लगने के \\
नए नए साधन अपनाते\\
देख छवि पानी में अपनी \\
खुशी से इतराते शर्माते\\
\\
बहुत सुंदरता अच्छी भी है और बुरी भी\\
रंगों भरी जवानी अच्छी भी और बुरी भी\\
\\
चाहत भरा चुम्बन प्यार से \\
सहलाना अच्छा लगता था\\
अफ़सोस सुंदर चेहरा फूलों के \\
व्यापारी को अच्छा लगता था\\
\\
भरी जवानी रंगों से लदे \\
फूलों की तलाश में था\\
मुझ पर नज़र पड़ी पर \\
मुझे अभी एहसास ना था\\
\\
उस की काली नज़र ने \\
मुझ में शोहरत पैसा देखा\\
गुलदस्ते में खूब सजे गा \\
ये रंगीं सुंदर चेहरा\\
\\
मन ही मन उस कातिल ने \\
बुरी नज़र से सोचा था\\
बेखबर अनजान था मैं \\
कातिल में आशिक देखा था\\
\\
ज़ुल्मी ने चमकती कैंची निकाली झोले से\\
पकड़ गर्दन अलग किया मुझे मेरी माँ से\\
\\
एक ही झटके से दर्द दिया \\
लाखों सोये सपने तोड़े\\
दो पल खुशी की खातिर \\
मुझे और मेरे साथी जोड़े\\
\\
बेदर्दी ने बाँध रस्सी से \\
हम सब को कैदी बनाया\\
पानी भरे शीश महल में \\
बेचारों को खूब रुलाया\\
\\
शादी की खुशी मनाने \\
मेज़ों पर सब को सजाया\\
जिस की खातिर हम ने जान गंवाई \\
खून बहाया\\
वो मस्ती में माशूका से लिपटा झूमा लहराया\\
इक बार भी उस ने मुझे ना देखा\\
इक बार भी उस ने मुझे ना देखा \\
ना कुर्बानी मेरी\\
\\
शाम ढली फिर रात हुई धीमे से बात हुई मेरी\\
"बहुत सुंदर हैं ना ये फूल, बहुत महंगे हों गे"\\
मन रोया सुन ज़िंदगी के सपने पैसे में तुलते\\
\\
मेरा रोना किसी ने देखा ना दुःख समझ पाया\\
खाना पीना खत्म शुक्रिया कोई कह ना पाया\\
\\
किस्मत वाले गए किसी संग घर सजाए\\
मेरे जैसे कुचले बेचारे कचरे में समाए\\
क्या सोचा था कितने सपने दिल में थे बनाए\\
\\
अपनी ज़िंदगी रंगीन कई महीनों खिलें गे\\
इक संसार नया हो गा बीज बच्चे मिलें गे\\
\\
कोई किस्मत से लड़ नहीं पाता\\
लिखा कोई मिटा नहीं पाता\\
अब दम तोड़ रहा हूँ कूड़े के बर्तन में\\
\\
पानी में बहते मिले मेरे आँसू\\
पर दिल से दुआ देता हूँ\\
खुश रहें जिन कि खातिर\\
मेरा कत्ल हुआ खून बहा\\
लम्बी उम्र हो उन की\\
बेवक्त ना उन को कोई काटे\\
\\
आँखें बन्द किए बेहोशी में\\
बस यही दुआ देता हूँ\\
लम्बी उम्र हो उन की\\
बेवक्त ना उन को कोई काटे\\
बेवक्त ना उन को कोई काटे
}
\end{verse}

\poemtitle{\texthindi{पुनर्जन्म}}
\begin{verse}[\versewidth]\texthindi{
ये रिश्ते भी इक मौसम हैं\\
हर पल हर मास बदलते\\
साथ साथ दिन रात गुज़ारे\\
अब अपनी राह पे चलते हैं\\
\\
कल कलियाँ अब फूल खिले\\
कड़ी धूप ने इन्हें कुम्हलाया\\
ना हुई बरदाश्त कड़क गर्मी\\
घबरा के पतझड़ आया\\
\\
बिखर गये हरे लाल और पीले\\
उतरे मस्त जवानी के रंग\\
हुए बेरंग फिर भी सिमट कर\\
चिपके रहे वो मां के संग\\
\\
आया सर्द हवा का झोंका\\
रिश्ते सब कमज़ोर हुए\\
कोई पहले तो कोई पीछे\\
सब के सब झँझोर हुए\\
\\
सजा खूब बदन गुलशन का\\
हल्के रंगीन पत्तों से\\
रोयें बहुत ऊपर से शाखें\\
ले के आंसू शबनम से\\
\\
सुबह की पहली किरण देख\\
भेजें असीसें टिप टिप रे\\
“बहुत ही अच्छा लगता था\\
जब साथ सभी तुम थे मेरे\\
\\
रब से माँगूँ तेरी खुशियाँ\\
रहना तुम सदा आबाद\\
बदलना जीवन का दस्तूर\\
रखना हमेशा ये तुम याद\\
\\
ये सुंदर चमचमाता बदन\\
इक दिन ज़रूर ढल जाये गा\\
आहिस्ता आहिस्ता खामोशी से\\
तेरा रंग बदलता जाये गा\\
\\
आये गा इक दिन बाग का माली\\
जब तुम सूख जाओ गे\\
बनाये गा इक ढेर तुम्हारा\\
मेरे सामने जल जाओ गे\\
\\
जिन्हें छुपाया था आँचल में\\
अब वो खाक बन जायें गे\\
बदलियाँ बन आंसू मेरे\\
चिता तुम्हारी सहलायें गे\\
\\
ना घबराना ना रोना बच्चो\\
जड़ें तुम्हारी राह देखें गी\\
मिला के पानी बारिश का\\
तुम्हें सीने में छुपा लें गी\\
\\
बाद कड़क सर्दी के ज़रूर\\
हम फिर से मिलें गे\\
नये सावन में नये रूप में\\
हम सब हंस के खिलें गे\\
\\
सदियों से ये खेल कुदरत का\\
चलता जाये गा\\
जो आया था चला गया\\
फिर चोला बदल के आये गा\\
\\
ना फूलों में ज़्यादा खुश होना\\
ना पतझड़ में रोना तुम\\
दिल से कभी भुला ना देना\\
मेरा ही इक अंग हो तुम\\
कभी पास तो दूर कभी\\
पर हरदम मेरे संग हो तुम\\
कभी पास तो दूर कभी\\
मेरा ही इक अंग हो तुम\\
मेरा ही इक अंग हो तुम
}
\end{verse}

\poemtitle{\texthindi{सुखी जीवन}}
\begin{verse}[\versewidth]\texthindi{
बीते कल के गम दफ़ना कर\\
जोड़ो मीठे पलों की यादें\\
जीवन नए रंग दिखलाता\\
ना देखो कल के टूटे धागे\\
\\
कोई रहता अतीत में डूबा\\
कोई सीखता उस की भूलों से\\
कदम वही जो बढ़ते आगे\\
खुले आँख सिर्फ़ देखे आगे\\
\\
डूबता सूरज लाए अँधेरा\\
काट खाए जीवन का इक दिन\\
सुबह की किरण लाए नया दिन\\
गहरे सपनों से जब हम जागे\\
रब ने अपनी गिनती कर के\\
बाँटे कुछ दिन हम सभी को\\
उन में हँसी भरो या रोना\\
खुशी अब की या कड़वी यादें\\
\\
किसी बच्चे के आँसु पोंछो\\
किसी बूढ़े को दे दो सहारा\\
मज़ा देना लेने से बेहतर\\
दूजों की खुशी से अपने गम भागे\\
\\
पालती पोस्ती माँ बच्चों को\\
 अपने सारे दुःख दर्द भुला के\\
कड़वा सच है उन से छुपाती\\
उन का बचपन स्वर्ग बना दे\\
\\
इच्छा लालसा का अंत नहीं\\
इक मारो दूजी जग जाती\\
इन से दूर ही रहना सुखमय\\
समझें ज्ञान ज्योत जब जागे\\
इन से दूर ही रहना सुखमय\\
समझें ज्ञान ज्योत जब जागे
}
\end{verse}

\poemtitle{\texthindi{सरसराहट}}
\begin{verse}[\versewidth]\texthindi{
पत्ते हिले पेड़ों में सरसराहट\\
हिरण ने नज़र अन्दाज़ किया\\
छुपा शेर झपटा पल में\\
रात का खाना हज़म किया\\
\\
मदिरा पी के कार चलाई\\
दोस्तों ने बहुत मना किया\\
जवानी में किस ने देखी मौत\\
रोते बाप ने हाथों आग लगाई\\
\\
टेनिस खेलते एक दोस्त ने\\
अंदर बॉल को बाहर कहा\\
ना सोचा, उस संग धंधा किया\\
अब ना पैसा ना दोस्त रहा\\
\\
मैं सिगरेट नहीं पीता मना किया\\
मेरी लिये एक ले लो बोला दोस्त\\
सरसराहट ना समझा ना पहचाना\\
चल बसा घर वालों को रुला दिया\\
\\
रब ने सही आवाज़ सब में छुपायी\\
काम क्रोध मद लोभ की धूल छाई\\
आवाज़ बचपन की जवानी ने खाई\\
अब वो लगे सरसराहट की परछाई\\
\\
गर जीवन को सही राह पे लाना\\
साँसों में हो लीन रब को पाना\\
रामायण गीता से धूल हटा कर\\
सरसराहट की आवाज़ फिर जगाना\\
सरसराहट की आवाज़ फिर जगाना
}
\end{verse}

\poemtitle{\texthindi{निंदा}}
\begin{verse}[\versewidth]\texthindi{
निंदा करना फ़र्ज़ उन का \\
चुगली उन की जात\\
दूजों ने क्या पहना किया\\
या कह दी कोई बात\\
\\
मुँह पे बोलें शहद से मीठा \\
तारीफ़ के पुल बनायें\\
ज़बान से उन को काटते\\
 छुप पीठ पे करते घात\\
\\
निंदक के हैं कान दो\\
 मुँह चार से करते बात\\
उगलें दुगना जो सुने \\
ज़बान चले दिन रात\\
\\
साँप का काटा बच सके\\
 इन का मर जात\\
निंदक निर्मोही ना छोड़ें \\
अपने माँ और बाप\\
इन के सामने भगवान भी \\
खा जाता है मात\\
\\
भेड़ की खाल में भेड़िया\\
 कई ऐसे हैं यार\\
जो दिखते वो हैं नहीं\\
 ना करना एतबार\\
ऐसे दोस्त से दुश्मन भले\\
 करें सामने वार\\
दूर से पहचान हो \\
ढोंग ना नकली प्यार\\
\\
दूजों को गिराना धर्म है \\
करते निश दिन पाप\\
चुस्कियाँ ले मसाले लगा \\
बड़ा के करते बात\\
निंदा दूजों की तुम से करें\\
 तेरी औरों के साथ\\
\\
निंदक से दूर रहो ना जाने \\
कब करे ये घात\\
निंदक से दूर रहो ना जाने \\
कब करे ये घात
}
\end{verse}

% Somthing goes wrong here and becomes bold text
\poemtitle{\texthindi{कल
}}\begin{verse}[\versewidth]\texthindi{
कल के पल में बहुत ही बल है\\
वर्तमान खा जाता है\\
कल की यादें कल की आशा\\
इस पल में रंग भर जाता है\\
\\
बीता कल या आने वाला\\
इस पल के दो चेहरे हैं\\
इस पल के फूलों से सजे\\
दो अलग अलग सेहरे हैं\\
\\
कल ने दोस्तों की मिठास\\
दुश्मनों की तलवार छुपाई\\
कल ने मेरा आज बनाया\\
जीने की रीत सिखलायी\\
\\
कल में माँ बाप भाई बस्ते\\
मेरा मासूम भोला बचपन\\
कल को गर खो दिया\\
खो जाए मेरा अपनापन\\
\\
आने वाले कल में सोए सपने\\
आशाओं के नक्शे रहते हैं\\
कल के गर्भ में पलते पनपते\\
चाहत के झरने बहते हैं\\
\\
बीते कल से सीख समझ\\
इस पल को जीता हूँ\\
कल मीठा करने खातिर\\
ज़हरीले आँसू पीता हूँ\\
\\
बीता कल या ये पल\\
पलक झपकते बुझ जाता\\
वही दीपक काठी बन\\
कल की राह दिखाता\\
\\
ये पल कठिन या मधुर सुहाना\\
इक पल में बीता कल बन जाये\\
धुँध है आकाश की\\
जीवन की तपस से जल जाये\\
\\
जो बीत गया ना लौट के आए\\
आने वाला कल आए ना आए\\
छोड़ उन्हें जीना सीखो इस पल में\\
यही अतीत यही भविष्य बन जाए\\
यही अतीत यही भविष्य बन जाए
}
\end{verse}

\poemtitle{\texthindi{वक्त
}}\begin{verse}[\versewidth]\texthindi{
वक्त इक पंछी, उड़ता झूमता \\
फिर पंख समेट के गिर जाता\\
वक्त इक दरिया, उछलता इठलाता\\
समन्दर में खो जाता\\
\\
वक्त इक झोंका, छूता जिसे \\
उसे हिला के शांत हो जाता\\
वक्त आवाज़, बोलता गरजता चिल्लाता\\
चुपके सन्नाटा बन जाता\\
\\
वक्त इक फूल, रंग भरा खिलता इतराता\\
शाम ढली मुरझा जाता\\
\\
वक्त इक साँस, मेरा हक है उस पे आज\\
कल दूजे का हो जाता\\
\\
मैं पंछी दरिया हवा आवाज़ साँस का झोंका\\
इक पल का मालिक\\
नादानी से समझा अपना\\
वक्त यहीं था यहीं रहे गा\\
मेरा वजूद खत्म हो जाता\\
फिर याद बन रह जाता\\
फिर याद बन रह जाता
}
\end{verse}

\poemtitle{\texthindi{वक्त या पैसा
}}\begin{verse}[\versewidth] \texthindi{
है जीवन में किस की कमी\\
वक्त की या पैसे की\\
खोया पैसा फिर कमा लो\\
बीता वक्त जैसे खोया मोती\\
\\
रिश्ते भाईचारे यारी की इमारत\\
वक्त की नींव पे कायम होती\\
पैसा खोखली दीवार दिखावी\\
बिन वक्त गुज़ारे गिरती रोती\\
\\
मिल के बैठना सुख दुःख बाँटना\\
वक्त बिताने से है होता\\
पैसा सिर्फ़ ज़रूरत\\
वक्त जीवन के हीरे मोती\\
\\
जाने बाद भी पैसा रिश्ते मिटाता\\
अपनों की दीवार तोड़ गिराता\\
मीठे वक्त की यादें उस की बातें\\
कई पुश्तें उन की माला पिरोता\\
\\
पैसे से चीज़ खरीदी जाती\\
खुशी नहीं हँसी नहीं ना सेहत\\
हाथ पकड़ना गले लगाना\\
सेहत बनानी वक्त से होती\\
\\
वक्त है तेज़ बहती धारा\\
लाख रोकने पर रुक नहीं पाती\\
पैसा जीने का साधन जीना नहीं\\
वक्त की नादिया इसे डुबोती\\
\\
उम्र गुज़ारी पैसा जोड़ते\\
अनमोल वक्त की ना परवाह\\
वक्त अब लम्बा करने खातिर\\
पैसे को दी नामुमकिन चुनौती\\
\\
रुको सच्चे दिल से सोचो\\
पैसा चाहे सर चढ़ के बोले\\
वक्त के सामने हार है होती\\
वक्त के सामने हार है होती
}
\end{verse}

\poemtitle{\texthindi{खुशी अंदर है}}
\begin{verse}[\versewidth]\texthindi{
खुशी शांति नसीब बाहर नहीं मिलते\\
किस्मत के धनी गंदे तालाब में खिलते\\
कमल फूल को साफ़ पानी नहीं ज़रूरी\\
अंदरूनी बल से कीचड़ में रंग भरते\\
\\
अपने दुःख लोगों को बार बार सुना लिये\\
किसी ने दोषी कहा किसी ने कंधे दिये\\
उन कच्ची दीवारों का सहारा ना ढूँढ\\
सुख शांति सुकून अपने अंदर मिलते\\
\\
लोगों को दर्द बताओ कई सुनते नहीं\\
कई कहते अपने पास रखो\\
घर भरा है मेरा दुःखों से\\
खाली नहीं तेरे कचरे के लिये\\
नज़र छुपा कह अपनी राह चलते\\
गर जीवन को सुखी बनाना है\\
तो अपना दर्द छुपा\\
 पी ले अपने आँसू\\
दूजों के नैन सुखा\\
\\
लोगों की दर्द भरी कहानियाँ\\
सुन उन को सुलझा\\
कर किसी और का घर रौशन\\
खुद रब करें तेरा घर जग मग\\
खुशी से रब संग दीवाली मना\\
खुशी से रब संग दीवाली मना
}
\end{verse}

\poemtitle{\texthindi{बाँट के चीज़ 
}}\begin{verse}[\versewidth]\texthindi{
बाँट के चीज़ और भी स्वाद हो जाती\\
चीज़ का कुछ नहीं जाता\\
इज़्ज़त उस की दुगनी चौगुनी हो जाती\\
\\
घर अपना सभी सजाएँ फूल लगाएँ\\
जो पड़ोसियों का संवारें सजाएँ\\
इस से बड़ी ख़ुशी पायी नहीं जाती\\
दौलत अपने पे खर्चना बहुत आसान\\
उसी पैसे का तोहफ़ा दे कर \\
लाखों बच्चों की भूख मिटाई जाती\\
\\
फल खुद को मीठा बनाता फलने के लिए\\
दूसरे खा के बीज दूर पहुँचाएँ\\
खुदगर्ज़ी में भी औरों की भलाई की जाती\\
\\
ज़रूरत से ज़्यादा दिया काबिल समझा\\
लाखों जीवन की गहराइयों में डूब गए\\
धन बाँटनें में रब की मेहरबानी पायी जाती\\
बाँट के चीज़ और भी स्वाद हो जाती\\
चीज़ का कुछ नहीं जाता\\
इज़्ज़त उस की दुगनी चौगुनी हो जाती\\
इज़्ज़त उस की दुगनी चौगुनी हो जाती
}
\end{verse}

\poemtitle{\texthindi{शब्द शक्ति
}}\begin{verse}[\versewidth]\texthindi{
शब्दों की मंडी में लफ़्ज़ हैं लाखों\\
आओ उठा लें जिसे भी चाहें\\
कोई काँटों के साथ जुड़े\\
कोई फूल गले में डाले बाहें\\
\\
कोई लोगों में बाँटे खुशियाँ\\
कोई घाव दे काँटों से ज़्यादा\\
कोई दुःख को दुगना चौगुना कर दे\\
कोई उस को कर दे आधा\\
\\
शब्द ऐसे भी देखता हूँ जो\\
रोते दुखियों के आँसू सुखाएँ\\
कुछ ऐसे चुभते हैं जो\\
खुश हंसते लोगों को रुलाएँ\\
\\
दो शब्द तारीफ़ के पिछड़ों को\\
फिर चलना सिखाएँ\\
जली कठोर निकली होंठों से बातें\\
उम्र भर कैसे बिसराएँ\\
\\
मीठा शब्द इक अजनबी को\\
महबूब जीवन साथी बनाए\\
बेसमझी से गुस्से में कहे शब्द\\
ना जुड़ने वाली दीवार बनाए\\
\\
पीठ पीछे कही कड़वी बातें\\
जब उन तक पहुँच जाएँ\\
सालों के बनाए रिश्ते नाते\\
पल में टूटें फिर जुड़ ना पाएँ\\
\\
कुछ कलियों में छुपा के\\
काँटे चुभाया करते हैं\\
दिखा के सब्ज़ बाग\\
झूठ से लुभाया करते हैं\\
\\
कुछ बेवक्त बेहूदा कहे शब्द\\
लोगों को परेशान करते\\
कहने वाले को ख़बर नहीं\\
सुनने वाले दिल ही दिल मरते\\
\\
“चिपके गाल वज़न कम\\
कमज़ोर हो गये हो”\\
ऐसी बातें सुन अच्छा होते\\
मरीज़ को फिर से बीमार बनायें\\
\\
कुछ शब्द हैं जिन्हें बिन कहे\\
लब आँखें बेहतर समझाएँ\\
कई ऐसे हैं जिन्हें कहने पर\\
खुद आँखें शर्म से झुक जाएँ\\
\\
कुछ शब्द जिन्हें कहना चाहता हूँ\\
वो सुनने वाले ही ना रहे\\
वो कैदी बन चुपचाप मन में तड़पे\\
फिर आँसुओं में जा बहे\\
\\
शब्द बहुत ही शक्तिमान\\
दुनियाँ को बदल डाले इक इंसान\\
शब्दों को तोलो ये वापिस ना आएँ\\
जैसे तीर ने छोड़ा कमान\\
शब्दों को तोलो ये वापिस ना आएँ\\
जैसे तीर ने छोड़ा कमान
}
\end{verse}

\poemtitle{\texthindi{खोखली हंसी
}}\begin{verse}[\versewidth]\texthindi{
हँस के मिले, अंदर खुश ज़रूरी तो नहीं\\
हँसी के परदे में लाखों गिले घर बसाते\\
\\
दिल ना चाहे मिलना नज़र मिलने से कतराती\\
रस्में ज़माने की निभाने चले आते\\
गुज़रे कल के घाव हैं ताज़ा नज़र ना आते\\
अंदर कैदी बन, रिहा हो ना पाते\\
\\
आड़ दोस्ती की ले खंजर चलाये घाव दिये\\
अनजाने सहते रहे अब जान के सह ना पाते\\
बेखबर रहना था बेहतर खुशी के लिये\\
देख पानी में मक्खी प्यासे भी पी ना पाते\\
\\
बीता कल मर गया पर कड़वी यादें ज़िंदा\\
बुरे ख्याल राहे दोस्ती में पत्थर बन जाते\\
झूठी हँसी टूटे रिश्ते की लाज बनी\\
असलियत दिल की आँखें छुपा ना पाते\\
\\
घायल दिल कल भुला अब जीना चाहे\\
गड़े मुर्दे की बदबू नये फूल दबा ना पाते\\
\\
समझौता करते हैं बची ज़िंदगी के लिए\\
सुकून पाने खातिर हर बोझ उठाये जाते\\
\\
सुनसान अकेलेपन से खोखला संग बेहतर\\
मजबूर दिल चुपके रो के ज़हर पिये जाते\\
\\
\\
\\
फिर रब से दुआ की दुःखों का अंत माँगा\\
उस ने दर्द जाना सच्चे मन की पुकार सुनी\\
\\
अब कृपा सहारे शांत दिन गुज़ारे जाते\\
दिन रैन प्रभु का धन्यवाद किये जाते
}
\end{verse}

\poemtitle{\texthindi{वक्त चला गया
}}\begin{verse}[\versewidth]\texthindi{
ये वक्त जाने कहाँ चला गया\\
अभी तो जीना सीखा था\\
काम क्रोध ईर्षा मद छोड़\\
प्यार का अमृत पीना सीखा था\\
\\
आज व्यस्त हूँ कल कर लूं गा\\
बिछड़ों से नाता जोड़ूँ गा\\
अपनों को मिल सुख दुःख बाँटना\\
गले लगाना सीखा था\\
ये वक़्त...\\
\\
सालों से लगाये पौधे फल फूल खिले\\
अब रसीली खुशबू पीना सीखा था\\
चलना सीधी राह पे सीखा\\
दिल खोल के जीना सीखा था\\
ये वक़्त…\\
\\
कल खेल शुरू आज हुआ तमाम\\
ऐसा कभी ना सोचा था\\
उड़ता बादल\\
डूबता सूरज जीवन है \\
ऐसा कभी ना सोचा था\\
ये वक़्त...\\
\\
\\
\\
खुद को माफ़ औरों को माफ़\\
उन से माफ़ी माँगना सीखा था\\
बुलबुला पानी का पल में ओझल\\
काल का कांटा तीखा था\\
ये वक्त…\\
\\
इक इक पल की टिक टिक\\
आखरी पल का ऐलान करे\\
छोटी चीज़ें ज़िंदा रहते\\
बेवजह हमें परेशान करें\\
ना लाए ना ले जाना है\\
कई जन्मों का सामान भरें\\
दिलो दिमाग घर हल्का कर\\
पंछी की तरह उड़ना सीखा था\\
ये वक्त…\\
\\
जी भर जियो अपनों दूजों को मिलो\\
गले लगाओ कल ना रहें गे\\
इक तरफ़ा पगडंडी के सब राही\\
वापिस लौट बीते पल ना मिलें गे\\
ये वक्त…\\
\\
बारिश का कतरा मिट्टी में समाता\\
नाम निशान भी नहीं रहें गे\\
बुज़ुर्गों हालातों को देख कर\\
जीने का तरीका सीखा था\\
\\
ये वक्त जाने कहां चला गया\\
अभी तो जीना सीखा था\\
ये वक्त जाने कहां चला गया\\
अभी तो जीना सीखा था
}\end{verse}

\poemtitle{\texthindi{ज़िंदगी
}}\begin{verse}[\versewidth]\texthindi{
ज़िंदगी ऐसे मोड़ पे आई\\
शिकवा गिला ना शिकायत\\
मन में शांत मुस्कान छाई\\
\\
जिन्हें नौ मास कोख में पाला\\
दिन का चैन रात की नींद गंवाई\\
उफ़ ना की तन मन लुटाया\\
अब उन्हीं से सहारा हिम्मत पाई\\
जिन्हें सोचा बुढ़ापे का सहारा\\
वो सुख दुख में मेरे साथ हैं\\
भँवरे निकले इक फूल की आशा\\
कलियाँ हज़ारों हाथ में आईं\\
\\
गलतियाँ पहचान उन से माफ़ी पाई\\
बचे दिन रहने की रीत समझ आई\\
मैं का भूत सर पे था सवार\\
कृपा से उसे गिरा मन में शांति पाई\\
\\
जितना वक्त बचा हमारा\\
हँसी खुशी गुज़ारें गे हम\\
कल के गम भूलें दफ़ना के\\
जीते जी उन से रिहाई पाई\\
\\
ज़िंदगी ऐसे मोड़ पे आई\\
शिकवा गिला ना शिकायत\\
मन में शान्त मुस्कान छाई
}\end{verse}

\poemtitle{\texthindi{बात
}}\begin{verse}[\versewidth]\texthindi{
बात मुँह से नहीं दिल से की जाती है\\
ज़बान झूठ फ़रेब में है माहिर\\
दिल से कोई बात छुपाई ना जाती है\\
\\
हाथों से सहलाते छुरे घोंपते देखे\\
गले मिल गले घोंटते काटते देखे\\
दिल दूर से भेजे दुआ पहुँच जाती है\\
बात मुहँ...\\
\\
आँख दे जाती धोखा झूठे आँसू गिरा\\
मुस्कान छुपाती मतलबी शहद मिला\\
सच्चाई सिर्फ़ दिल में नज़र आती है\\
बात मुँह...\\
\\
फ़ासले जिस्मों में सात समंदर दूर\\
दिल दिल से मिले चाहे कोसों दूर\\
सरहदें मुल्कों की मिटायी जाती हैं \\
बात मुहँ...\\
\\
दिल से दिल बात करता चुपके से\\
हरकतों थर्राते खामोश बंद होंठों से\\
खामोशी में भी किताब लिखी जाती है\\
बात मुँह…\\
\\
आँख बंद कर दिल की निगाह से देखो\\
बात मुँह से नहीं दिल से की जाती है\\
खामोशी में भी किताब लिखी जाती है
}\end{verse}

\poemtitle{\texthindi{नये पंछी
}}\begin{verse}[\versewidth]\texthindi{
नये पंछी नया घरौंदा बनाने\\
तेरे दर पे आये हैं\\
घर सालों का छोड़\\
डरते घबराते तेरे घर में आये हैं\\
\\
साथी संगी काम, हर ईंट की यादें\\
पीछे छोड़ के आये हैं\\
हर पौधा पेड़ फूल हाथों से लगाया\\
अब वो रोते मुरझाये हैं\\
अपनी बच्चों की धुंधली यादें छोड़\\
इक नये मोड़ पे आये हैं\\
नये दोस्तों संग नयी उमंगें\\
नयी ज़िंदगी बनाने आये हैं\\
\\
तुम्हारी सतरंगी से रंग लें गे\\
कुछ अपने ले कर आये हैं\\
क्रिके ट की चौथी इनिंग्ज़ है\\
हंस खुल सेंचरी लगाने आये हैं\\
\\
जैसे भी हैं कबूल करो हमें\\
ये सपने उम्मीदें ले कर आये हैं\\
नये पंछी नया घरौंदा बनाने\\
तेरे दर पे आये हैं
}\end{verse}

\poemtitle{\texthindi{रौशनी की इज़्ज़त
}}\begin{verse}[\versewidth]\texthindi{
रौशनी की इज़्ज़त होती\\
अंधेरे में रहने के बाद\\
साथी की कमी महसूस होती\\
उस के जाने के बाद\\
\\
बच्चों का बचपन उड़ता बादल\\
आँख बरसाती आंसू\\
घोंसला खाली होने के बाद\\
शोख जवानी का नशा झूमता कुछ पल\\
मोम सा पिघल जाता बुढ़ापा आने के बाद\\
\\
नाशवान जिस्म पे गरुर ना कर\\
बिखरता मिट्टी में बीमारी लग जाने के बाद\\
\\
माँ बाप बुढ़ापे में चंद दिन के मेहमान\\
तस्वीर में याद आयें मर जाने के बाद\\
\\
कल जल गया राहे ज़िं दगी में\\
मीठी कड़वी यादें छोड़ गया\\
आज को हँसी खुशी से भर लो\\
ये खाक हो जाये कल आने के बाद\\
\\
रौशनी की इज़्ज़त होती\\
अंधेरे में रहने के बाद\\
साथी की कमी महसूस होती\\
उस के जाने के बाद
}\end{verse}

\poemtitle{\texthindi{इंसान की इज़्ज़त
}}\begin{verse}[\versewidth]\texthindi{
इंसान की इज़्ज़त है जब उस की ज़रूरत होती\\
उस पल में रब से भी ऊँची उस की मूरत होती\\
\\
ताकत पैसा हो तो दोस्तों की कतार बन जाती\\
ना हो उन से मदद तो दूर पहचान ना हो पाती\\
\\
बुरे वक्त में सुनते हैं गधा बाप बन\\
जाता\\
सुख मिलते ही बूड़ा लाचार बाप गधा\\
कहलाता\\
माँ ने अपनी कोख में पाला\\
सब कुछ लुटा के बड़ा किया\\
बोझ बनी बुढ़ापे में सुनती ताने\\
“सब करती हैं तुम ने क्या अजब किया!”\\
इंसान की इज़्ज़त है जब उस की ज़रूर होती\\
उस पल में रब से ऊँची उस की मूरत होती
}\end{verse}

\poemtitle{\texthindi{चाँद 
}}\begin{verse}[\versewidth]\texthindi{
देख सुर्ख सूरज की लाली\\
चाँद का चेहरा उतर गया\\
चाँदनी रूठ गई उस से\\
दामन हर तारा छुड़ा गया\\
\\
देख के सुंदर चंदा को\\
सारी रात चकोर मंडराता था\\
अब अकेला नील गगन में\\
मन ही मन घबराता था\\
\\
चढ़ते सूरज की पूजा सब करें\\
मुश्किल में दोस्त ना साथ रहे\\
करते थे वादा सहारे का\\
घायल देख के भी ना पास रहे\\
\\
चमक जहां सोने की देखी\\
यारी वहीं जमा बैठे\\
दोस्त भाई बहन छोड़े पीछे\\
पैसे की महफ़िल में जा बैठे\\
\\
कल तक मेरे हमदम थे\\
अब साये से कतराते हैं\\
मिलने से पहले औकात देखो\\
तीखे कड़वे ताने सुनाते हैं\\
\\
अपने भी अच्छे दिन थे\\
अब जीवन ने मारा है\\
बोझ अकेले ही उठाना\\
अपनों ने किया किनारा है\\
\\
ना कर शिकवा ओ चाँद मेरे\\
कुछ घंटों की ही बात है ये\\
सूरज की तपस जल जाये गी\\
रंग उस के उतरते जायें गे\\
\\
फिर रात तेरी चमचमाये गी\\
चाँदनी लौट चली आये गी\\
घर रूठे सितारे आयें गे\\
खुशियों की चादर लहराये गी\\
हर दिन के बाद रात घनेरी\\
हर खुशी के बाद गम आता है\\
जहां हिम्मत हो अपने दिल में\\
हरयाला सावन फिर आता है\\
चाँद फिर गगन में चमचमाता है\\
चाँद फिर गगन में चमचमाता है
}\end{verse}

\poemtitle{\texthindi{दोस्त
}}\begin{verse}[\versewidth]\texthindi{
दोस्त पुराने सूखे फूलों जैसे\\
दिल की किताब में रखते हैं\\
नये दोस्त जीवन में आये\\
जिन संग खेलें रोते हँसते हैं\\
\\
पुरानी यादें फूलों की खुश्बू\\
नज़र ना आये\\
नये दोस्त जीवन की बगिया को\\
रंग बिरंगा करते हैं\\
\\
हर कोई अपना रंग ले आता\\
मिल जुल के हार बनाते हैं\\
सरगम की माला है धड़कन\\
इक धागे से जुड़ जाते हैं\\
\\
दिन में सूरज दोस्त रात को\\
चाँद सितारे बन जाते हैं\\
जीवन की धूप छाँव में\\
इक दूजे का हाथ बटाते हैं\\
}\end{verse}

\poemtitle{\texthindi{नकली दोस्त
}}\begin{verse}[\versewidth]\texthindi{
सौ ठीक करो उन की सुनो\\
दोस्त तभी कहलाओ गे\\
इक भूल हुई या गलतफहमी\\
पैरों में कुचले जाओ गे\\
\\
भरी जेब देख दोस्त बन जायें\\
भँवरे बन वो फूलों पे मंडरायें\\
पर्दे पीछे जेब खाली कर जायें\\
नया शिकार देख अनजान बन जायें\\
हाकिम जो बन के बैठे हैं\\
दोस्त नहीं वो हो सकते\\
ना जीने दें गे जीते जी\\
मातम घर में भी कोसे जाओ गे\\
\\
सौ फूल चुनो इक काँटा चुभ जाए गा\\
सौ काम करो ज़बाँ खोलो\\
सब को नहीं वो भाए गा\\
किस किस को खुश रखूँ यारो\\
हर सोच मेरी हर काम मेरा\\
उन के तराज़ू में तोला जाये गा\\
\\
जिन्हें मज़बूत सहारा समझा था\\
बैठ किनारे मुसकाते नकली दोस्त\\
बेसहारा डूबता देख\\
नज़र फेर लेते हैं नकली दोस्त\\
\\
कच्चे धागे से होते हैं नकली दोस्त\\
तुनका लगा कटी पतंग सा छोड़ देते हैं \\
अकेला भटकता तड़पता छोड़ देते हैं\\
\\
सूखे फूल सा काट के फेंक जाते हैं \\
मिलना तो दूर बात करने से कतराते\\
राह में दिख जायें रास्ता बदल जाते हैं\\
\\
सामने से मुस्कान पीठ में खंजर\\
अब दोस्ती के नाम से डर लगता\\
शीतल हवा से तूफ़ान भला लगता\\
शान्त नदी से भँवर अच्छा लगता\\
अब तन्हाई लगती है अच्छी\\
लोगों की परछाई से डर लगता\\
\\
सब अपनी धुन में मस्त फिरें\\
कोई जिए या वो हो मरता\\
\\
नकली मुस्कान दे दे के थक गया हूँ\\
अब अलविदा कहने के दिन आये\\
तुम खुश रहो हम दुआ किए जाते हैं\\
\\
शुक्रगुज़ार हूँ तेरा समझा जिसे दोस्त\\
वक्त पे आँख मेरी खोल दी\\
वरना गलतफ़हमी में सारा जीवन गुज़रता\\
वरना गलतफ़हमी में सारा जीवन गुज़रता
}\end{verse}

\poemtitle{\texthindi{अतीत के भूत
}}\begin{verse}[\versewidth]\texthindi{
अतीत के भूत चलते हैं साथ\\
हँसाते कम ज़्यादा सताते\\
सुख के लाखों पल हंसी खुशी\\
दुःख के बादल में छुप जाते\\
\\
सारा संसार खिलाफ़ इन्हीं के\\
हर इंसान को दुश्मन बताते\\
सौ अच्छा किया वो भूल गए\\
इक बुरे की आग में रुलाते\\
अपने में मग्न भूत\\
दूजों को चोट लगा जाये\\
हर कर्म झील में फेंका पत्थर\\
शांत सोई लहरें जगाये\\
\\
भूल अपनी कमियाँ\\
औरों को दोषी ठहराते\\
कल का बीता पल गुज़र गया\\
बदकिस्मत उस में डूबे रह जाते\\
\\
आज के अस्थाई सुंदर पल में\\
खिलने वाले सपने ठुकराते\\
\\
जाने अनजाने जिन्होंने ने दुःख दिया\\
काँटे चुभाए उन्हें माफ़ करो\\
कल से सीख जीयो वर्तमान में\\
जहां सुनहरी बीज पनपते\\
\\
अतीत के भूत चलते हैं साथ\\
हंसाते कम ज़्यादा सताते\\
सुख के लाखों पल हंसी खुशी\\
दुःख के बादल में छुप जाते
}\end{verse}

\poemtitle{\texthindi{सवेरा
}}\begin{verse}[\versewidth]\texthindi{
किसी को दुःख मिले किस्मत से\\
कोई खुद ढूँढ ले आता है \\
दुनिया का खज़ाना पैरों में पड़ा\\
जाने अनजाने ठुकराता है \\
\\
धूप बाँटता सूरज हर झोली में\\
कोई बादल ढूँढ ले आता\\
बिन बरसात की बारिश खुद लाए\\
अब भीगता रोता पछताता\\
बच्चों को मुस्काते चेहरे मिले\\
घर खुशियों से भर जाता\\
काम क्रोध अहंकार लोभ\\
उसे आँसुओं में बदल जाता\\
\\
किसी को सेहत या पैसे की कमी\\
किसी को बच्चे का गम रुलाता\\
भूला बुरे दिन सदा नहीं रहते\\
चाँद में बस दाग ही देख पाता\\
\\
\\
भूल के अपनी कमियाँ दोष\\
दूजों पे उँगली उठाता\\
खुद को सही साबित करते\\
घर अपना बच्चों का जलाता\\
\\
वक्त की आग झुलस डालती\\
आशियाँ बहुत जल्द\\
चलता शरीर रुक जाये\\
गरुर इंसान ये सच भूल जाता\\
\\
किसी को वर्तमान कोई\\
वर्तमान को खा जाता\\
बीते पुराने घाव कुरेद\\
नयी कोमल त्वचा गंवाता\\
\\
फिर हो गा वही चोट लगे गी\\
यादों के भूत रख ज़िंदा\\
उम्मीद के बीज नहीं लगाता\\
सुखी रंगीन दुनिया नहीं बनाता\\
\\
बीता कल मर गुज़र गया\\
क्यों उस के भँवर में डूबा रहता\\
चाहे जगते देर हुई\\
आँख खुले तभी सवेरा कहाता\\
आँख खुले तभी सवेरा कहाता
}\end{verse}

\poemtitle{\texthindi{आग में सुलगना
}}\begin{verse}[\versewidth]\texthindi{
भूलों गलतियों की लकड़ियों पर\\
झुलसता जलता रहता हूँ\\
दिन रात ख्यालों में डूबा\\
यादों की आग में सुलगता हूँ \\
\\
गुज़रा वक्त ना वापिस आए\\
जो घट चुका बदलता नहीं\\
इक इक याद का कड़वा घूँट\\
ना चाहते रो के पीना पड़ता\\
माँ बाप सदा रहें गे\\
कल उन संग बैठूँ गा\\
वो कल नहीं आया\\
बारी बारी सब ने छोड़ा जगाया\\
फिर भी होश में नहीं आया\\
\\
जो हैं ज़िंदा उन से शिकवे गिले\\
मिलने से कतराता\\
अपने में मगन जिस्म नाशवान\\
इस सच को भूल जाता\\
रंगीन जवानी सदा रहे गी\\
ये गलतफ़हमी सामने आयी\\
बुढ़ापे ने बीमारियाँ ला के\\
भयानक जब सूरत दिखाई\\
\\
जो वक्त बचा है गर हो सके\\
गलतियों का एहसास करो\\
जो चाहने वाले हैं उन्हें गले लगा\\
जी भर उन की प्यास भरो\\
\\
जिन्हें ठुकराया गले लगा लो\\
जिन से गलती की माफ़ी माँगो\\
जिन्हों ने चोट लगायी दर्द दिया\\
दिलो दिमाग़ से उन्हें माफ़ करो\\
जो चाहने वाले हैं उन्हें गले लगा\\
जी भर उन की प्यास भरो
}\end{verse}

\poemtitle{\texthindi{कच्चे घड़े
}}\begin{verse}[\versewidth]\texthindi{
कच्चे घड़े को ध्यान से तराशो\\
उँगलियों के निशाँ उम्र भर रहते\\
मासूम बच्चे हैं कच्चे घड़े\\
कही सुनी देखी का असर सहते\\
\\
ज़हन पे छा जातीं\\
अच्छी देतीं खुशी सुकून\\
दुःखी कठोर उम्र भर सतातीं\\
\\
बचपन के निशाँ बन के तराज़ू\\
दुनियाँ को उस में तोलते\\
रंग चढ़ जाता चश्मे पे\\
जिस हालात से गुज़रते\\
\\
कुम्हार धरती के टुकड़े को\\
कोमल हाथों से संवारता\\
प्यार भरा हाथ मिट्टी को\\
सुंदर मीठा घड़ा बनाता\\
\\
कठोर हाथ उम्र भर \\
बेढंगा बेरूप कर जाता\\
हँसी खुशी प्यार भरा घर\\
सूखा बाग बंजर बन जाता\\
\\
झगड़ा झूठ ग़ुस्सा नशा\\
नर्क की नींव बन जाता\\
चमचमाता घड़ा बनाओ\\
जो जग में रोशनी फैलाए\\
देखने करने में कोमल\\
दुनिया को मीठा पानी पिलाए\\
\\
बहुत जल्द पर निकल आयें\\
पलक झपकते पंछी उड़ जायें\\
प्यार से तराशे खुद को प्यार करें\\
दुनिया में ऊँचा नाम कमायें\\
\\
कच्चे घड़े को ध्यान से तराशो\\
उँगलियों के निशाँ उम्र भर रहते\\
मासूम बच्चे हैं कच्चे घड़े\\
कही सुनी देखी का असर सहते
}\end{verse}

\poemtitle{\texthindi{बच्चों की मुस्कुराहट
}}\begin{verse}[\versewidth]\texthindi{
बच्चों की खिलखिलाहट में ज़िंदगी की\\
रौशनी देखता हूँ\\
किलकारियों में घनघनाते बादल बिजलियों\\
की चमक देखता हूँ\\
\\
बिन वजह हंसना कूदना नाचना\\
होठों से मोती छलकाना\\
ना पीछे का गम ना आगे की चिंता\\
हर पल में रब की मेहर देखता हूँ\\
उनकी मुस्कराहट के सामने\\
फूल कमज़ोर फीके पड़ जाते\\
तितलियाँ सीखें उड़ना उनसे\\
पंछियों को गाना सीखते देखता हूँ\\
\\
छोटी सी उँगलियाँ हाथ पकड़\\
दिल तक पहुँच जाते\\
रूखे खूँखार इंसान का पत्थर दिल\\
मोम सा पिघलते देखता हूँ\\
\\
छोटी छोटी चीज़ों में खुशियाँ पाना झूमना\\
मुस्कुराना\\
लैगो लॉली पॉप आइसक्रीम में खुशियों का\\
खज़ाना\\
हर एक को मुस्कान\\
सब को प्यार घुट के मिलना\\
बहुत नाज़ुक हैं गुड्डे गुढ़ियाँ\\
कड़वी आवाज़ से चेहरा उतरते देखता हूँ\\
\\
ना छल कपट ना झूठ\\
ना जात पात या रंग का भेद\\
उनकी नज़र में रब की सारी कायनात\\
एक ही रंग में देखता हूँ\\
\\
ना चोरी ना लूटना ना झूठी\\
मतलबी बनावटी मक्कारी\\
कुदरत का पूरा खज़ाना\\
हर मासूम बच्चे के अंदर देखता हूँ\\
बच्चों की खिलखिलाहट में ज़िं दगी की\\
रौशनी देखता हूँ\\
किलकारियों में घनघनाते बादल बिजली की\\
चमक देखता हूँ\\
बिन वजह हंसना कूदना नाचना होठों से मोती\\
छलकाना\\
जब किसी बच्चे को देखता हूँ\\
उन में रब की मेहर देखता हूँ
}\end{verse}

\poemtitle{\texthindi{नया जीवन
}}\begin{verse}[\versewidth]\texthindi{
बचपन के घर की दीवार पे बैठा सोचूँ\\
अलविदा स्वागत कहने का वक्त आया\\
बिछड़ना मुश्किल है अपनों से\\
छोड़ूँ कैसे जो मेरा अपना ही साया\\
\\
रोना हँसना बोलना चलना सीखा\\
कल ही इस घर में था मैं आया\\
उँगली पकड़ छोड़ने बीच वक्त कम\\
पलक झपकते आँगन बना पराया\\
मीठा बचपन बेपरवाही के दिन रात\\
चढ़ती जवानी ने दो पल में खाया\\
दिल में उमंगें सपने नये सोच बदली\\
पंछी खुद उड़ना सीखा पंख फैलाया\\
आगे बड़ूँ या ना छोड़ूँ बचपन\\
सोच में डूबा रहता मन घबराया\\
\\
बचपन के घर की दीवार पे बैठा सोचूँ\\
अलविदा स्वागत कहने का वक़्त आया\\
(ये कविता बड़े दोहते के लिए लिखी थी।\\
वो 18 साल का होने के बाद घर छोड़ कॉलेज\\
जा रहा था। उन सब बच्चों को अर्पित करता\\
हूँ जो इस दोराहे से गुज़रते हैं

}\end{verse}

\poemtitle{\texthindi{छे फ़ुट का फ़ासला
}}\begin{verse}[\versewidth]\texthindi{
ना कोई मारे जफी ना हाथ मिलाता\\
गले लगा पैर छू दुआ लेने से कतराता\\
इंसान को देखते इंसान घबरा जाता\\
छे फ़ुट के फ़ासले ने सब बदल दिया\\
\\
जिस की कमी थी अब कटता नहीं\\
वक्त अकेले खाने को आता\\
पशु घूमें बेधड़क पंछी चैचहाता\\
छे फ़ुट के फ़ासले...\\
\\
\\
खुशी बाहर नहीं अंदर है\\
मन मुस्कान का मंदर है\\
ज़रूरतें बहुत कम मालूम हुआ\\
अकेले रह अब एहसास हुआ\\
छे फ़ुट के फ़ासले…\\
\\
साफ़ हवा नीला आकाश\\
चाँद चमकता नहीं पत्ते पे धूल\\
खुल के साँस ले सकता\\
कुदरत ने रक्षा खातिर वार किया\\
छे फ़ुट के फ़ासले…\\
\\
पाप का घड़ा भर जाता\\
भगवन लेते हैं अवतार\\
ज़रूर नहीं इन्सान के रूप में आये\\
अन दिख कोरोना का आँचल ओड़े\\
गरुर इन्सान को सच्ची राह दिखाये\\
छे फ़ु ट के फ़ासले ने सब बदल दिया\\
छे फ़ु ट के फ़ासले ने सब बदल दिया\\
\\
साथी\\
\\
तुम हो तो मैं हूँ वरना ये\\
चार दीवारी में सूना अकेलापन\\
ना हँसना ना हैरान होना देख\\
ढलता सूरज या पहली किरण\\
\\
हाथ पकड़ना गले लगाना बिन वजह\\
मुस्कुराना\\
ज़िंदगी है कोई साथ हो वरना सिर्फ़\\
साँसों का अमन गमन\\
साथी से रूठना रोना रुलाना\\
उसे मनाना अकेलेपन से बेहतर\\
जीवन के गम खुशियाँ बाँटना\\
अकेले अमृत पीने से बेहतर\\
\\
साथी चाहे बीमार अपाहिज\\
जीवन का मकसद बन जाता\\
उस की छोटी खुशी मुस्कान\\
अकेले अपार खुशी से बेहतर\\
\\
कोई साथ तो दुनिया जगमग\\
वरना भीड़ में भी अकेले हैं\\
खुशकिस्मत तकदीर के धनी\\
जो पास हैं जिन संग खेले हैं\\
\\
हर कोई अपना साथी खोजता\\
पौधे पेड़ पशु इंसान\\
अकेले वीरां जंगल सहरे\\
साथी हो तो खुशियों के मेले हैं\\
साथी हो तो खुशियों के मेले हैं
}\end{verse}

\poemtitle{\texthindi{घर लुटवाना
}}\begin{verse}[\versewidth]\texthindi{
घर लुटाना है तो अपनों से लुटवाओ\\
कमबख्त दुश्मन हमेशा लूटते रहते हैं\\
दुश्मनों की तरकीबों से हम थे वाकिफ़\\
अब अपनों की असलियत पहचानते हैं\\
\\
दुश्मनों को गले मिलने से सदा थे कतराते\\
मौत अपनों के कंधे पे हुई ये फ़ायदा होता है\\
\\
भाई भाई को छुरा घोंपे दर्द ज़्यादा होता है\\
दिखावे की ज़िंदगी से अंत बेहतर होता है\\
\\
आँखों पे पट्टी थी बंधी अचानक खुल गई\\
जिस का लेता सहारा कच्ची दीवार टूट गई\\
अपनों की ईंटें मेरी कब्र का सामान बन गयीं\\
\\
खुश हूँ मेरे दुश्मनों की ये हसरत भी पूरी ना\\
हुई\\
खुश हूँ मेरे दुश्मनों की ये हसरत भी पूरी ना\\
हुई
}\end{verse}

\poemtitle{\texthindi{औरत
}}\begin{verse}[\versewidth]\texthindi{
नज़रें ज़माने की चुभती तीर सी\\
जब घर से बाहर जाती हूँ\\
जिस तन से जन्मे दूध पिया\\
उसी को उन से छुपाती हूँ\\
\\
मर्दों की हरकतों से डर लगता\\
आँख ज़मीं में गड़ जाती\\
लार टपकते होंठों से\\
नोचते हाथों से जिस्म बचाती हूँ\\
सर से पाँव तक हर\\
हिस्से का मुआइना करवाती\\
दोष है उन की काली नज़र का\\
द्रोपति मैं बनाई जाती हूँ\\
\\
मेरे जिस्म पे मेरी चाल\\
कपड़ों पे ताने कसे हैं जाते\\
कसूर कमीनों मर्दों की हवस\\
बेशर्म दोषी मैं कहलाती हूँ\\
\\
\\
चुपके सिमटे छुपती चलती\\
ना आँख किसी से मिलाती\\
मतलब गलत निकाले दुनिया\\
गरूर भरी मैं कहलाती हूँ\\
\\
कोई घिनौने हाथ लगाता\\
कोई आगे पीछे छुए मसले\\
हक समझें अपना वो दरिंदे\\
खिलौना मैं बनाई जाती हूँ\\
\\
\\
कुछ पाना है कुछ खोना पड़े गा\\
ऐसी रस्में बताई जाती हूँ \\
सिर्फ़ औरतों को हैं लागू\\
वो राह दिखाई जाती हूँ\\
\\
मर्दों की है दुनियाँ\\
उन की हुकूमत कानून उन्हीं के\\
फिर भी घर छोड़\\
बाज़ार या काम पे जाती\\
जिस्म वजूद खो कर अपना\\
किस्मत मैं आज़माती हूँ\\
\\
नज़रें ज़माने की चुभती तीर सी\\
जब घर से बाहर जाती हूँ\\
जिस तन से जन्मे दूध पिया\\
उसी को उन से छुपाती हूँ\\
उसी को उन से छुपाती हूँ
}\end{verse}

\poemtitle{\texthindi{काश हम मिले ना होते
}}\begin{verse}[\versewidth]\texthindi{
काश हम मिले न होते\\
ज़हरीले बीज बोये न होते\\
काँटों भरे फूल खिले न होते\\
दिन में अंधेरा रात में आँसू\\
हर पल शिकवे गिले न होते\\
\\
एक भूल की सज़ा उम्र भर\\
नींव के पत्थर हिले न होते\\
लोगों के फ़ैसले जड़ें सुखा गये\\
ऊँचे पेड़ के तने हिले न होते\\
\\
भूखा अतीत दीमक बन गया\\
घर दरो दीवार गिरे ना होते\\
बच्चे जवानी पैसा सब था\\
नज़र अन्दाज़ किये न होते\\
\\
कोई किस्मत से लड़ नहीं पाता\\
समझौते दोनों ने कर लिये होते\\
अब अंत समय में मन रोया\\
आँख से आंसू गिरे न होते
}\end{verse}

\poemtitle{\texthindi{काश हम बिछड़े न होते
}}\begin{verse}[\versewidth]\texthindi{
काश हम बिछड़े न होते\\
दुःख ये अकेले सहना न पड़ता\\
सुख खुशियाँ दिखायी न होतीं\\
जाना तेरा इतना न अखरता\\
\\
वो पहली नज़र का जादू\\
न चाहते आँख का झुक जाना\\
फिर देखना तो झपकना भूल\\
तेरे चेहरे से नज़र न हटाना\\
उम्र ने बदन बदला पर तेरी\\
यादों की धूप ताज़ा अभी भी है\\
काले बादल छाये दिल पे\\
दिन में अंधेरा अभी भी है\\
\\
न देखा होता वो चेहरा न\\
याद आता न दुःख पहुँचाता\\
याद कर आँखों से आँसू छलके\\
दिल तुझे भुला नहीं पाता\\
\\
\\
जीवन का काम है चलना\\
समय थमता नहीं किसी के लिए\\
बाकी बची ज़िंदगी ऊपर से हँसी\\
दिल रोता किसी के लिए\\
\\
मेरी खुली हँसी दबी सी जाती\\
क्यों कि तू मेरे साथ नहीं\\
भरी महफ़िल में अकेली पड़ जाती\\
क्यों कि तू मेरे साथ नहीं\\
\\
\\
हर कोई देता है सहारा\\
दो शब्द हिम्मत के कोई कहता\\
तेरे न होने का घाव गहरा\\
मेरे रोम रोम में हरदम रहता\\
\\
जिधर भी देखूँ छवि तेरी दिखती\\
तेरी नज़र देती ठंडी छाँव\\
अकेली नहीं तू चलता संग\\
मेरे साथ छुप के दबे पाँव\\
इतनी सी कीमत है चुकानी\\
उन सुनहरी यादों के लिए\\
खुशी से पी लूँ हर पल का ज़हर\\
उन लाखों लमहों के लिए\\
\\
तू ही बता मैं क्या करूँ\\
जानती तू सदा साथ रहता है\\
फिर भी पागल दिल अकेली\\
रातों में रोता है फिर कहता है\\
काश हम बिछड़े न होते\\
अकेले ये दुःख सहना न पड़ता\\
सुख खुशियाँ दिखाई न होतीं\\
तेरा जाना इतना न अखरता
}\end{verse}

\poemtitle{\texthindi{जीवन पथ
}}\begin{verse}[\versewidth]\texthindi{
हर मुश्किल पे रुक गये तो\\
मंज़िल को ना पाओ गे\\
जीवन पथ में काँटों के डर से\\
फूल नहीं चुन पाओ गे\\
\\
रंग बिरंगी तितली उड़ने खातिर\\
दर्द भरी राह अपनाती\\
जन्म नव जीवन को देने खातिर\\
लाखों कष्ट माँ उठाती\\
\\
बाधा आने से ना रुकती नदी\\
अपनी नयी राह बनाती\\
साथ सहारे छोड़ें जब सारे\\
वो इक सुन्दर झरना बनाती\\
\\
\\
कुत्ते के भौंकने से रुक गये\\
घर अपने भी ना जा पाओ गे\\
आलोचना लोगों की डर से\\
सपने पूरे ना कर पाओ गे\\
\\
बच्चा गिर के संभलता\\
उठ चलने भागने लगता\\
गिरने के डर से जो रुका\\
खुद अपना पिंजरा बनता\\
\\
\\
हर मुश्किल पे रुक गये तो\\
मंज़िल को ना पाओ गे\\
जीवन पथ में काँटों के डर से\\
फूल नहीं चुन पाओ गे\\
जीवन पथ में काँटों के डर से\\
फूल नहीं चुन पाओ गे
}\end{verse}

\poemtitle{\texthindi{दर्द ए दिल
}}\begin{verse}[\versewidth]\texthindi{
आग जलती तो रोशनी होती है\\
धुआँ उठता है आवाज़ होती है\\
दिल जलता है छुप के चुपके\\
सिर्फ़ आँखों से बरसात होती है\\
\\
आग का काम है जलना\\
यज्ञ या मातम दिया ना पहचाने\\
दिल जले जिस की खातिर\\
वो बेखबर उस का दर्द ना जाने\\
\\
प्यासा सूखा पत्ता जले\\
अपनों की ही लौ से\\
फिर राख बन जाता\\
दिल जले अपनों की\\
कड़वी बातों से\\
दुनियाँ को दिखा ना पाता\\
\\
पानी बुझाता चमकती लौ\\
राख धुआँ बाकी रह जाता\\
आँख से आँसू बहते मगर\\
दिल का दर्द नहीं जाता\\
\\
आग जलती तो रोशनी होती है\\
धुआँ उठता है आवाज़ होती है\\
दिल जलता है छुप के चुपके\\
सिर्फ़ आँखों से बरसात होती है
}\end{verse}

\poemtitle{\texthindi{गुस्सा
}}\begin{verse}[\versewidth]\texthindi{
ज्वालामुखी लगता इन्सान\\
 गुस्से में जब घिर जाता\\
लावा टपकता होठों से\\
जो सामने उसे जलाता\\
\\
गुस्सा बुद्धि का है दुश्मन\\
सोच पे पर्दा पड़ जाता\\
पल में उम्र भर का रिश्ता\\
टूटी दीवार बन जाता\\
\\
पैसे का नशा उगले या\\
रंगे शराब आये सामने\\
लपटों में खुद जलता\\
दूजों को भस्म कर जाता\\
\\
\\
शांत चमक हुई आग की\\
खामोश नज़ारा होता\\
शाख जली पेड़ राख हुए \\
नामों निशाँ मिट जाता\\
\\
अब गिड़गिड़ाए माँगे माफ़ी\\
भूचाल से टूटा जुड़ता नहीं\\
जो राख हुआ चुपके पी आँसू\\
दिल ही दिल मर जाता\\
\\
\\
सुर्ख तीर शब्दों का अन मिट\\
गहरा घाव लगाता\\
धरती बंजर नया पेड़ आँ गन में\\
उगने से घबराता\\
\\
ज्वालामुखी लगता इन्सान\\
जब गुस्से में घिर जाता\\
लावा टपकता होंठों से\\
जो सामने उसे जलाता
}\end{verse}

\poemtitle{\texthindi{एक हाथ की ताली
}}\begin{verse}[\versewidth]\texthindi{
सुनते हैं ताली एक हाथ से नहीं बजती\\
मैं कहता हूँ —गलत\\
ताली एक हाथ से भी बजती\\
मगर वो चाँटा कहलाती\\
\\
इक खुशियों के जश्न मनाती\\
साथ मिल जाती पूरी बस्ती\\
इक गुस्से में बदकिस्मत\\
कमज़ोर गाल पे बजती\\
\\
साथी संगी खामोश खड़े\\
वो अपनी नज़र में गिरती\\
मासूम बच्चों पे ज़ुल्म गिराती\\
तरह तरह से उन्हें सताती\\
\\
एक अकेले हाथ की ताली\\
जीवन भर के घाव लगाती\\
कच्ची कली मसली जाती\\
फूल कभी वो बन नहीं पाती\\
\\
दो हाथ ताली खनकती कुछ पल\\
फिर शांत हो जाती \\
यादें कुछ दिन रहती दिल में \\
फिर मन से मिट जाती \\
\\
इक हाथ की ताली गरजती\\
गाल पे लाल निशान बनाती\\
अन मिट आवाज़ पीड़ा दिल में\\
यादें आँसू उम्र भर दे जाती\\
\\
सुनते हैं ताली एक हाथ से नहीं बजती\\
मैं कहता हूँ —गलत\\
ताली एक हाथ से भी बजती\\
मगर वो चाँटा कहलाती
}\end{verse}

\poemtitle{\texthindi{मकड़ी जाल
}}\begin{verse}[\versewidth]\texthindi{
फँस गया हूँ अपने ही जाल में\\
जो हंसी खुशी बनाया था\\
सोचा जोड़ूँ गा जीने के साधन\\
काम में दिन रात गवाया था\\
\\
इतना ताना बोया जो रखना\\
सम्भालना छोड़ना मुश्किल\\
धन जोड़ने में उम्र गुज़ारी\\
अब समझा ये सदा पराया था\\
\\
ना सेहत ना देखे दोस्त परिवार\\
नींद को भी भुलाया था\\
रब का ध्यान ना शुक्रिया बोला\\
अपनी धुन में समाया था\\
\\
माँ बाप जिन्हों ने जन्म दिया\\
पाला पोसा बड़ा किया\\
अपने में खोया ना देखे आँसू\\
जब जब उन्हें रुलाया था\\
\\
ना सोचा सुनसान महल चाहिए\\
या चहचहाता खुशी भरा घर\\
मालूम हुआ जब सब छोड़ गये\\
रोता देख अपना साया था\\
\\
बहता वक्त रुकता नहीं\\
तोड़ देता जिसे उस ने बनाया\\
ना जाने क्यों हम सुनते नहीं\\
जब बुज़ुर्गों ने समझाया था\\
\\
सिर्फ़ मेरा जाल अटूट\\
आँधी बारिश इसे ना तोड़ें\\
इसी भूल में धरती के\\
आँचल में गिरा पछताया था\\
\\
फँस गया हूँ अपने ही जाल में\\
जो हंसी खुशी बनाया था\\
फँस गया हूँ अपने ही जाल में\\
जो हंसी खुशी बनाया था
}\end{verse}

\poemtitle{\texthindi{राजनेता
}}\begin{verse}[\versewidth]\texthindi{
देश को लूटते हैं पौलीटीशन\\
बहुत बेहया हैं x2\\
नहीं है शरम\\
देश को लूटते हैं पौलीटीशन\\
\\
चुनाव से पहले ये सेवक हमारे\\
जीत के बाद भूले वादे वो सारे\\
अब बने ये राजा x2\\
नौकर हैं हम\\
देस को लूटते…\\
\\
कपड़े सफ़ेद पर दिल इन का काला\\
बगल में छुरी छुपायें हाथ में माला\\
दो मुहाँ ये सांप है x2\\
डसें हर दम\\
देश को लूटते…\\
\\
चाहे वो जीतें चाहे वो हारें\\
सब पार्टीयाँ मिल के जनता को मारें\\
दोष किसे दें x2\\
फूटे करम\\
देश को लूटते…\\
\\
राजनेता के घर में हीरे मोती बरसे\\
जनता बेचारी सूखी रोटी को तरसे\\
जितना भी रोयें x2\\
आँसू हैं कम\\
देश को लूटते…\\
देश को लूटते हैं पौलीटीशन\\
बहुत बेहया हैं x2\\
नहीं है शरम\\
देश को लूटते हैं पौलीटीशन
}\end{verse}

\poemtitle{\texthindi{पुनर्मिलन
}}\begin{verse}[\versewidth]\texthindi{
कल बच्चे थे अब बढ़े हुए\\
ना बूढ़ा कोई कहलाता\\
क्या चाल थी कई बाल थे\\
अब ढूँढना पड़ जाता\\
मक्कड़ मेकार मदन डैन\\
जुगिन्दर जे बन जाता\\
बाहरी शक्ल बदल गयी\\
अब नाम भी बदला जाता\\
\\
चमक थी सपनों की आँखों में\\
अब मोतिया छा जाता\\
फले फूले बड़ी इमारत में\\
खंडहर नज़र अब आता\\
अभी कल की ही बात है\\
कोरा कागज़ थे कुछ पास न था\\
जीवन की कलम क्या लिखे गी\\
बेखबर हमें एहसास ना था\\
\\
कुछ खुद लिखा कुछ औरों ने\\
कुदरत ने ज़्यादा लिखवाया
}\end{verse}

\poemtitle{\texthindi{नज़रिया\\
}}\begin{verse}[\versewidth]\texthindi{
नज़रिया ठीक हो ज़िंदगी खूबसूरत होती है\\
नज़रिया ठीक हो ज़िंदगी खूबसूरत होती है\\
\\
शुरू हुआ खत्म भी हो गा कोई पीछे कोई\\
पहले\\
इस बीच मिले हैं जो पल हँसी खुशी से\\
भर ले\\
सुबह हुई जब आँख खुली सारे अंग सलामत\\
किस्मत वालों को मिलते उन्हें स्वागत कर ले\\
जो मिला प्रसाद रब का नम्रता स्वीकार करो\\
कर्मों के हिसाब से सब के हिस्से बाँटा जाए\\
हर पल में रब की कृपा मिली है दूजों पे\\
लुटाओ\\
तुझे बनाया साधन चारों ओर खुशियाँ\\
बरसाओ\\
जोड़ने से निर्धन बाँटने से अमीर बनता इंसान\\
दूजों का दर्द मिटा के अपने दर्द दूर भगाओ\\
\\
मिला तुझे जितना दुनियाँ आधा पाने को रोती\\
ना गिनो अपने आशीर्वाद दूजों के तराज़ू से\\
चमकती ज़िंदगियाँ अनगिनत गम पिरोती \\
नज़रिया ठीक हो ज़िंदगी खूबसूरत होती है\\
नज़रिया ठीक हो ज़िंदगी खूबसूरत होती है\\
\\
किस्मत के धनी\\
रब कृपा से किस्मत के धनी\\
संसार का दुख देख सकते हैं\\
जो मिला बाँटते दिल खोल\\
जितना भी है दे सकते हैं\\
\\
ऐसे हीरे कभी कभी\\
दुनियाँ में जन्म लेते हैं\\
\\
अक्सर उन से पूछता हूँ राज़\\
कहते हैं जो पाया है रब से\\
इक जन्म काफ़ी नहीं लौटाने को\\
\\
\\
उस ने मुझे काबिल समझा\\
अपना काम मेरे हाथों कराने को\\
\\
नाम की चाहत ना इनाम की इच्छा\\
चुप गरीब पिछड़ों की सेवा करते हैं\\
बाहरी इज़्ज़त मेडल या चर्चा छोड़\\
हर दम रब का शुक्रिया करते हैं\\
\\
रब कृपा से किस्मत के धनी\\
संसार का दुख देख सकते हैं\\
ऐसे हीरे कभी कभी\\
दुनियाँ में जन्म लेते हैं
}\end{verse}

\poemtitle{\texthindi{रामायण सारांश में किस्मत
}}\begin{verse}[\versewidth]\texthindi{
बाप ने मानी बीवी की\\
बेटे ने मानी बाप की\\
मारीच बना स्वर्ण हिरन\\
रावण किया सीता हरण\\
भर ली गठरी पाप की\\
 \\
राम बेर शबरी के खाये\\
बानर पूँछ से लंका जलाये\\
घर का भेदी लंका ढाये\\
राम लखन सीता घर लाये\\
भरत खड़ांवें गद्दी हटाये\\
लोग खुशियों के दीप जलाये\\
\\
वाल्मीकि घर लव कु श जन्मे\\
सीता बारह साल बिताई\\
धोबी कारण धरती जानकी खाई\\
सीता ने दुख पाया जान गवाई\\
राम ने रघुकुल मर्यादा निभायी\\
\\
तब से दुनियाँ कहती आयी\\
सिया पति राम चंद्र की जय\\
सिया पति राम चंद्र की जय\\
}\end{verse}

\poemtitle{\texthindi{महाभारत सारांश में\\
}}\begin{verse}[\versewidth]\texthindi{
कौरव पांडव चचेरे भाई\\
ज़मीं खातिर करी लड़ाई\\
एक चुने कृष्ण को साथी\\
दूजों ने उन की सेना पाई\\
\\
अर्जुन अपने देख सामने\\
मन बदन ढीला पड़ा\\
सुन कृष्ण से गीता सच समझा\\
गुरु मित्र भाइयों से लड़ा\\
\\
चक्षु नेत्र से संजय नैन हीन\\
धृतराष्ट्र को कथा सुनाए\\
सौ बच्चों की मौत सुनी जब\\
अंधी आँखों में आँ सू आये\\
\\
अठारह दिन कुरुक्षेत्र में\\
युद्ध से लाखों का खून हुआ\\
पांडव जीते कौरव हारे\\
लिखा खेल अब खत्म हुआ\\
\\
आओ सब मिल बोलो\\
बाल गोपाल की जय\\
कन्हैया लाल की जय\\
कृष्ण भगवान की जय
}\end{verse}

\poemtitle{\texthindi{उम्मीदे वफ़ा\\
}}\begin{verse}[\versewidth]\texthindi{
दूसरों से उम्मीदे वफ़ा की बात कैसे कहें\\
खुद हम ने पीछे मुड़ कब उन को देखा था\\
ना देखे छलकते आँसू सिसकियाँ झुके काँधे\\
उड़ते परिंदों ने आखरी सलाम ना देखा था
}\end{verse}

\poemtitle{\texthindi{सुनामी\\
}}\begin{verse}[\versewidth]\texthindi{
गलती इंसान ही नहीं\\
बार बार करता भगवान\\
वरना क्यों ये बाढ़ कातिल तूफ़ान\\
ज़मीं सागर बनी टूट पड़ा आसमान\\
बच्चे बूढ़े डूबे कोशिश बाद जवान\\
समंदर को धरती पे लाना\\
ऐसी गलती क्यों भगवान
}\end{verse}

\poemtitle{\texthindi{एक रब कई नाम\\
}}\begin{verse}[\versewidth]\texthindi{
ना जाने क्यों किसी ने रब के\\
लाखों हिस्से बनाये नाम दिये\\
राम कृष्ण खुदा कहे कोई\\
मसीहा सतनाम कोई बुलाये\\
वो इक रहता कण कण में\\
ना दिखे ना नाम है उस का\\
चुपके सारा संसार चलाये
}\end{verse}

\poemtitle{\texthindi{घड़ी\\
}}\begin{verse}[\versewidth]\texthindi{
घड़ी की टिक टिक\\
हर पल करे एलान\\
उठ सोये बेहोश बंदे\\
कर ले रब से प्यार\\
और उस के कृष्मों से\\
चंद घड़ियाँ बची हैं\\
प्यार जताने के लिये\\
किताब के पन्नों में खुशियाँ मिलीं\\
कुछ ने आँसू छलकाया\\
\\
खाली मटका ले निकले घर से\\
सोचा ठंडा मीठा पानी भरें गे\\
लोक सेवा शानों शोहरत पैसा\\
दुनियाँ में ऊँचा नाम काम करें गे\\
राहे ज़िं दगी में साथी मिले गा\\
रंग बिरंगे फूल खिलें गे\\
किस्मत वालों ने झोली भर ली\\
कु छ बेवक्त मौत मिलें गे\\
किसी ने सोच समझ के\\
रास्ता अपना खुद ही तराशा\\
पत्तों तरह कुछ बहे नदी में\\
अनजान मंज़िल की अभिलाषा\\
किसी ने कुछ खोया बहुत पाया\\
जो सपनों में भी ना सोचा था\\
किसी बदकिस्मत नासमझ को\\
बुरी संगत ने आन दबोचा था\\
\\
कोई किस्मत से या मेहनत से\\
अच्छी सेहत पा जाते हैं\\
बीमारियाँ दुःख ले नाम अलग\\
कइयों के घर में समाते हैं\\
\\
पचास साल बाद फिर मिलें गे\\
सपने में भी ना सोचा था\\
जसबीर मदन की मेहनत ने\\
भूली यादों को खरोंचा था\\
अब यादें ज़्यादा सपने कम\\
उन यादों को दोहराएँ गे\\
अभी हैं ज़िंदा अभी भी दम है\\
फिर मिले गीत खुशी के गाएँ गे\\
फिर मिले गीत खुशी के गाएँ गे\\
खुश पचासवाँ पुनर्मि लन\\
क्लास 1966
}\end{verse}
