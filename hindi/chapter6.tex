%chapter 6
\chapter{\texthindi{
बुढ़ापा बीमारी मौत
}} 

%\begin{multicols}[2]
\poemtitle{\texthindi{बड़ती उम्र
}}\begin{verse}[\versewidth]\texthindi{
ज़िंदा हो शुक्र करो लाखों को नसीब ना होता \\
हर रात को सुबह का सूरज नसीब नहीं होता\\
\\
खुश रहो हर हाल में, वक्त और बुरा हो\\
सकता है\\
रब के तोहफ़े गिनो, कमियाँ हर कोई गिन\\
सकता है \\
\\
पुस्तक पढ़ो सुंदर दुनियाँ की सैर करो \\
रब ने जिस काम का बीज दिया \\
उसे जगा कर दुनियाँ में बाँटा करो \\
बहाने हर कोई बना सकता है \\
\\
मैं ने पूछा ज्योतिषी से दिन कितने बाकी बचे\\
हैं मेरे\\
जितनी लोगों की सेवा करो रब दुगने कर दे\\
गा तेरे\\
\\
ज़िंदा रहो खुल के जब तक साँसों का \\
अमन गमन है \\
किसी के काम आ सको तो वीरान बंजर भी \\
चमन है \\
\\
जीते रहे तो उम्र का नम्बर बड़ता जाए गा\\
ज़रूरी नहीं जवानी का नम्बर घटता जाए गा\\
\\
मौत तो सब को आनी है\\
बेमौत मरना ज़रूरी नहीं\\
बड़े होना जीवन की रीत है\\
बूड़े होना ज़रूरी नहीं\\
बड़े होना जीवन की रीत है\\
बूड़े होना ज़रूरी नहीं
}\end{verse}

\poemtitle{\texthindi{मेरी उम्र
}}\begin{verse}[\versewidth]\texthindi{
बच्चे मेरे बड़े हो गये\\
दोहते दोहतियाँ जवाँ हो गयीं\\
पर मैं वही जवान गबरू\\
दिमागी उम्र बढ़ना भूल गयी\\
\\
शीशे में परछाई ना पहचानू\\
ये बाबा कहाँ से आया\\
बिन कार्ड डिस्काउंट मिले\\
कैसा ज़माना है आया\\
\\
लोग कहें अंकल मुझे\\
शक उन की अक्ल पे होता\\
बहन भाई बूड़े बस मैं जवान\\
ऐसा गुमान दिल में होता\\
\\
अभी भी फेरूँ हाथ मैं सर पे\\
भले चट खाली मैदान वहाँ\\
बंजर कैसे ये बन गया\\
अभी हरा भरा था खेत वहाँ\\
\\
स्याही बचाने वास्ते छोटे अक्षर\\
अखबार वाले लिखने लगे\\
दर्द गले का बचाने खातिर\\
लोग धीमी आवाज़ बोलने लगे\\
\\
टी वी पे बच्चे खबर सुनायें मुझे\\
समझायें, अजब ज़माना आया\\
मेरा डॉक्टर लगे स्कूल का बच्चा\\
ज़रूर घोर कल युग है आया\\
\\
झुर्रियाँ जोड़ों के दर्द फूली साँसें\\
ये औरों की वसीयत है\\
मेरे जिस्म में कैसे घुस आयीं\\
गलत पते की मिली नसीहत है\\
\\
 भाई बहन झेलें बीमारियां बूढ़ों की\\
इक हल्का ख्याल दिल में आता\\
कल मेरा भी नंबर लगे गा\\
मैं ना रहूँ गा यकीन नहीं आता\\
\\
अब साथी संगी झड़ने लगे\\
शमशान के चक्कर लगने लगे\\
खुद को देखूँ चादर में लिपटा\\
सुनू लोग बातें मेरी करने लगे\\
\\
ये उम्र ना जाने कहाँ चली गयी\\
अब सोचूँ तरकीबें लंबी करने की\\
दवाइयों की लिस्ट बढ़ती गयी\\
सुनू बात ऑपरेशन और मरने की\\
अनगिनत दिन मिले थे मुझे\\
अधूरी तमन्नाएँ पूरी कर लूँ गा\\
कई ख्याल सपने थे मन में\\
जल्दी क्या है कल कर लूँ गा\\
\\
शाम ढली पर कल नहीं आया\\
इंसाफ़ कहाँ का है बतलाओ\\
मेरा सूरज डूब रहा\\
नई फसल को जा समझाओ\\
\\
भूल कड़वी यादें, मीठी बातों को\\
याद करो\\
खुल के जियो, दूजों का दामन\\
खुशी से भरो\\
चाहे लगे उम्र लंबी पर है छोटी\\
छोड़ शिकायत गिले पछतावा\\
इस पल को रौशन करो\\
इस पल को रौशन करो
}\end{verse}

\poemtitle{\texthindi{खोई जवानी
}}\begin{verse}[\versewidth]\texthindi{
खोई जवानी ढूँढे जब वो\\
हाथ छुड़ा कर चली गयी\\
आँख उठा ना देखा उस को\\
चुपके घर से निकल गयी\\
\\
बहता तेज़ झरना है जवानी\\
वापिस ना आए गिरता पानी\\
\\
जिस्म की ताकत खो गई\\
झुरियों की माला लिपट गई\\
\\
नशा किया कसरत ना की\\
खाना कई जन्मों का खाया\\
काम से फुर्सत मिली नहीं\\
पार्टियों में जा वक्त गँवाया\\
\\
अब बिन तेल की सब्ज़ी माँगे\\
मक्खन लदे पराँठे छोड़ दिये\\
खरीदने से पहले लेबल देखे\\
चीनी से रिश्ते नाते तोड़ दिये\\
\\
अंडे के पीले फेंक दिये\\
मिठाइयों से परहेज़ किये\\
पीते थे वाइन और बीयर\\
अब ईसबगोल हर शाम पिये\\
\\
देर आये दरुस्त आये\\
वरज़िश अब शुरू करें गे\\
लगन लगा लक्ष्य बना\\
बीमारियाँ अब दूर करें गे\\
\\
बिन मेहनत के ना माँ का\\
दूध मिले ना साँसें चलती\\
आलस छोड़ हिम्मत कर\\
जीवन का सुधार करें\\
\\
खोई जवानी को हिम्मत से\\
वापिस लाने की मेहनत करें\\
खोई जवानी को हिम्मत से\\
वापिस लाने की मेहनत करें
}\end{verse}

\poemtitle{\texthindi{दोस्तों की नई तस्वीरें
}}\begin{verse}[\versewidth]\texthindi{
पुराने दोस्तों की नई तस्वीरें देखता हूँ\\
खंडहरों में रंगीन महल ढूँढता हूँ\\
\\
वो अधखिले सपने, सितारों की चाहत\\
टूटे ख्वाबों के बेजान टुकड़े देखता हूँ\\
\\
झुर्रियों पीछे जवानी की निशानियाँ ढूँढता\\
चेहरों पे ठोकरों की परेशानियाँ देखता हूँ\\
ना जाने कब चमक पे बादल छा गये\\
घने बाल उड़े जो बचे रंग बदला गये\\
\\
जीवन के थपेड़े सीधी कमर झुका गये\\
बुढ़ापे की सारी निशानियाँ देखता हूँ\\
\\
फीकी मुस्कानों में हल्का दम है\\
भविष्य के सपने ख्वाहिशें कम हैं\\
\\
कोई मिलने ना आता आँखें नम हैं\\
कमज़ोर थके हारे पिंजर देखता हूँ\\
बीमारियों ने घर बनाया किसी को\\
कुछ चेहरे काल ने बेवक्त खा लिये\\
धुंधली यादों में खोये साथी देखता हूँ\\
\\
पुराने दोस्तों की नई तस्वीरें देखता\\
टूटे ख्वाबों के बेजान टुकड़े देखता हूँ
}\end{verse}

\poemtitle{\texthindi{बुढ़ापे के रंग
}}\begin{verse}[\versewidth]\texthindi{
बच्चों ने बुढ़ापा नहीं देखा\\
ये तो औरों की बीमारी है\\
मौत या उस का साया दूर\\
मस्त जवानी की खुमारी है\\
\\
या रब इन्हें कुछ दिन बुढ़ापा\\
और फिर से जवानी लौटा दे\\
फिर ये शायद समझ पायें\\
आते दिनों की कठिन राहें\\
\\
वो क्या जाने बुढ़ापे के गम\\
वो चार दीवारी की बंद घुटन\\
सन्नाटा सुनसान अकेलापन\\
\\
मौत का चारों ओर मण्डराना\\
साथी का अचानक छोड़ जाना\\
इक काठी से संभलना मुश्किल\\
सफ़र ज़िंदगी का अकेले निभाना\\
\\
ऊँचा सुनना अंधेपन का डर\\
रात में उठना फिर नींद ना आना\\
आ भी गयी तो ना उठने का डर\\
भूली यादों भरा गीला सिरहाना\\
\\
जन्म दिन पे मिले तोहफ़ा नया\\
जोड़ जुड़ें कैंसर ने धर लिया\\
फेफड़ों की बीमारी साँस फूलना\\
चढ़ती जवानी ने रुख मोड़ लिया\\
\\
कच्ची दीवारों पीछे सुनू बातें\\
क्यों ना जाते उस पार जहाँ से\\
बेवजह बैठे हैं ना कोई मकसद\\
पार क्यों नहीं करते ये सरहद\\
\\
बने रुकावट जीवन की राह में\\
पत्थर जैसे नदी के परवाह में\\
आखरी साँस इन की ये शायद\\
सुनता हूँ उन की दबी ज़बान में\\
\\
भूले इन्हीं हाथों ने चलना सिखाया\\
इसी आवाज़ ने बोलना समझाया\\
इन्हीं कन्धों पे बैठ दुनिया को देखा\\
पार की बचपन से जवानी की रेखा\\
\\
फिर सोचता मैं ने यही किया था\\
जब थी उन्हें मेरी ज़रूरत\\
मैं ने भी छोड़ा मुँह फेरा था\\
कल कर लूँ गा वो सदा रहेंगे\\
ना खत लिखा बात भी ना की\\
अपनी रंगीन राहों में खोया था\\
\\
जो बोया मिलता वही जीवन के खेत में\\
खिलते नहीं रंगीन फूल बंजर सूखी रेत में\\
\\
लाचारी बच्चों की हम समझ पायें\\
संवारे अपनी क्यारी और गायें\\
उन को मिली एक ही फुलवारी\\
जिस के खिलने की खुशी मनायें\\
\\
पलक झपकते ये बचपन से हुए जवान\\
ना जाने कब मैं जवान से बूढ़ा हो गया\\
कमर झुकी चेहरे पे झुर्रियाँ मेरी पहचान\\
रंगे जवानी राह में जाने कहाँ खो गया\\
\\
मिलती एक बारी ज़िंदगी के खेल में\\
इक तरफ़ा रास्ता है जीवन के मेल में\\
गुज़रा वक्त ना लौट के आए कभी\\
माँ बाप का अपना, बच्चों का\\
ख्याल रखो मेल में\\
हर पल को चमक सुनहरी दो\\
इस सुंदर छोटे से खेल में\\
\\
जितना दिया है रब ने\\
वो झोली से भी ज़्यादा है\\
भरपूर बचपन जीवन बाद\\
बुढ़ापे का गम आधा है\\
भरपूर बचपन जीवन बाद\\
बुढ़ापे का गम आधा है
}\end{verse}

\poemtitle{\texthindi{आल्ज़ाइमर्ज़
}}\begin{verse}[\versewidth]\texthindi{
(बुज़ुर्ग एक जवान बच्चे से बात कर रहा है।\\
अक्षर बदल कर ये ख्याल आदमी या औरत\\
के लिए लिखा जा सकता है।
)\\
याद है रात भर सर्द पट्टियाँ तेरे सर पे लगाया\\
करता था\\
लिटा गोद में लगा सीने बाल सहलाया करता\\
था\\
तेरी बीमारी मुझे मेरी उम्र तुझे\\
रब से भीख माँगा करता था\\
\\
तेरी साँसों की आवाज़ सुनने\\
अपनी साँस थामा करता था\\
\\
चोट लगती तुझे दर्द होता मुझे\\
दिल रातों में रोता था\\
तेरी खुशी के खज़ाने बना मोती\\
दिल में हार पिरोता था\\
\\
छुपा दर्द, जीवन तेरा रंगीन\\
बनाने के साधन सोचता था\\
दुनियाँ की दर्दनाक खबरें\\
तेरे कानों से रोकता था\\
\\
हटा नोकीले काँटे राहों से\\
फूल बिछाया करता था\\
ना दुःख दे ना मुश्किल तुझे\\
मैं दुनियाँ से झगड़ता था\\
\\
जब किसी ने दिल तोड़ा तेरा\\
दिल मेरा भी मुरझाता था\\
देख तेरा दिलदार साथी\\
मैं मन ही मन मुसकाता था\\
\\
फुलवारी खिलते देख तेरी\\
दिल मेरा भी खिल जाता था\\
सब को हँसता फलता देख\\
अंदरूनी सुकून मिल जाता था\\
\\
तुम कम मिलते, व्यस्त थे\\
झलक पाने को तरसता था\\
हर पल यादों का ले सहारा\\
दिन रात गुज़ारा करता था\\
\\
ना जाने क्यों, कब, कैसे\\
यादों पे बादल छाने लगे\\
कहाँ था मैं कौन था मैं\\
संगी साथी दूर जाने लगे\\
\\
गुज़री पुरानी यादें ज़िंदा हैं\\
जैसे कल की ही बात है वो\\
जो कल बीता या आज हुआ\\
भूला ज्यूँ काली रात है वो\\
\\
जिस हाल में हूँ खुश हूँ बहुत\\
शायद तुम्हें मालूम नहीं\\
मेरा सुंदर अतीत मेरी दुनिया है\\
शायद तुम्हें मालूम नहीं\\
\\
ना जानू तुझे ना पहचानू तुम्हें\\
पर यादें तेरी ज़िंदा हैं दिल में\\
इस ख्याल से रहना खुश बेटी\\
बस तू ही तू रहती है दिल में\\
इस ख्याल से रहना खुश बेटी\\
बस तू ही तू रहती है दिल में
}\end{verse}

\poemtitle{\texthindi{हैं से थे बन गये
}}\begin{verse}[\versewidth]\texthindi{
न जाने कब हैं से थे बन गये\\
धूप पे काले बादल छा गये\\
कल कलियों से फूल खिले\\
कल आते सूख के मुरझा गये\\
\\
कल थे हम हसीन जवान\\
अपना नया ज़माना था\\
रंगीन सावन को ज़ालिम\\
पतझड़ खाक बना गये\\
\\
बर्फ़ करे फ़ख्र चमक का\\
नासमझ माना जाये गा\\
कड़ी गर्म धूप बेरहमी से\\
पिघला के पानी बना गये\\
\\
समंदर में लहरें उठें बलखाती\\
छूना चाहें आसमान\\
दो पल में गिरें बेनाम\\
वजूद पानी में समा जाये\\
\\
ये नाम सूरत धन ओहदा\\
चन्द दिनों के हैं मेहमान\\
समय बहुत बलवान है\\
वक्त की चादर से ढके गये\\
\\
जियो इस पल में गले लगाओ\\
आज को गम चिंता से न जलाओ\\
तेरे जैसे लाखों करोड़ों\\
इसी राह से आ कर गुज़र गये\\
\\
न जाने कब हैं से थे बन गये\\
धूप पे काले बादल छा गये\\
कल कलियों से फूल खिले थे\\
कल आते सूख के मुरझा गये\\
कल आते सूख के मुरझा गये
}\end{verse}

\poemtitle{\texthindi{जीवन का खेल
}}\begin{verse}[\versewidth]\texthindi{
बदल जाते खिलाड़ी पर\\
खेल चलता जाता है\\
एक खिलाड़ी चला गया\\
दूजा उस जगह आ जाता है\\
\\
खिलाड़ी टीम की शक्लें नाम\\
कायदे कानून बदले\\
लाख कोशिशें करने पर भी\\
कोई बच के नहीं जा पाता है\\
\\
जीतने खातिर झगड़ा लड़ा\\
कुछ जीता कुछ चुराया\\
आगे बड़ने के लालच में परेशान\\
खेल का मज़ा ले ना पाया\\
\\
झूठ बोला तरकीबें बनायीं\\
स्वार्थी बन दूजों को गिराया\\
ये खेल है, खत्म हो गा भूला\\
बस अपनी धुन में ही समाया\\
\\
हर खिलाड़ी सोचे वो है पहला \\
खेल में, उसी वास्ते खेल रचाया\\
आया मैदान में खेला कूदा\\
बड़ा हुआ चिल्लाया शोर मचाया\\
बादल गरजा बिजली चमकी\\
फिर आँसुओं का झरना बरसाया\\
\\
चुपके हार गिरा पस्त हुआ\\
लोगों ने कंधों पे उठा के हटाया\\
देख जगह खाली अगले खिलाड़ी ने\\
उछलते कूदते कदम बढ़ाया\\
ना शुक्रिया उन का जो पहले आये\\
उस के लिए मैदान बनाया\\
\\
धीरे धीरे थका कदम उठाना मुश्किल\\
खेल के अंत का ध्यान आया \\
बहुत देर हो चुकी थी संभलते\\
नये खिलाड़ी ने बेरहमी से आन गिराया\\
\\
बदल जाते खिलाड़ी पर\\
खेल चलता जाता है\\
एक खिलाड़ी चला गया\\
दूजा उस की जगह आ जाता है
}\end{verse}

\poemtitle{\texthindi{बहुत देर
}}\begin{verse}[\versewidth]\texthindi{
आये मेरी मैयत पे गिरा आँसू\\
दो फूल चढ़ा के चले गये\\
पुल बांधे तारीफ़ों के\\
अनगिनत खामियाँ भुला गये\\
\\
ना बरसों मिले बात भी ना की\\
अब मुँह दिखलाने आ गये\\
काश वक्त साथ गुज़ारा होता\\
ज़िंदा रहते तोहफ़ा भेजा होता\\
गिरे को सहारा, पोंछता आँसू\\
रातों को रौशन किया होता\\
आँख तरस गई फिर नम हुई\\
राह देखते नैन थक सूख गये\\
बहुत देर कर दी आने में\\
आये तुम जब हम ही ना रहे\\
तेरा नाम लेते जहाँ से गुज़र गये\\
तेरा नाम लेते जहाँ से गुज़र गये
}\end{verse}

\poemtitle{\texthindi{मोम के पुतले
}}\begin{verse}[\versewidth]\texthindi{
जाने कब ये मोम पिघल जाये\\
ढाँचा भी आग में जल जाये\\
दो दिन के हैं मेहमान सभी\\
मुट्ठी राख पानी में मिल जाये\\
\\
कल साथ था मेरे अब साया नहीं\\
दरवाज़ा खुला पर वो आया नहीं\\
दो पल बाहर गया लौट आये गा\\
यादें छोड़ गया फिर आया नहीं\\
\\
दो पल उस के गीत गाये\\
तस्वीर पे इक दो फूल चढ़े \\
याद रखने का वादा किया\\
फिर अपनी राह पे चल पड़े\\
भूले नाम जो दो दिन पहले मरे\\
किसे है वक्त बेवा का पूछे हाल\\
अपने मकड़ी जाल में सभी घिरे\\
\\
जिस की मैयत पे आये हो\\
वो कल किसी और पे आया था\\
हम आये हैं आज अलविदा कहने\\
कल कोई हमें विदा करने आये\\
\\
जाने कब ये मोम पिघल जाये\\
ढाँचा भी आग में जल जाये\\
दो दिन के हैं मेहमान सभी\\
मुट्ठी राख पानी में मिल जाये
}\end{verse}

\poemtitle{\texthindi{जन्म मरण
}}\begin{verse}[\versewidth]\texthindi{
जब बर्फ़ की तितलियां पिघलती हैं\\
पेड़ से पत्ता गिरता है\\
घर छोड़ सितारा चमके कुछ पल\\
मिट्टी में जा मिलता है\\
\\
तेज़ शोरगुल पानी नदी का\\
चुप सागर में मिलता है\\
साथी संगी मित्र नाता तोड़\\
दुनिया से बिछड़ता है\\
\\
जिन संग खेले बड़े हुए हँसते गाते\\
अपने सामने मरता है\\
सोच के अपना अंत\\
खामोश ख्याल दिल में उभरता है\\
\\
शमशान में लेटा जिस्म है दूजा\\
चेहरा अपना दिखता है\\
क्या यही है ज़िंदगी जिस खातिर\\
इंसान दिन रात भटकता है?\\
\\
बैठ पेड़ की छाँव में, काश\\
उन संग समय बिताया होता\\
मात पिता भाई बहनों का संग\\
किस्मत वालों को मिलता है\\
\\
अधखिले सपने यादों का गुबार\\
फूट के बाहर निकलता है\\
कुछ दिन संभल जाता इंसान\\
फिर झूठी राह पे फिसलता है\\
\\
काम क्रोध मद अहंकार का\\
फिर धीरे धीरे कै दी बनता है\\
अक्लमंद पहचान अस्ल को\\
जल्द सीधी राह पकड़ता है\\
\\
ना आने का वक्त है हाथों में\\
ना ही यहाँ से जाने का\\
कुदरत का जन्म मरण का खेल\\
अपने असूलों से चलता है\\
कुदरत का जन्म मरण का खेल\\
अपने असूलों से चलता है
}\end{verse}

\poemtitle{\texthindi{मौत
}}\begin{verse}[\versewidth]\texthindi{
मौत को इतने करीब देखा है\\
अब जानी पहचानी लगती है\\
दिखता उस का रूप आईने में\\
इक सच्ची कहानी लगती है\\
\\
जब जान लिया पहचान लिया\\
अब उस की चिंता कम लगती\\
जीवन किताब का पहला अक्षर\\
आखरी पन्ने की निशानी लगती\\
\\
शरीर नाशवान सुना पढ़ा और देखा\\
साफ़ दिखा अपनों को जलते देखा\\
हो गा यही हाल कल मेरा खेल तमाम\\
चादर ढका घर वालों को रोते देखा\\
\\
मुश्किलें परेशानियां खत्म हो जायें गी\\
जीवन मौत की रेखा मिट जाये गी\\
जीवन सूर्य उगता ले उमंगें लाखों\\
शाम सूरज की तपस ढल जाये गी\\
\\
शोहरत पैसा शरीर गुमान\\
चंद दिन मेहमान घर में आये\\
पलक झपकते रुखसत हों गे\\
पंछी के पर निकले उड़ जाये\\
\\
गस्सा गरुर चिंता लोभ ना रखो\\
झुलसें जिस्म संग इक पल में\\
जोड़ा खज़ाना यहीं रह जाये\\
\\
करो तमन्नायें पूरी बाँटो प्यार\\
ढलता जिस्म सदा रह ना पाये\\
ये जीवन मिला है कुछ पल\\
हँसो हसाओ खुशी से जी भर\\
बीते कल की भूलें भुला दो\\
अगला पल आये ना आये\\
साँसों का आना जाना नियामत\\
ना जाने ये ताँता कब रुक जाये\\
ना जाने ये ताँता कब रुक जाये
}\end{verse}

\poemtitle{\texthindi{यादों के खंडरात
}}\begin{verse}[\versewidth]\texthindi{
घर भरा हुआ लोगों से पर मेरे लिए खाली है\\
हर शै पे नाम लिखा हर पौधे का तू माली है\\
\\
बच्चों में छवि तेरी दोहते दोहती में रूप तेरा\\
उन की सोच हरकतों के पीछे देखूँ हाथ तेरा\\
\\
नानी बीवी मौम, दोस्त नाम से याद करें गे\\
बारी बारी चले गए मैं और तेरी यादें रहें गे\\
देखूँ जिधर मैं नज़र तू आये\\
बोलूँ पर जवाब ना आये\\
मौत के सामने बस नहीं\\
आँखों ने आँसू छलकाये\\
\\
दूजों संग होते बहुत सुना था\\
इतना मुश्किल हो गा ये\\
ऐसा कभी ना सोचा था\\
\\
अब मालूम हुआ तो ज़िंदगी\\
वापिस लाना चाहता हूँ\\
लाखों ख्याल आते मन में\\
तेरी बातें सुनना चाहता हूँ\\
तेरी ख्वाहिशें इच्छायें\\
पूरी करना चाहता हूँ\\
काश समझ पाता मुश्किलें तेरी\\
सोच पिछले कुछ सालों की\\
वो तू नहीं तेरी बीमारी थी\\
ले ज़बान शिकायत आँसुओं की\\
\\
कौन रोके टोके गा मुझे\\
ऐसे सवालों में डूबा रहता हूँ\\
अकेले बैठ जीवन भर\\
गम में खोया उलझा रहता हूँ\\
जब भी आहट होती सोचता हूँ\\
लौट के तू आ जाए गी\\
अकेले अब चलते उम्र भर की\\
सज़ा भुगतनी पाए गी\\
\\
कम्बख़्त वक्त बहुत ही ज़ालिम\\
पंछी की तरह उड़ जाता\\
लाख करो दुआएँ घड़ी की\\
सुई का काँटा लौट ना आता\\
\\
यही है पल जिस में है जीवन\\
इस को अपनों पे लुटा डालो\\
जो तुम से प्यार करें\\
उन खातिर खुदी मिटा डालो\\
\\
देखो मुझे मेरे गम मजबूरी\\
छलकते आँसुओं को\\
उसकी मौत मेरे दुख से सीखना है\\
तो उस के लिये ज़िंदा रहना सीखो\\
वो चले गए तो सर पकड़ रोओ गे\\
तनहा यादों की माला पिरोओ गे\\
\\
यादों के खंडरात का कैदी रात मेरी काली है\\
घर भरा हुआ है लोगों से मेरे लिए ये खाली है
}\end{verse}

\poemtitle{\texthindi{दुआ
}}\begin{verse}[\versewidth]\texthindi{
(अपने एक प्रिय मित्र को खुले कास्केट में\\
और परिवार को रोते देखने के बाद ये कविता\\
लिखी थी
)\\
साँस रुकी धड़कन दिल की\\
फिर भी नज़ारा देख सकता हूँ\\
तेरा रोना देखा ना जाता\\
ना आँसू पोंछने आ सकता हूँ\\
\\
देखूँ बच्चों को रोते बिलखते\\
हाथ पकड़ते फूल चढ़ाते\\
गर्मी बारिश में बादल बन छाता\\
सदा नहीं मैं रह सकता हूँ\\
\\
मुझे तो मुक्ति मिल गई\\
जो देखना करना बहुत किया\\
साथ तुम्हारा मिला किस्मत से\\
जीवन का रस बहुत पिया\\
\\
ना सोचा ना माँगा फिर भी\\
किस्मत ने झोली भर दी\\
ज़मीन से छू लिया आसमान\\
हम को रब ने बहुत दिया\\
\\
जितनी साँसें मिलीं तुम्हें\\
हँसते खेलते पूरी करना तुम\\
खुल के जीना मेरे हमदम\\
मेरे हिस्से का भी जीना तुम\\
\\
दुगनी खुशियाँ बाँटना हमेशा\\
दुगने पोंछना औरों के आँसू\\
मीठी यादों का ले के सहारा\\
दूजों की काठी बन जाना तुम\\
\\
इक रोज़ तेरा भी दिन आए गा\\
तू भी यहीं पर लेटी हो गी\\
जिन को जन्मा बड़ा किया\\
सर थामे तेरी बेटी हों गी\\
\\
करता हूँ दुआ वो दिन आये\\
इक लम्बे अरसे के बाद\\
करूँ गा तेरा इंतज़ार\\
मुलाकात अगले जन्म हो गी\\
\\
साँस रुकी धड़कन दिल की\\
फिर भी नज़ारा देख सकता हूँ\\
तेरा रोना देखा ना जाता\\
ना आँसू पोंछने आ सकता हूँ
}\end{verse}

\poemtitle{\texthindi{जीवन और मौत
}}\begin{verse}[\versewidth]\texthindi{
हवा का झोंका, बुलबुला या उड़ते बादल\\
का साया\\
इक सपना है जीवन, चमकता सुनहरी मृग\\
छाया\\
भूले नशीली जवानी में, शरीर नाशवान जल\\
जाए गा\\
जिस खातिर की चोरी धोखे, दिल अपनों का\\
दुखाया\\
जन्म से अंत नामी जिल्द से\\
बंद किताब में समाया\\
जीवन गाथा खुद लिक्खी या\\
किस्मत ने लिखवाया\\
\\
अगले साँस की खबर नहीं\\
कई जन्मों की करी तैयारी\\
काम क्रोध मद लोभ माया में\\
भूले भटके जन्म बिताया\\
\\
जब थे ज़िंदा लोग ना मिलते\\
पीछे पीठ बातें कोसा करते\\
आँख बंद हुई बड़ी कतार लगा\\
पुल तारीफ़ों के बांधा करते\\
\\
वो भी शरम से मिलने आते\\
जो सालों घर ना आये थे\\
दो पल मिलना मुश्किल था\\
अब दिन भर मुँह दिखाई करते\\
\\
मरने बाद इतने फल फूल मिले\\
जो ज़िंदा होते ना पाए\\
जितनी बड़ाई मरने बाद हुई\\
ज़िंदा रहते सुन ना पाए\\
\\
ज़िंदा थे लोग गलतियों खामियों\\
पर तवज्जो देते बातें करते\\
जाने बाद अच्छाइयों की चर्चा करते\\
\\
जब था मैं बीमार अपाहिज \\
कोई सालों ना मिलने आता\\
अब नहीं हूँ मिलने वालों का\\
ताँता खत्म ना होने पाता\\
\\
दूजों से अच्छा लगने खातिर\\
कोई पक्की दोस्ती जताता\\
अब बना हमदर्द जो पहले\\
आँख मिलाने से कतराता\\
\\
बीवी, शौहर या अपने छोड़\\
सब ढोंगी नज़र आते\\
कोई चुपके खून के आँसू छुपाता\\
कोई नकली पानी छलकाते\\
\\
किसी की ज़िंदगी इक पल में\\
आसमान से मिट्टी में मिल जाती\\
कोई ज़िंदा रहते उँगली ना पकड़ता\\
अब कंधों का सहारे दिलाते\\
\\
इक पल ज़िंदा फिर साँस धड़कन रूकती\\
लाख यत्न करे इंसान, जीत मौत की होती\\
शव देख कर दो पल अपने अंत का सोचें\\
जल्द भूलते जीवन का सच\\
नकली कहानी शुरू फिर होती\\
\\
हवा का झोंका, बुलबुला या\\
उड़ते बादल का साया\\
इक सपना है जीवन\\
चमकता सुनहरी मृग छाया\\
चमकता सुनहरी मृग छाया
}\end{verse}

\poemtitle{\texthindi{जीवन का अन्त
}}\begin{verse}[\versewidth]\texthindi{
बहुत अजब है जीवन\\
चन्द बरसों में मेरा नाम ना हो गा\\
हमारे ख्वाब किस्से ऊँचाइयाँ\\
गहराइयाँ कोई ना जाने गा \\
घर बनाया सजाया चमकाया\\
मिट्टी का ढेर बन जाए गा\\
\\
हमारी कश्मकश मेहनत काम\\
किसी को याद नहीं\\
कौन था मैं क्या क्यों करता था\\
पूछने का वक्त ना हो गा\\
\\
अपनी दुनियाँ के चर्चे\\
गर पूछें तो लोग हैरान हों गे\\
किस दुनियाँ से आए हो\\
खिलाड़ी बदले खेल वही हों गे\\
\\
गुज़रे जीवनों के क़िस्से\\
किसी को याद ना हों गे\\
\\
बहुत अजब है जीवन\\
चन्द बरसों में नाम ना हो गा\\
गनीमत समझो दो पीढ़ियों तक\\
शायद मेरा नाम तो हो गा \\
शायद मेरा नाम तो हो गा
}\end{verse}

\poemtitle{\texthindi{समय की धूल
}}\begin{verse}[\versewidth]\texthindi{
समय की धूल सदियों को छुपा जाती \\
समुद्र की लहर रेतीले महल मिटा जाती \\
\\
चलती फिरती बोलती बड़ती ज़िंदगी\\
साँस रुकने से पल में खत्म हो जाती\\
\\
तस्वीरें यादें नाम काम कुछ पल ज़िंदा\\
अगली पीढ़ी जीने में मग्न, भुला जाती \\
\\
शिकवे गिले दो पल के मेहमान\\
शिकायत करने सुनने वाले ना रहे \\
इक उम्र गुज़रने पे समझ आती\\
\\
सूरत पैसा ओहदा मृग छाया\\
सदा इन्हें कोई बांध ना पाया\\
सावन के फल फूल रंगीन पत्ते \\
पतझड़ की तूफ़ानी तेज़ हवा \\
उड़ा बिखरा मिट्टी में मिला जाती\\
\\
राजे महाराजे वक्त की नदी में बह जाते    \\
 ऊँची जात के अभिमानी राख बन जाते \\
दस नामो के बादशाह बेनाम बन जाते \\
समय की भूख मिट्टी को मिट्टी से मिलाती \\
\\
हैसियत खाक में मिल जाती\\
एक दो पीढ़ी तक रहती निशानी \\
फिर समय की गोद में सो जाती\\
\\
समय की धूल सदियों को छुपा जाती \\
समंदर की लहर रेतीले महल मिटा जाती
}\end{verse}
%\end{multicols}