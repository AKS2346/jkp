\documentclass{book}
\usepackage{verse}
\usepackage{polyglossia}
\usepackage{titlesec}

% Set up fonts to use Hindi
\newfontfamily\devtransl[Mapping=DevRom]{Times New Roman}
\setmainlanguage{english}
\setotherlanguages{hindi}
\newfontfamily\hindifont{Noto Sans Devanagari}[Script=Devanagari] % Use any Devanagari font on your system

% set up formating for the verse package
\newcommand{\attrib}[1]{\nopagebreak{\raggedleft\footnotesize #1\par\newpage}}
\renewcommand{\poemtitlefont}{\normalfont\large\bfseries\centering}

% Create the chapter format using titlesec
\titleformat
{\chapter}
[display] %shape
{\bfseries\LARGE} %format
{Chapter \thechapter} % label
{2cm} %sep
{} % before code
[\newpage] %after code

\begin{document}
\title{\texthindi{कवितायें \\
\large आत्मिक, परिवार, जीवन, बुढ़ापा, मौत}}
\author{}
\maketitle
\tableofcontents
\chapter{Hindi Poems}
\poemtitle{\texthindi{हमारा बचपन}}
\begin{verse}
{
\texthindi{
बच्चे चार साइकिल दो से
इतनी ज़्यादा ख़ुशी
हम आठ, साइकल एक,
दुगनी थी हमारी ख़ुशी
इक निक्कर क़मीज़, चप्पल का जोढ़ा ख़ज़ाना था
हर ज्श्न मनाते धूम धाम से,
प्यार भर देता था ख़ुशी

माँ बाप मुसकाते
चुपके पीते ज़हर,
शहद हमें पिलाते थे
ख़ुद रह भूखा मक्खन लदे
पराँठे हमें खिलाते थे
इक बादशाही ज़िंदगी से
इक दिन में बने खानाबदोश
ना जाने कैसे हस के,
राज गद्दी पे हमें बिठाते थे

पेड़ों पे आम अमरूद नहीं,
मीठा अमृत रस मिलता था
तंदूर से आग नहीं, नर्म सेक
दिल को सकून मिलता था
पैसा एक ना पल्ले
घर शीश महल दिखता था
ख़ुशियों के फवारे गूंजते,
बेफ़िक्र सुख चैन मिलता था

नाम पानीपत पर अक्सर नल में पानी नहीं था
दो Hand पम्प थे कसरत, कोई गिला नहीं था
कभी आयी कभी गई, बिजली खेले आँख मिचौली
हाथ के पंखे, मोम बत्तियाँ, कमियों का पता नहीं था

कभी गुलि डंडा पिठ्ठु,
कभी cricket की थी बारी
कँचे लुक्कन छुप्पी झूला
गुलेल से पथरी मारी
पड़ाई क्या है चीज़
उस बारे कम सोचा था
अभी है बचपन खेलो कूदो,
पड़ने लिखने को उम्र है सारी

हवा में पतंगें फल फूल ज़मीं पे
भर देते रंगीन नज़ारा
ना परवाह दूजों पास है क्या
घड़ा रहता भरा हमारा
आँगन दिन में खेल मैदान
मछरदानी में तारों के नीचे
सोने का कमरा हमारा

बचपन के अनमोल दौर की
तस्वीरें मन में जब खोलूँ
ना ग़म ना ज़्यादा सपने, बस वर्तमान ही काफ़ी था

स्वामी जी शकुन्तला दर्शी
माँ का आशीर्वाद बरसता है
ऐसा सुंदर सुहाना बचपन
क़िस्मत वालों को मिलता है

जैसे हवा में ख़ुशबू, तालाब में रंगीन कमल खिलता है
ऐसा सुंदर बचपन क़िस्मत वालों को मिलता है
ऐसा सुंदर बचपन क़िस्मत वालों को मिलता है

जुगिंदर लूथरा
}
}
\end{verse}
\attrib{JK Luthra, April 8, 2023}

\poemtitle{\texthindi{
एक फूल 
की कहानी उस 
की ज़ुबानी
}}

\begin{verse}
{
\texthindi{

खुली हवा में मस्ती से मैं इठलाता लहराता था \\
महक उभरती शोख़ लबों से गीत मधुर मैं गाता था 

साथ थे मेरे संगी साथी रंग बिरंगे और निराले \\
भँवरे तितली पीते रस उन का जो छलकते क़िस्मत वाले 

ले के चुम्बन एक फूल का दूजे पर वो जाते थे \\
सात सुरों से मुझे रिझाने गीत ख़ुशी के गाते थे 

दूजे से सुंदर लगनें के नए नए साधन अपनाते \\
देख छवि पानी में अपनी ख़ुशी से इतराते शर्माते 

बहुत सुंदरता अच्छी भी है और बुरी भी \\
रंगों से भरी जवानी अच्छी भी है और बुरी भी 

चाहत भरा चुम्बन प्यार से सहलाना अच्छा लगता था \\
अफ़सोस मेरा सुंदर चेहरा फूलों के व्यापारी को अच्छा लगता था 

भरी जवानी रंगों से लदे फूलों की तालाश में था \\
मुझ पर नज़र पढ़ी पर मुझे अभी अहसास ना था 

उस की काली नज़र नें मुझ में शोहरत पैसा देखा \\
गुलदस्ते में ख़ूब सजे गा ये रंगीं सुंदर चेहरा 

मन ही मन उस क़ातिल नें बुरी नज़र से सोचा था \\
बेख़बर अनजान था मैं क़ातिल में आशिक़ देखा था 

ज़ुल्मी नें चमकती कैंची निकाली अपने झोले से \\
पकड़ के गर्दन अलग किया मुझे मेरी ही माँ से 

एक ही झटके से दर्द दिया लाखों सोए सपने तोड़े \\
दो पल की ख़ुशी की ख़ातिर मुझे और मेरे साथी जोड़े 

बेदरदी नें बाँध रस्सी से हम सब को क़ैदी बनाया \\
पानी भरे शीश महल में बेचारों को ख़ूब रुलाया

शादी की ख़ुशी मनाने मेज़ों पर सब को सजाया \\
जिस की ख़ातिर हम नें जान गँवायीं ख़ून बहाया \\
वो मस्ती में माशूक़ा से लिपटा झूमा लहराया 

इक बार भी उस नें मुझे ना देखा ना क़ुर्बानी मेरी \\
शाम ढली फिर रात हुई धीमे से बात हुई मेरी 

"बहुत सुंदर हैं ना ये फूल, बहुत महंगे हों गे" \\
मन रोया सुन ज़िंदगी के सपने पैसे में तुलते 

मेरा रोना ना किसी नें देखा ना दुःख समझ पाया \\
खाना पीना ख़त्म शुक्रिया तक कोई कह ना पाया 

कुछ क़िस्मत वाले गए किसी संग घर सजाए \\
कुछ मेरे जैसे बेदरदी कचरे में जा समाए 

क्या सोचा था क्या क्या सपनें दिल में थे बनाए \\
अपनी ज़िंदगी रंगीं हो गी कई महीनों खिलें गे \\
अपना इक संसार नया हो गा बीज बच्चे मिलें गे 

कोई क़िस्मत से लड़ नहीं पाता \\
लिखा कोई मिटा नहीं पाता \\
अब दम तोड़ रहा हूँ इक कूड़े के बर्तन में \\
पानी में मिले मेरे आँसू पर दिल से दुआ देता हूँ 

ख़ुश रहें जिन कि ख़ातिर मेरा क़त्ल हुआ ख़ून बहा \\
लम्बी उम्र हो उन की बेवक्त ना उन को कोई काटे \\
आँखें बन्द किए बेहोशी में बस यही दुआ देता हूँ \\
लम्बी उम्र हो उन की बेवक्त ना उन को कोई काटे
}
}\end{verse}



\poemtitle{\texthindi{शब्द शक्ति}}
\begin{verse}
{
\texthindi{
शब्दों की मंडी में लफ़्ज़ है लाखों आओ उठा लें जिसे भी चाहें
कोई काँटों के साथ जुड़े कोई महके फूल गले में डाले बाहें

कोई लोगों में बाँटे ख़ुशियाँ
कोई ज़ख़्म दे काँटों से ज़्यादा
कोई दुःख को दुगना चौगुना कर दे कोई उस को कर दे आधा

शब्द ऐसे भी देखता हूँ जो रोते दुखियों के आँसू सुखाएँ
कुछ ऐसे चुभते हैं जो ख़ुशक़िस्मत हँस्ते लोगों को रुलाएँ

दो शब्द तारीफ़ के गिरे पिछड़ों को फिर चलना सिखाएँ
जली कठोर निकली होठों से बातें उमर भर कैसे बिसराएँ

मीठा शब्द इक अजनबी को महबूब जीवन साथी बनाए
बेसमझि से ग़ुस्से में कहे
ना जुड़ने वाली दीवार बनाए

पीठ पीछे कही कड़वी बातें
जब उन तक पहुँच जाएँ
सालों के बनाए रिश्ते नाते
पल में टूटें फिर जुड़ ना पाएँ

कुछ शब्द हैं जिन्हें बिन कहे
लब आँखें बेहतर समझाएँ
कई ऐसे हैं जिन्हें कहनें पर
ख़ुद शर्म से आँखें झुक जाएँ

कुछ कलियों में छुपा के
काँटों से चुभाया करते हैं
दिखा के सब्ज़ बाग़ झूठी
तरकीबों से लुभाया करते हैं

कुछ बेवक्त बेहूदा कहे शब्द
लोगों को परेशान करते हैं
कहने वाले को ख़बर नहीं
सुनने वाले दिल ही दिल मरते हैं

“चिपके गाल वज़न कम कमज़ोर हो गये हो” बातें कही जायें
कठोर लफ़्ज़ अच्छा होते
मरीज़ को फिर से बीमार बनायें

कुछ शब्द जिन्हें कहना चाहता हूँ वो सुनने वाले ही ना रहे
वो क़ैदी बन चुपचाप मन में तड़पे फिर आँसुओं में जा बहे

शब्द बहुत ही शक्तिमान
दुनियाँ को बदल डाले इक इंसान
शब्दों को तोलो ये वापिस ना आएँ जैसे तीर नें छोड़ा कमान
शब्दों को तोलो ये वापिस ना आएँ जैसे तीर नें छोड़ा कमान
}
}
\end{verse}
\attrib{JK Luthra, April 8, 2023}

\chapter{English Poems}
\poemtitle{Be a Sun}
\begin{verse}
{
Illuminate whatever you touch
Be a giver, receivers misery too much
 
Your light free for all,
seek nothing back
Recipients circle, spin,
keep coming back
 
Give life to others, unaffected by them
They use or misuse not for you to judge them
 
Others may take you for granted
Keep glowing even if feel unwanted
 
You were born to shine, stay detached
Spend days giving,
no strings attached
 
Be not arrogant of your bright rays
One who made you gave limited days
 
So my daughter and my son
Stay bright and giving like a sun
Be a sun
}
\end{verse}
\attrib{JK Luthra, 2022}

\poemtitle{Spring}
\begin{verse}
{
Summer grants trees abundant leaves and fruits
Buds bloom flowers blossom
Pearly dew covers fresh shoots

What life gives, in time it snatches away
Autumn hits us all
Bad times, like winter Will hit us all,

Bare branches suffer deadly ice
Burden of heavy snow
Earth spins around sun, gets life it bestows

Ice n snow accept defeat, drip meekly away
Have faith in God when ups downs come your way

Have no doubt, bad times will pass as they came
Spring will arrive for sure, buds will bloom again
What was taken away mercilessly will return again

Bad times, like winter, hit us all
Stay sturdy, hang in like brown branches
Spring will arrive again, again and yet again
}
\end{verse}
\attrib{JK and Dolly Luthra}

\end{document}
