\documentclass{book}
\usepackage{verse}
\usepackage{polyglossia}
\usepackage{titlesec}

% Set up fonts to use Hindi
\newfontfamily\devtransl[Mapping=DevRom]{Times New Roman}
\setmainlanguage{english}
\setotherlanguages{hindi}
\newfontfamily\hindifont{Noto Sans Devanagari}[Script=Devanagari] % Use any Devanagari font on your system

% set up formating for the verse package
\newcommand{\attrib}[1]{\nopagebreak{\raggedleft\footnotesize #1\par\newpage}}
\renewcommand{\poemtitlefont}{\normalfont\large\bfseries\centering}

% Create the chapter format using titlesec
\titleformat
{\chapter}
[display] %shape
{\bfseries\LARGE} %format
{Chapter \thechapter} % label
{2cm} %sep
{} % before code
[\newpage] %after code

\begin{document}
\title{\Huge \texthindi{जीवन यात्रा}}
\author{\huge \texthindi{डॉ जुगिन्दर लूथरा}}
\maketitle
\tableofcontents
\newpage
\chapter{\texthindi{क्रम}}
\begin{verse}
{
    \texthindi{
    इस किताब की यात्रा को पूरा करने में बहुत

    लोगों ने सहायता की है। मेरी तुकबंदियों को

    कविता का दर्जा दे कर मुझे प्रोत्साहन दिया

    है। उन के सहयोग और प्रोत्साहन के बिना

    इसे पूरा करना संभव नहीं था।

    डॉ शिववरण सिंह रघुवंशी, जयदेव तनेजा, कृष्णा शर्मा, आरती पिंटो, पंकज महरोत्रा का इस किताब के पूरा होने में हाथ है। मेरी पत्नी, डौली लूथरा ने कविताओं को सुना और संवारा। नमिता रोहिनी और रश्मी ने किताब लिखने लिए प्रोत्साहित किया। प्रेम लूथरा ने कविता लिखने की योग्यता देखी। 

    हमारे दोहते, अर्जन बीर सिंह ने फूलों की तस्वीर स्वयं लेने के बाद पुस्तक का आवरण बहुत प्यार और लगन के साथ बनाया। नमिता लूथरा ने कई सुझाव दिये। 

    अंत में, उन अनगिनत व्यक्तियों का धन्यवाद

    करता हूँ जिन्होंने इस यात्रा में अपने

    अनुभव, कहानियाँ और सुझाव मेरे साथ

    साझा किये—आप सभी ने इस पुस्तक में

    हो और योगदान दिया है।आप ने दिल से सराहना कर के मेरा हौसला बढ़ाया।

    यह पुस्तक मैं अपने गुरु जी को, अपने परिवार और आप सब को समर्पित करता हूँ

    जुगिन्दर लूथरा\par
    }
}
\end{verse}

\poemtitle{\texthindi{क्रम}}
\begin{verse}
    {\texthindi{

     आध्यात्मिक

    भगवान

    स्रोत

    सर्वव्यापी

    मोक्ष

    दो दिन का मेहमान

    असली रूप

    खुदी
    
    नाशुक्रा

    दो राहें

    जीवन का मकसद\par
    }
    }
\end{verse}

\poemtitle{\texthindi{परिवार}}
\begin{verse}
    {\texthindi{
        दो नम्बर मकान

        पहला मिलन

        मुहब्बत

        नाचूँ खुशियों से

        लगता है मैं घर आ गया हूँ

        मैं कहाँ फँस गया हूँ
    I    
        अपनी मिट्टी

        हमारा बचपन

        दिल करता है

        व्यापारी की इज़्ज़त

        माँ

        उर्मिल की कहानी

        प्रेम लूथरा श्रद्धांजली
        
        एफ़ जी टी मर गई

        जीवन

        अब नहीं तो कब

        एक फूल की कहानी

        पुनर्जन्म

        सुखी जीवन

        सरसराहट

        निंदा

        कल

        वक्त

        वक्त या पैसा

        खुशी अन्दर है

        बाँट के चीज़

        शब्द शक्ति

        खोखली हंसी

        ये वक्त जाने कहाँ चला गया\par
        }
}
\end{verse}

\poemtitle{\texthindi{ज़िन्दगी}}
\begin{verse} {
    \texthindi{
        दो नम्बर मकान

        पहला मिलन

        मुहब्बत

        नाचूँ खुशियों से

        लगता है मैं घर आ गया हूँ

        मैं कहाँ फँस गया हूँ

        अपनी मिट्टी

        हमारा बचपन

        दिल करता है

        व्यापारी की इज़्ज़त

        माँ

        उर्मिल की कहानी

        प्रेम लूथरा श्रद्धांजली

        एफ़ जी टी मर गई

        जीवन

        अब नहीं तो कब

        एक फूल की कहानी

        पुनर्जन्म

        सुखी जीवन

        सरसराहट

        निंदा

        कल

        वक्त

        वक्त या पैसा
        
        खुशी अन्दर है

        बाँट के चीज़

        शब्द शक्ति

        खोखली हंसी

        ये वक्त जाने कहाँ चला गया\par
        }
}
\end{verse}

\poemtitle{\texthindi{ज़िन्दगी}}
\begin{verse}{
    \texthindi{
    बात

    नये पंछी

    रौशनी की इज़्ज़त

    इन्सान की इज़्ज़त

    चाँद

    दोस्त

    नकली दोस्त

    अतीत के भूत

    सवेरा

    आग में सुलगना

    कच्चे घड़े

    बच्चों की मुस्कुराहट

    नया जीवन

    छे फ़ुट का फ़ासला

    साथी

    घर लुटवाना

    औरत

    काश हम मिले न होते

    काश हम बिछड़े न होते

    जीवन पथ

    दर्दे दिल

    गुस्सा

    एक हाथ की ताली

    मकड़ी जाल

    राजनेता

    पुनर्मिलन

    नज़रिया

    किस्मत के धनी

    रामायण सारांश में

    महाभारत सारांश में
    
    सुनामी

    इक रब के कई नाम

    घड़ी\par
    }
}
\end{verse}


\end{document}