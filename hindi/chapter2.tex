%Chapter 2
\chapter{\texthindi{आध्यात्मिक}}

%\begin{multicols}[2]
\poemtitle{\texthindi{भगवान}}
\begin{verse}[\versewidth]\texthindi{
बिन बुलाए मेहमान घर में नहीं आते\\
मैं कब से तैयार तुम ही नहीं बुलाते\\
दिल से बुलाओ छुपे भगवान चले आते\\
दिल से बुलाओ छुपे भगवान चले आते\\
\\
तेरे न्योते के इंतज़ार में आँखें बिछाये\\
बार बार सोचूँ कब नींद से जग जाये\\
गहरी नींद में तुम ने कितने जन्म गंवाये\\
ना धन से ना हीरे मोती से मुझे सजाये\\
बैठा सच्चे प्यार लगन की आस लगाये\\
\\
तेरी शोहरत ताकत पैसा मुझ से मिलता\\
मेरे हुक्म बिना पत्ता भी ना हिलता\\
ना कर लोभ गुमान क्रोध ईर्षा शिकायत\\
जितना कर्मों ने कमाया उतना ही मिलता\\
\\
प्यार से खिला रूखी सूखी खा लूँ\\
पैरों में आ जाये उठा गले लगा लूँ\\
है अंश मेरा क्यों खुद से जुदा बना लूँ\\
अपनी मैं छोड़ दे, तुझे अपने में मिला लूँ\\
अपनी मैं छोड़ दे, तुझे अपने में मिला लूँ
}
\end{verse}

\poemtitle{\texthindi{स्रोत}}
\begin{verse}[\versewidth]\texthindi{
माँग जहाँ से सब कुछ आये\\
माटी से क्यों आस लगाये\\
बीज अक्षर तेरे बीच रमा है\\
जो सारा संसार चलाये\\
माँग जहां से...\\
\\
कोई जन धन से महान कहाये\\
कोई तन से बलवान कहाये\\
सुंदर काया शव कहलाये\\
जिस तन से श्री राम सिधाये\\
राम ही धन है, राम ही शक्ति 2\\
राम ही बेड़ा पार लगाये\\
माँग जहाँ से सब कुछ आये\\
माटी से क्यों आस लगाये\\
माँग जहां से...\\
\\
जीवन एक हवा का झोंका\\
आज उठा है कल ना रहे गा\\
ये मेरा है वो मेरा है\\
वो तो रहे गा तू ना रहेगा\\
राम सदा थे राम सदा हैं\\
जुग जुग चाहे बीत ही जायें\\
माँग जहाँ से सब कुछ आये\\
माटी से क्यों आस लगाये\\
बीज अक्षर तेरे बीच रमा है\\
जो सारा संसार चलाये\\
माँग जहां से…
}
\end{verse}

\poemtitle{ \texthindi{ सर्वव्यापी } }
\begin{verse}[\versewidth]\texthindi{
देखूँ जिधर मैं तुझ को पाऊँ\\
जब जब तेरा ध्यान लगाऊँ\\
देख तेरी लीला महिमा मैं गाऊँ\\
जोड़ के हाथों को सीस झुकाऊँ\\
\\
देखूँ जिधर मैं तुझ को पाऊँ\\
जब जब तेरा ध्यान लगाऊँ\\
\\
कोई तोहे राम कहे कोई हरि पुकारे\\
वाहे गुरु अल्लाह यीशु नाम तिहारे\\
जिस नाम से भी तुझ को पुकारूँ\\
पल में तेरा दर्शन पाऊँ\\
देखूं जिधर…\\
\\
दरस तेरा उस को मिल जाता\\
जिस पे कृपा हो जाये तेरी दाता\\
चरणों में तुम रख लो मुझ को\\
दर दुनिया मैं छोड़ के आऊँ\\
देखूँ जिधर...\\
\\
पाप की गठरी ढो कर आया\\
लाया वही जो मैं ने कमाया\\
दे दो सहारा ओ मेरे मालिक\\
मैली चादर धो कर जाऊँ\\
\\
देखूं जिधर मैं तुझ को पाऊँ\\
जब जब तेरा ध्यान लगाऊँ\\
देख तेरी लीला को\\
महिमा मैं गाऊँ\\
जोड़ के हाथों को सीस झुकाऊँ \\
जोड़ के हाथों को सीस झुकाऊँ
}
\end{verse}

\poemtitle{ \texthindi{ मोक्ष } }

\begin{verse}[\versewidth]\texthindi{ 
पलक झपकते जीवन बीता\\
पंछी को उड़ जाना है\\
कौन है अपना कौन पराया\\
दो दिन का ये ठिकाना है\\
पलक झपकते…\\
\\
बचपन बीता आई जवानी\\
फूल ही फूल थे रुत मस्तानी\\
काल खड़ा देखे राहें तेरी\\
छोटी सी है ये ज़िंदगानी\\
राम से मन का मेल मिला ले x2\\
तन तो यहीं रह जाना है\\
पलक झपकते…\\
\\
जिन से मोह ममता कर बैठे\\
वो ना कभी तुम्हारे थे\\
जन्म जन्म का जिन से नाता\\
उन को ही क्यों बिसारे थे\\
भगवन तुझ से दूर नहीं है x2\\
एक ही बार बुलाना है\\
पलक झपकते…\\
\\
झूठी काया झूठी माया\\
मृग तृष्णा में क्यों भरमाया\\
राम स्वरूप सुनहरा पंछी\\
तन की आँख से देख ना पाया\\
बाहर माटी में तू ढूँढे x2\\
मन के बीच खज़ाना है\\
पलक झपकते…\\
\\
काम क्रोध मद मोह और माया\\
हथकड़ियाँ बन जायें गे\\
मात पिता सुत बीवी भाई बहना\\
साथ ना तेरे जायें गे\\
सच्चे कर्म और नाम राम का x2\\
साथ ही तेरे जाना है\\
पलक झपकते…\\
\\
गुरु और गुर की महिमा जानो\\
राम का रूप हैं तुम पहचानो\\
गुरु कृपा देखो दीप जला कर\\
राह दिखाये ओ अनजानो\\
तन से पूजो मन से ध्याओ x2\\
आत्म लीन हो जाना है\\
पलक झपकते…\\
\\
गुरु ने राम से मेल कराया\\
राम का निश दिन ध्यान करो\\
राम नाम की नाव में चढ़ कर\\
भव सागर को पार करो\\
जन्म मरण का खेल मिटा कर x2\\
मोक्ष तुझे अब पाना है\\
पलक झपकते जीवन बीता\\
पंछी को उड़ जाना है\\
कौन है अपना कौन पराया\\
दो दिन का ये ठिकाना है\\
पलक झपकते…
}
\end{verse}

\poemtitle{ \texthindi{दो दिन का मेहमान
}} \begin{verse}[\versewidth]\texthindi{ 
दो दिन का मेहमान रे तू\\
खुद को अब पहचान रे तू x2\\
कल तू आया कल है जाना\\
काहे करे अभिमान रे तू\\
दो दिन का मेहमान रे तू\\
\\
जिस को तू ने घर है समझा\\
ये तो एक सराये है\\
सदा नहीं कोई रहता इस में\\
इक आये इक जाये है\\
इक आये इक जाये है\\
राम शरण में तुझ को जाना\\
वहीं लगा ले रे ध्यान तू\\
दो दिन का मेहमान रे तू\\
\\
कौड़ी कौड़ी माया जोड़ी\\
लाख कमाए फिर भी थोड़ी\\
इस पैसे के लालच ने\\
सब रिश्तों की कड़ी है तोड़ी\\
सब रिश्तों की कड़ी है तोड़ी\\
हाथ तो खाली जाना है\\
झूठी बनाए क्यों शान रे तू\\
दो दिन का मेहमान रे तू\\
\\
जिस माटी ने तुझे बनाया\\
उस में ही मिल जाना है\\
जब तक तू है इस दुनिया में\\
कर्म भला कर जाना है\\
कर्म भला कर जाना है\\
दुखियों का दुःख बाँट ले बन्दे\\
जन्म का कर कल्याण रे तू\\
दो दिन का मेहमान रे तू\\
\\
दो दिन का मेहमान रे तू\\
खुद को अब पहचान रे तू\\
कल तू आया, कल है जाना\\
काहे करे अभिमान रे तू\\
दो दिन का मेहमान रे तू
}
\end{verse}

\poemtitle{ \texthindi{असली रूप
}}\begin{verse}[\versewidth]\texthindi{ 
मन मंदिर में घोर अंधेरा\\
जोत जले हो जाये सवेरा\\
मूँद के आँखें ध्यान लगा लो\\
तन तेरा है राम का डेरा\\
मन मंदिर…\\
\\
मंदिर मस्जिद गुरुद्वारा मन\\
दर दर भटके क्यों प्राणी जन\\
श्वास की धारा में बह कर देखो x2\\
कण कण में राम जी का बसेरा\\
मन मंदिर में…\\
\\
सुषमणा खोलो कुंडलिनी जागे\\
तन मन लागें कच्चे धागे x2\\
प्राणायाम से योग मिला लो x2\\
उस से असली जो रूप है तेरा\\
\\
मन मंदिर में घोर अंधेरा\\
जोत जले हो जाये सवेरा\\
मूँद के आँखें ध्यान लगा लो\\
तन तेरा है राम का डेरा\\
तन तेरा है राम का डेरा
}
\end{verse}

\poemtitle{ \texthindi{खुदी
}}\begin{verse}[\versewidth]\texthindi{
खुदी को मार दो खुद मरने से पहले\\
फिर देखो जीने का मज़ा क्या है\\
अंदर झाँक के देखो रब का रूप\\
इधर उधर भटकने में रखा क्या है\\
\\
यही रब मुझ में जो छुपा तुझ में\\
अलग नाम देने का फ़ायदा क्या है\\
दीन धर्म मज़हब इंसानों की हैं देन\\
असल को खिताबों से लेना क्या है\\
\\
जिधर भी देखो उस की ही सृष्टि\\
समझो उस बीच रमा क्या है\\
उस की सोच से तेरी बहुत छोटी\\
होता वही जो उस की रज़ा है\\
\\
जीवन डोर उसे थमा दे जो सारा संसार\\
चलाए\\
वही बनाए वही चलाए फिर मिटा के नया\\
बनाए\\
इंसान को खुदी की ज़रूरत क्या है\\
खुदी को मार दो खुद मरने से पहले\\
फिर देखो जीने का मज़ा क्या है\\
फिर देखो जीने का मज़ा क्या है
}
\end{verse}

\poemtitle{ \texthindi{नशुकरा
}}\begin{verse}[\versewidth]\texthindi{
गिनती खत्म हो जाती है\\
जब तेरी मेहरबानियाँ गिनता हूँ \\
आँख झुक जाती है शर्म से\\
जब और भी मिन्नतें करता हूँ \\
\\
भूला भटका नाशुकरा लोभी\\
फिर से भिखारी बन जाता हूँ \\
भूल खिलौने तोहफ़े सेहत\\
खुशी के नये साधन अपनाता हूँ \\
\\
जो मिला मुझे मेरी मेहनत थी\\
जो ना दिया गिला तुझ से\\
अपनों से ऊँचों को देख जलूँ\\
भूला सभी जो मिला तुझ से\\
\\
जब देखूँ अंधे को, कुछ पल\\
आँखों पे गरुर आ जाता\\
देखूँ शव, इक हल्का एहसास\\
खुद ज़िंदा होने का आ जाता\\
\\
सोचूँ दुःख बीमारियाँ मौत\\
रब ने बनाये दूजों के लिए\\
मैं तो सदा रहूँ गा ज़िंदा\\
हस्पताल शमशान दूजों के लिए\\
\\
फिर इक दिन कैन्सर या\\
दिल का दौरा पड़ जाता\\
इक बुलबुला हूँ सागर में\\
साफ़ दिखाई पड़ जाता\\
\\
तब सोचूँ कितना दिया तू ने\\
जिसे मैं ने नज़र अन्दाज़ किया\\
भाई बहन साथी घर छोड़े\\
रब सेहत को भुला बस\\
पैसे शान का नशा पिया\\
\\
आधी बन्द आँखें बेहोशी में\\
अंत ख्याल मुझे आता\\
पर किस्मत वाला तेरी मेहर से\\
जल्द ज्ञान पा जाता\\
क्या ?\\
गिनती खत्म हो जाती है\\
जब तेरी मेहरबानियाँ गिनता\\
आँख झुक जाती है शर्म से\\
जब और भी मिन्नतें करता
}
\end{verse}

\poemtitle{\texthindi{दो राहें
}}\begin{verse}[\versewidth]\texthindi{
दो राहें तेरे मन को मिलीं थी\\
एक को क्यों तू ने छोड़ दिया\\
एक को क्यों तू ने छोड़ दिया\\
\\
भूल गया तुझे जिस ने बनाया\\
भटका जहाँ उस की है माया\\
माया मृग छाया है भोले x2\\
उस के पीछे क्यों दौड़ लिया\\
एक को क्यों तू ने छोड़ दिया...\\
\\
सूरज जिस से रौशनी पाये\\
जो सारा संसार चलाये\\
उस दीपक से उस शक्ति से\\
मुख को क्यों तू ने मोड़ लिया\\
एक को क्यों तू ने छोड़ दिया ...\\
\\
दो राहें तेरे मन को मिलीं थी\\
एक को क्यों तू ने छोड़ दिया\\
एक को क्यों तू ने छोड़ दिया
}
\end{verse}

\poemtitle{\texthindi{ जीवन का मकसद 
}}\begin{verse}[\versewidth]\texthindi{
तर्ज़ — तू गंगा की मौज\\
है सदियों से ये सवाल चलता ही आया x2\\
काहे कुदरत ने इंसान जहाँ क्यों बनाया\\
\\
गीता में अर्जुन ने कृष्ण से पूछा\\
कृष्ण से पूछा\\
कल युग में शिक्षक ने गुरुओं से पूछा\\
गुरुओं से पूछा\\
रब ने बनाया तुझे प्रेम खज़ाना\\
प्रेम खज़ाना\\
गम अपना भूल तुझे जग को हँसाना\\
हर कोई अपना है ना कोई पराया\\
काहे कुदरत ने…\\
\\
किस्मत तू लिखे हाथों से अपनी\\
हाथों से अपनी\\
मिलता वो ही फल जो बोये तेरी करनी\\
बोये तेरी करनी\\
मन छोड़ बुद्धि से काम जो ले तू\\
काम जो ले तू\\
रब तेरे संग है अकेला नहीं तू\\
सफ़र ज़िंदगी में वो तेरा ही साया\\
काहे कुदरत ने इंसान जहां क्यों बनाया\\
है सदियों से ये सवाल चलता ही आया x2\\
काहे कुदरत ने इंसान जहाँ क्यों बनाया
}
\end{verse}
%\end{multicols}