\poemtitle{\texthindi{बाँके लाल का ढाबा\\
\\
बाँके लाल तानसेन के घराने से था। संगीत\\
उस की रग रग में बसा था।\\
उस ने बॉम्बे में गाने रिकॉर्डिंग स्टूडियो के\\
बाहर एक ढाबा खोला।\\
मशहूर बैकग्राउंड गाने वाले खाना इस ढाबे में\\
खाते थे। बाँके का हुक्म था की हर बात या\\
आर्डर गाना गा के होना चाहिये।\\
सब से पहले मन्ना डे आये\\
अरे बाँके भाई, गरम गरम नान खिला दे\\
बाँके—“गा के माँगो तो मिले गा”\\
तर्ज़ —ओ मेरी ज़ोहरा जबीं…\\
ओ मेरे बांके भय्या\\
दे दूँ गा दस रुपइया x2\\
दे दे गरम नान\\
भूख से निकली मेरी जान\\
मेरी जान\\
\\
प्रदीप\\
अरे बाँके आज मीठा खाने का मन कर रहा\\
है।\\
बाँके— “गा के माँगो प्रदीप जी”\\
तर्ज़ —देख तेरे संसार की हालत\\
दूध मलाई लड्डू पेड़े दे दो बाँके लाल x2\\
खाये हुए पूरा हुआ है साल\\
खाये हुए पूरा हुआ है साल\\
\\
तलत महमूद ने उबले अंडे माँगे थे। पर बाँके\\
किसी और को दे देता है। तलत शिकायत\\
करते हैं।\\
तर्ज़ —-जलते हैं जिस के लिये\\
लाया था मेरे लिये\\
अंडे तू ने उस को दिये\\
जल्दी से वापिस ला\\
वो तो थे मेरे लिये\\
लाया था मेरे लिये x2\\
\\
मुकेश शाकाहारी खाना माँगते हैं\\
तर्ज़ —मुझ को इस रात की तन्हाई में\\
अंडे मछली ना दो\\
मीट लहसन ना दो\\
और प्याज़ ना दो ओ\\
और प्याज़ ना दो ओ ओ\\
और प्याज़ ना दो\\
दाल भाजी दे दो\\
चाहे चावल दे दो\\
मगर प्याज़ ना दो ओ\\
मुझे प्याज़ ना दो ओ\\
ओ प्याज़ ना दो\\
आशा भोंसले को बहुत ज़्यादा भूख लगी है।\\
खूब सारा खाना माँगती है\\
तर्ज़ —दम मारो दम मिट जाये गम\\
गोभी शलगम, आलू हों दम\\
चिकन गरम, रोटी पूरी चावल नान\\
रोटी पूरी चावल नान\\
\\
हेमंत खाना खा लेते हैं। जेब में हाथ डालते हैं\\
पैसे देने के लिये। मगर जेब खाली है। बाँके\\
को बताते हैं क्या हुआ।\\
तर्ज़ —जाने वो कैसे लोग थे जिन के…\\
खाने के पैसे थे पॉकेट में\\
छोड़ा जब घर का द्वार\\
राह में चोर खड़े थे लूटा\\
उन्हों ने बारम्बार\\
खाने के पैसे थे पॉकेट में\\
छोड़ा जब घर का द्वार x2\\
\\
असित सेन आते हैं और कहते हैं “मुझे कुछ\\
नहीं खाना।\\
तर्ज़ —-एहसान तेरा हो गा मुझ पर\\
एहसान तेरा हो गा मुझ पर\\
ये लड्डू बर्फ़ी रहने दो\\
मुझे तोंद से नफ़रत हो गई है\\
मुझे डायट पर ही रहने दो\\
\\
बाग में जब हम सैर को निकलें\\
इन से नहीं लोग मुझ से पूछें\\
ये लड़का है या लड़की है\\
बाकी एक है या महीने दो\\
मुझे तोंद से नफ़रत हो गई है\\
मुझे डायट पर ही रहने दो\\
\\
मोहम्मद रफ़ी के आने पर बाँके बोला\\
तर्ज़—ए मेरे दिल कहीं और चल\\
सब्ज़ी मीट चावल चना दाल दही\\
आलू गोभी पराँठा ख़त्म हुआ\\
ढूँढ लो तुम कोई ढाबा नया x2\\
\\
रफ़ी साहिब बोलते हैं।\\
तर्ज़ — बाबुल की दुआएँ\\
दो रोटी मुझ को देता जा\\
थोड़ा प्याज़ अचार ही दे देना\\
मीठा जो अगर कुछ बच जाये\\
थोड़ा सा वो भी देना\\
(रोते हुए)\\
दो रोटी मुझ को देता जा\\
थोड़ा प्याज़ अचार ही दे देना\\
\\
ढाबा बंद करते हुए बाँके लाल गाता है\\
तर्ज़ — चल उड़ जा रे पंछी\\
उठ जाओ रे… गवियो ओ ओ ओ\\
उठ जाओ रे गवियो कि अब ये ढाबा बंद हुआ\\
जाओ अपने घर सारे अब ये ढाबा बंद हुआ\\
\\
फिर मिलें गे इसी जगह पर\\
खाना जहां तुम खाते हो\\
जो जी चाहे सब मिलता है\\
गाना जब तुम गाते हो\\
किसी चीज़ की कमी नहीं यहाँ\\
बाँके को तुम भाते हो\\
जल्दी वापिस आना यारो\\
रब से करता दुआ\\
उठ जाओ रे गवियो कि\\
अब ये ढाबा बंद हुआ\\
अब ये ढाबा बंद हुआ\\
\\
\\
\poemtitle{\texthindi{पोकर\\
सच को झूठ, झूठ को जीत बनाना है\\
दूजे की दौलत को अपने घर लाना है\\
तो भैया पोकर सीखो\\
\\
किस्मत से पत्ते भारी हैं तो जीत गये\\
खुद को बुद्धिमान समझना है\\
तो भैया पोकर सीखो\\
\\
जीवन में हर कोई नहीं जीतता\\
इक पत्ते की कमी बना घर गिरता\\
तो भैया पोकर सीखो\\
\\
अपनी जीत दूजे की हार है होती\\
उसे रुला कर खुश होना है\\
तो भैया पोकर सीखो\\
\\
हल्के हों या भारी पत्ते किस्मत से हैं\\
जो जीवन देता उस में खुश रहना है\\
तो भैया पोकर सीखो\\
\\
जीत में खुश और हार में रोना\\
दो पल का दौर गुज़र जाये गा\\
नया पत्ता नई आशा ले आए गा\\
तो भैया पोकर सीखो\\
रब की कृपा से खेलना मिला\\
पोकर या जीवन में\\
अगर जीतना है दोनों में\\
खुश रहने के साधन अपनाओ\\
किस्मत को दोषी ना ठहराओ\\
तो भैया पोकर सीखो\\
भैया पोकर सीखो\\
\\
\\
\\
\poemtitle{\texthindi{बीवी\\
\\
तर्ज़ —ज़िंदगी खाब है \\
\\
बीवी ऐसी चाहिए मेरी वो गुलाम हो\\
सुबह हो या शाम हो x 2\\
\\
पांच फुट तीन इंच हाइट हो\\
जींज़ जिस की टाइट हो\\
ज़ुल्फ़ काली घटा हो जैसे\\
रंग जिस का वाइट हो\\
चेहरे पे मुस्कान रहे सदा\\
मुझ से कभी ना फाइट हो\\
बीवी ऐसी चाहिए...\\
\\
बाहर का भी काम करे\\
सास ससुर की सेवा करे\\
घर चमकाए शीशे की माफ़िक\\
खाना नित नया बना करे\\
घर जब लौटूं गोल्फ खेल के\\
मालिश मेरी किया करे\\
बीवी ऐसी...\\
\\
एक दर्जन बच्चे दे दे\\
टीम करिकेट की घर में हो\\
घर में मेला शोर शराबा\\
कार्निवल भी घर में हो\\
सारी पलटन को ये सँभाले\\
हाथ मेरे बीयर ही हो\\
बीवी ऐसी चाहिए...\\
बीवी ऐसी चाहिए मेरी वो गुलाम हो\\
सुबह हो या शाम हो\\
\\
\poemtitle{\texthindi{पति\\
\\
ऐसा पति दे भगवन\\
सुबह उठ छुए मेरे चरण\\
बिस्किट और चाय वो लाये\\
प्यार से वो मुझ को जगाये\\
आँख खोलो मेरे सजन\\
ऐसा पति…\\
\\
मुझे जगाने से पहले\\
बच्चों को वो तयार करे\\
स्वामी नाश्ता भी तयार करे\\
खर्चा चलाने की खातिर\\
सोलह घंटे काम भी करे\\
ऐसा पति दे भगवन…\\
\\
जितने भी पैसे कमाये\\
मेरे हाथ में ही थमाये\\
स्वामी पांच वीज़ा कार्ड दिलाये\\
शॉपिंग मुझे ले जाये\\
सोना और हीरे दिलाये\\
ऐसा पति दे…\\
\\
साड़ी मेरी प्रेस कर के\\
नेल पॉलिश मुझे लगाये\\
स्वामी जूते भी चमकाये\\
पार्टियों में सज के मैं जाऊँ\\
रोल्स रॉयस में बिठाये\\
ऐसा पति दे…\\
\\
आमिर खान जैसी स्माइल हो\\
देव आनंद जैसे बाल हों\\
स्वामी उस के गाल लाल लाल हों\\
अमिताभ जैसी ऊँचाई x 2\\
सलमन खान जैसी चाल हो\\
ऐसा पति दे भगवन\\
\\
ऐसा पति दे भगवन\\
सुबह उठ छु ए मेरे चरण\\
बिस्किट और चाय वो लाये\\
प्यार से वो मुझ को जगाये\\
आँख खोलो मेरे सजन\\
ऐसा पति दे भगवन\\
\\
\\
\poemtitle{\texthindi{पति पत्नी की नोक झोंक\\
\\
पर्दा उठता है। सोफ़े पे सफ़ेद कोट पहने\\
डॉक्टर कुर्सी पे बैठा वाल स्ट्रीट जर्नल पढ़\\
रहा है। वाल स्ट्रीट शब्द लोगों की तरफ़ हैं।\\
मेज़ पे स्टेथोस्कोप है। साथ लक्ष्मी देवी की\\
फ़ोटो है।\\
पति—हे लक्ष्मी माँ, आज तो कृपा कर दे।\\
कितने दिन से स्टॉक नीचे ही नीचे जा रहे हैं।\\
इंटेल, बोइंग, सिस्को सब सो गये हैं। देवी उन\\
को जगाओ ना। प्लीज़! 2 दिन हो गये हैं व्रत\\
रखे हुए। कृपा कर दे मैया।\\
ओम ओम ओम…\\
\\
पत्नी स्टेज पे आती है। हाथ में एक महँगी\\
लगती साड़ी है। उस को कभी इधर से कभी\\
उधर से देखती है।\\
कहती है—-आ गये?\\
इतनी लेट?\\
फिर लग गये स्टॉक और बौंड के चक्कर\\
में!\\
(पति अखबार को बंद कर के उठता है।)\\
पति —-अच्छा बंद कर दी अखबार। बोलो\\
क्या हाल है जनाब का?\\
पत्नी—तुम्हें याद है आज कौन सा दिन है?\\
पति सोच के बोलता है —आज, आज सैटरडे\\
है। हफ़्ते में छे दिन काम कर कर के थक गया हूँ।\\
पत्नी— नहीं, आज हमारी पच्चीसवीं साल\\
गिरह है।\\
(पति घबरा के उठता है और कहता है)\\
अरे मैं तो भूल गया। माफ़ कर दो। मैं तुम्हारे\\
लिये कोई गिफ्ट भी नहीं लाया। आज तो\\
लेट हो गया हूँ। कल सुबह सब से पहला का\\
काम…\\
\\
पत्नी—शादी से पहले रोज़ कभी गुलाब के \\
फूल, कभी सिनेमा, कभी हैसियत से भी महँगे\\
तोहफ़े लाते थे। शादी हुई ये सब भूल गये।\\
मुझे मालूम था।\\
इस लिये मैं खुद ही तुम्हारी तरफ़ से इक\\
साड़ी ले आई हूँ।\\
देखो अच्छी है ना। इस में हीरे मोती जड़े हुए\\
हैं।\\
(पति हाथ लगाता है।)\\
पति—ये तो बहुत सुंदर है। तुम पर बहुत सजे गी। बहुत महँगी तो नहीं?\\
पत्नी —एक डॉक्टर की बीवी हूँ। सस्ती पहनू \\
गी तो लोग सोचें गे तुम अच्छे डॉक्टर नहीं हो,\\
यही सोच कर बस पच्चीस हज़ार की ही लायी हूँ।\\
पति—पच्चीस हज़ार रुपए?\\
पत्नी—आप भी ना बुद्धु ही रहे। रहते \\
अमरीका में और खर्चा रुपयों में। सिर्फ़ \\
पचीस हज़ार डॉलर की है! मिसेज़ वर्मा तो \\
पचास हज़ार से कम साड़ी देखती भी नहीं। मैं \\
ने तो सिर्फ़ पच्चीस हज़ार ही खर्चे हैं !\\
(पति सिर पकड़ के कुर्सी पे बैठ जाता है।\\
पत्नी उस की तरफ़ बड़ती है। पति खड़ा होता\\
है। गुस्से में गाता है।) \\
गाने की धुन—तुम रूठी रहो मैं मनाता रहूँ\\
मज़ा जीने का और भी आता है\\
\\
पति—तुम खरचती रहो मैं कमाता रहूँ\\
पैसा जैसे मुफ़त में आता है x 2\\
पत्नी—तुम स्टॉक खरीदो लाखों बॉण्ड खरीदो x2 \\
मेरा साड़ी का लाना नहीं भाता है\\
इक साड़ी का लाना नहीं भाता है\\
तुम स्टॉक खरीदो ओ ओ ओ\\
\\
पति —मैनेज्ड केयर ने बहुत सताया\\
एच एम ओ ने ठेंगा दिखाया X2\\
नींद वकीलों ने है चुराई\\
मेडिकेयर ने भी साथ छुड़ाया\\
सूरज उगने से पहले चाँद छुपने के बाद\\
तेरा घर वाला घर वापिस आता है\\
तुम खरचती रहो ओ ओ ओ\\
\\
पत्नी—भूल सदा तुम को है होती\\
तुम करो काम सारा दिन मैं तो सोती x2\\
घर का चलाना कोई आसान काम नहीं\\
माँजूँ मैं बर्तन तेरे कपड़े मैं धोती\\
तेरी रोटी पकाऊँ तेरे बच्चे भी पालूँ x2\\
सारा दिन तो यूँ ही गुज़र जाता है\\
तुम स्टॉक खरीदो ओ ओ ओ\\
\\
पति—मुझे तो खबर ना थी ऐसा तेरा हाल है\\
पत्नी—दिन ढले मैं जानू क्यों तू बेहाल है\\
दोनों मिल के—\\
हाथ पकड़ लो मेरा सजनवा\\
थाम मुझे तू रखूँ तेरा ख्याल मैं\\
चाहे रो के गुज़ारें चाहे हंस के बिताएँ\\
समा जीवन का यूँ ही गुज़र जाता है X 2\\
पत्नी—तुम कमाते रहो मैं खरचती रहूँ\\
मज़ा जीने का और भी आता है\\
पति—तुम खरचती रहो\\
पत्नी—तुम स्टाक खरीदो\\
तुम तुम तुम…\\
स्टेज से उँगलियाँ दिखाते चले जाते हैं\\
\\
\\
\\
\\
\poemtitle{\texthindi{पैसे के दो रूप\\
हाय पैसा हाय पैसा\\
रोग लगा रे हाय कैसा\\
मात पिता को रुला दिया\\
देश भी अपना भुला दिया\\
हाय पैसा हाय पैसा\\
\\
वाह पैसा वाह पैसा\\
देखा नहीं कोई तुझ जैसा\\
ऊँचे महल बनाये तू\\
हीरे मोती दिलाये तू\\
वाह पैसा वाह पैसा\\
\\
आँख को अंधा कर दे ये\\
कान को बहरा कर दे ये\\
अक्ल पे लग जायें ताले\\
काम कराये ये काले\\
हाय पैसा हाय पैसा\\
\\
झुक झुक लोग सलाम करें\\
नेता गण मेरा पानी भरें\\
जग में नाम कराये ये\\
बिगड़े काम बनाये ये\\
वाह पैसा वाह पैसा\\
\\
\\
भाई बहनों में वैर करा दे\\
अपनों को ये गैर बना दे\\
तोड़े यारों की यारी\\
माया का पलड़ा भारी\\
हाय पैसा हाय पैसा\\
\\
\\
\\
\\
\\
\\
सब से महँगी गाड़ी लें\\
हीरों से जड़ी साड़ी लें\\
जो जी आये खरीदें हम\\
यारों को भी खरीदें हम\\
वाह पैसा वाह पैसा\\
\\
\\
\\
\\
\\
\\
बच्चों का बचपन ना देखा\\
देखी पैसे की रेखा\\
सोलह घंटे काम किया\\
धन खातिर घर त्याग दिया\\
हाय पैसा हाय पैसा\\
\\
\\
\\
\\
\\
\\
दुनियाँ की ये सैर करा दे\\
चाँद और तारे ज़मीं पे ला दे\\
पूरे कर दे सब अरमान\\
पैसा खुशियों की है खान\\
वाह पैसा वाह पैसा\\
\\
\\
\\
\\
\\
\\
माया तो आनी जानी है\\
साथ नहीं ये जानी है\\
राम नाम धन क्यों भूला\\
असली रूप को क्यों भूला\\
हाय पैसा हाय पैसा\\
\\
\\
\\
\\
\\
\\
राम मिले ना पैसे से\\
यार मिले ना पैसे से\\
प्यार नहीं जहाँ धन का राज\\
इस सच को पहचान ले आज x2\\
हाय पैसा हाय पैसा\\
हाय पैसा हाय पैसा\\
हाय पैसा हाय पैसा\\
\\
पैसे वाला हाथ मसलता है\\
कोई जवाब नहीं सूझता।\\
स्वामी जी के पाँव पड़ जाता है।\\
कहता है —आप धन्य हैं। आप ने सुख और\\
शांति का पथ दिखाया। उस के लिये कोटी\\
कोटी धन्यवाद। भगवान का लाख लाख\\
धन्यवाद।\\
सिया पति राम चन्द्र की जय\\
दोनों बोलते हैं\\
सिया पति राम चन्द्र की जय\\
( ये एक स्किट बन सकता है।\\
एक किरदार कमीज़ या कोट पर नकली\\
रुपयों के या किसी भी किस्म के नोट चिपका\\
लेता है। ऊपर वाली जेब में रंगीन रूमाल\\
निकला हुआ है, महँगा चश्मा पहना है जो\\
एक अंतरे के बाद उतार देता है। चेहरे और\\
चाल में अकड़ है।\\
दूसरा किरदार भगवे कपड़ों में है। खड़ांवें\\
पहनी हैं। चेहरे पे मुस्कान है।)\\
\\
\\
\\
\poemtitle{\texthindi{शराबी\\
\\
शराबी मुख व्हिस्की के लिए ही दिया\\
शराबी मुख व्हिस्की के लिए ही दिया\\
\\
इंडिया का जॉनी वॉकर बॉम्बे में बस्ता\\
इन का जॉनी ढूँढे स्कॉटलैंड का रस्ता\\
शाम ढले घूँट लगाया दोस्तों ने साथ दिया\\
शराबी मुख व्हिस्की के लिये ही दिया\\
\\
सिंगल माल्ट पर पैग तो डबल हो\\
इन की मस्ती हो बीवी को ट्रबल हो\\
जूते पड़े डांट भी खाया\\
फिर भी मुँह बोतल में दिया\\
शराबी मुख व्हिस्की के लिए ही दिया\\
\\
पहले पेग ने शेर बनाया\\
दूजे ने बनाया बन्दर\\
सूअर जैसी सूरत बन गयी\\
तीसरा गया जो अन्दर\\
कुंभकर्ण से रिश्ता बनाया\\
खर्राटों ने सोने ना दिया\\
शराबी मुख व्हिस्की के लिए ही दिया\\
शराबी मुख व्हिस्की के लिए ही दिया\\
\\
\\
\poemtitle{\texthindi{आधुनिक दिवाली\\
\\
न फूल न पूजा की थाली\\
न कोई लेता राम का नाम\\
फ़िल्मी गाने भजन बने\\
चरणामृत शराब का जाम\\
\\
न मंदिर की घंटी ना शंख की\\
सुनी आवाज़\\
टकराते छलकते ग्लास बने\\
गीतों के साज़\\
\\
चढ़ावा मूर्ति पर नहीं\\
तीन पत्ति खेल पे था\\
न माँगी रब से शांति\\
मन ऊँची ट्रेल पे था\\
\\
घर में अगरबत्ती ना काली धूप जली\\
उड़ते बादल दिखे जब सिगरेट जली\\
\\
न सुंदरकांड पढ़ें\\
न हनुमान चालीसा बोली\\
चुटकुले सुनाए कोने में\\
हंसती इक नटखट टोली\\
\\
पैर लड़खड़ाए होंठ थर्राए\\
मदिरा अपना रूप दिखाए\\
श्रद्धा से साष्टांग नहीं\\
मदिरा फ़र्श पे उसे लिटाए\\
\\
शोरगुल में बोलें सभी\\
सुने न दूजे की बात\\
हैपी दीवाली सब कहें\\
पूछे ना दिल के हालात\\
\\
सीधे चल के आए थे\\
दीवाली की खुशी मनाने\\
कुछ होश में कुछ मदहोश\\
चले अपनी कार चलाने\\
\\
दिये गायब बनी चीन में\\
लड़ियाँ घर चमकायें\\
बचपन की दिवाली\\
बच्चों को कैसे समझायें\\
\\
वाह मेरे भाइयो बहनो\\
क्या हाल किया है\\
दीवाली का निकाल दीवाला\\
उसे बेहाल किया है\\
\\
बाहरी चमक धमक बंद कर\\
अंदर के दीप जला\\
आधुनिक दीवाली छोड़\\
बचपन की दीवाली मना\\
आधुनिक दीवाली छोड़\\
बचपन की दीवाली मना\\
\\
\\
\\
\poemtitle{\texthindi{जॉनी का सर दर्द\\
\\
पर्दा खुलता है। स्टेज पे एक सोफ़ा है। उस\\
की बाईं ओर छोटे मेज़ पे टेलीफ़ोन है। \\
दूसरी ओर मेज़ पे लैंप है। बीवी जिस का \\
नाम मेरी है, लैंप को कपड़े से साफ़ कर रही \\
है। उस का पति जॉनी दाईं तरफ़ से स्टेज पे\\
आता है। माथे को पकड़े हुए जैसे सर में बहुत दर्द है।\\
मेरी को कहता है।\\
तर्ज़—चंपी तेल मालिश, सर जो तेरा चकराये\\
जॉनी—मेरी ओ मेरी\\
मेरी—(झुंझला के ) अब क्या है ?\\
जॉनी—सर दर्द से फटा जाये\\
और दिल मेरा घबराये\\
इस से पहले दम तोड़ूँ मैं\\
डॉक्टर जल्दी बुलाये\\
डॉक्टर जल्दी बुलाये\\
\\
(मेरी डॉक्टर को फ़ोन करती है)\\
तर्ज़—मन डोले मेरा तन डोले\\
डॉक्टर आना जल्दी आना\\
मेरा घर वाला है बीमार रे ए\\
कौन बचाये तेरे बिना x2\\
डॉक्टर आता है। सफ़ेद कोट, गले में\\
स्टेथोस्कोप है\\
तर्ज़—दुनियाँ बनाने वाले क्या तेरे मन\\
कौन मरीज़ है बोलो\\
किस को मुझे है बचाना\\
दुखियों की पीड़ा मिटाना मुझे\\
दुखियों की पीड़ा मिटाना x2\\
जॉनी कहता है—\\
मेरा सर दर्द बहुत ही ज़्यादा बड़ गया है।\\
लगता है स्ट्रोक हो गयी है।\\
(गिरते हुए सोफ़े का सहारा ले कर फ़र्श पे\\
बैठ जाता है और कहता है)\\
हम छोड़ चले हैं महफ़िल को\\
याद आये…\\
(फ़र्श पे आँखें साँस बंद किए गिर जाता है)\\
मेरी कहती है—हाय हाय!\\
गाना तो ख़त्म कर जाते!\\
(जॉनी उठ जाता और गाता है)\\
कभी तो मत रोना\\
( आँखें साँस बंद कर के फिर लेट जाता है)\\
(डॉक्टर गाता है)\\
तर्ज़—सावन का महीना पवन करे सोर\\
डॉक्टर—जॉनी का नंबर आया\\
रब ने खींची ढोर\\
मेरी—ढोर नहीं डोर\\
डॉक्टर—रब ने खींची ढोर\\
मेरी—अरे बाबा ढोर नहीं डोर, डोर!\\
डॉक्टर—हाँ, रब ने खींची डोर\\
मेरी तू हो जा मेरी\\
चल नाचें जैसे मोर\\
(डॉक्टर और मेरी एक दूसरे का हाथ पकड़ते\\
हैं। इक दूसरे को देख कर मुसकाते हैं। दोनों\\
स्टेज से बाहर की तरफ़ गाते हुए जाते हैं\\
डॉक्टर—मेरी तू हो जा मेरी\\
चल नाचें जैसे मोर\\
मेरी—डॉक्टर तू हो जा मेरा चल नाचें जैसे मोर\\
(लाइट्स मद्धम हो जाती हैं)\\
दोनों —मेरी तू हो जा मेरी\\
डॉक्टर तू हो जा मेरा\\
चल नाचें जैसे मोर\\
\\
\poemtitle{\texthindi{जीवन का चक्कर\\
तर्ज़—(ये देश है वीर जवानों का अलबेलों का\\
मस्तानों का)\\
बड़े शौक से शादी करते हैं\\
और जल्दी से बच्चे करते हैं\\
इन बच्चों का यारो, होय!\\
इन बच्चों का यारो क्या कहना\\
ये मम्मी पापा का गहना\\
\\
ओ, ओ, ओ ओय ...\\
बड़े जोश से हनी मून गए\\
पर वहाँ तो एक्सीडेंट हुए\\
गए दो थे तीन, होए!\\
गए दो थे तिन वापिस आये\\
दोनों के कठिन अब दिन आये\\
\\
नौ मास मम्मी को तड़पाया\\
पापा भी लगे अब घबराया\\
घर छोटा पैसे, होए!\\
घर छोटा पैसे कम हों गे\\
अब बाल सफ़ेद और कम हों गे\\
\\
सारी सारी रात ये रोते हैं\\
और सारा दिन ये सोते हैं\\
सोया मुखड़ा रब का, होए!\\
सोया मुखड़ा रब का रूप लगे\\
और जगें तो जिन्न और भूत लगे\\
\\
मुश्किल दिन रात की भूल गए\\
जब आँख खुली तो स्कूल गए\\
दो शब्द इंग्लिश के , होए!\\
दो शब्द इंग्लिश के क्या जाने\\
मम्मी पापा को मूरख माने\\
\\
दो पल में टीनेजर बन गए\\
हम प्यादे ये मेजर बन गए\\
अब अक्ल का ठेका, होए!\\
अब अक्ल का ठेका इन्हें मिला\\
हर बात पे अब ये करें गिला\\
\\
\\
ओ, ओ, ओ ओय ... \\
पंछी की तरह घर में आयें\\
मौसम बदले ये उड़ जायें\\
दो दिन के ही, होए!\\
दो दिन के ही मेहमान हैं ये\\
सच मानो अपनी जान हैं ये\\
\\
चाहे जो भी करें जहाँ पर भी रहें\\
ये टुकड़े हमारे, होए!\\
ये टुकड़े हमारे दिल के हैं\\
सदा रहें हमारे दिल में हैं\\
(पलक झपकते ये टुकड़े ना जाने कब बड़े हो\\
गए और उन्हों ने क्या किया?)\\
\\
ओ, ओ, ओ ओय ... \\
बड़े शौक से शादी करते हैं\\
और जल्दी से बच्चे करते हैं\\
उन बच्चों का यारो क्या कहना\\
वो मम्मी पापा का गहना\\
जीवन का चक्कर फिर से शुरू हो गया और\\
चलता गया...\\
\\
\poemtitle{\texthindi{कोविड के दिन\\
तर्ज़—जायें तो जायें कहां\\
एक आदमी रॉकिंग कुर्सी पे बैठा है। हाथों पे\\
ग्लव्स, मुँह पे मास्क लगाया हुआ है। मेज़\\
पर किताबें, टी वी का रिमोट, टेलीफोन और पानी का गिलास है।\\
एक ओर छोटे मेज़ पे लाइसोल वाइप्स का\\
डिब्बा, ग्लव्स का डब्बा, एलकोहल की\\
शीशी है।\\
घर के दरवाज़े के बाहर एक बड़ा साइन है जेल\\
\\
\poemtitle{\texthindi{जाऊँ तो जाऊँ कहाँ\\
(गाने की केरियोकी शुरू होती है।\\
आदमी गाता है)\\
\\
जाऊँ तो जाऊँ कहाँ\\
जाऊँ तो जाऊँ कहाँ\\
देखूँ जिधर कोरोना वहाँ\\
मुँह को ढकूँ धोऊँ हाथ\\
डरता है दिल छुपा है कहाँ\\
जाऊँ तो जाऊँ कहाँ\\
\\
कोई खाँसे आगे\\
कोई खाँसे आगे कोई छींके पीछे\\
बचना है मुश्किल\\
छुपे वायरस से\\
छे फुट का रखूँ फ़ासला\\
जाऊँ तो जाऊँ कहाँ\\
\\
घर जेल बन गया\\
घर जेल बन गया खत्म है आज़ादी\\
बंद हैं दुकाने हुई बर्बादी\\
नेतागण् लड़ रहे\\
जनता तबाह\\
जाऊँ तो जाऊँ कहाँ\\
\\
जाऊँ तो जाऊँ कहाँ\\
जाऊँ तो जाऊँ कहाँ\\
देखूँ जिधर कोरोना वहाँ\\
मुँह को ढकूँ धोऊँ हाथ\\
डरता है दिल छुपा है कहाँ\\
जाऊँ तो जाऊँ कहाँ\\
\\
\\
\poemtitle{\texthindi{महँगे प्याज़\\
थका हारा दफ़्तट्रर से जब घर मैं आया\\
बीवी को बेहोश रसोई में पाया\\
सुन मेरी बोली आँख उस ने खोली\\
फ़रश पर पड़ा खाली थैला दिखाया\\
“मैं लेने गयी प्याज़ जब सब्ज़ी मंडी\\
सुन उस की कीमत हो गयी मैं ठंडी\\
रुपए सौ आधे किलो के उस ने माँगे\\
लगी काँपने मेरी नाज़ुक सी टाँगे\\
मुंह पे पसीना रुकी साँस मेरी\\
छाया अंधेरा ज्यों रात घनेरी\\
गिरती हुई मैं घर लौट आई\\
कमज़ोर कुर्सी पे बैठ ना पाई\\
अब तुम को मालूम बेहोशी का राज़\\
सोने से महँगे हैं नाचीज़ से प्याज़\\
सोने से महँगे हैं नाचीज़ से प्याज़\\
नाचीज़ से प्याज़ नाचीज़ से प्याज़।”\\
\\
\\
\\
\\
\poemtitle{\texthindi{फ़ोन\\
बचाओ इक चोर मेरे घर घुस आया\\
बचाओ बचाओ\\
दुनियाँ भुलाई नींद गंवाई इसी से दिल\\
लगाया\\
रिश्ते नाते भूल गये अब अपना भी लगे पराया\\
चोर नन्हें चमकते फ़ोन का साया\\
  \\
यही रकीब यही सौतन इसी ने चैन गंवाया\\
हंसते खेलते परिवार को इसी ने रुलाया\\
कैंसर का इलाज है इस ने सब को हराया\\
चोर नन्हें चमकते फ़ोन का साया\\
\\
नज़र इसी पे हरदम हाथों से सहलाया\\
बात करे ना कोई, साधन इसे बनाया\\
उँगलियाँ ज़बान बनी होंठ सिल गये\\
एक अजब नया युग आया\\
\\
इसी संग सोना इसी संग दिन बिताया\\
इक पल आँखों से ओझल दिल घबराया\\
निकालें घर से कैसे मालिक बन आया\\
\\
गायब पुराने अस्तर इस ने सब को खाया\\
कैमरा जी पी एस कैलकुलेटर बना पराया\\
छोटे से राजा ने हम सब को गुलाम बनाया\\
\\
इसी के तौर तरीकों को सब ने अपनाया\\
प्रेमी घुट के लिपटे नज़रें इस से मिलायें\\
झुकी आँखें नीचे मुगल राज लौट आया\\
चोर मेरे घर घुस आया\\
चोर चमकते फ़ोन का साया\\
बचाओ बचाओ\\
\\
\\
\poemtitle{\texthindi{स्टॉक्स\\
\\
तर्ज़—दुनियाँ बनाने वाले क्या तेरे मन में\\
समाई\\
स्टॉक्स खरीदने वाले क्या तेरे मन में समाया\\
सारा ही पैसा लुटाया तू ने\\
सारा ही पैसा लुटाया x 2\\
\\
जिस पैसे ने दिए पाँव में छाले\\
पेट में अल्सर जान के लाले\\
दिल की नाड़ी जिस पैसे ने बंद की x2\\
उसी पैसे को बेदर्दी उछाले\\
मेहनत की पूँजी को\\
जुए में क्यों रे लगाया\\
सारा ही पैसा लुटाया\\
स्टॉक्स खरीदने वाले...\\
\\
\\
\\
जिस दिन स्टॉक्स के स्पेलिंग तू जाने\\
उसी दिन खुद को पीटर लिंच माने\\
किस्मत से दो चार पैसे बन जाएँ कहीं से x2\\
सब पार्टियों में गाये खुशियों के गाने\\
ये धन जिसे अपना समझे\\
जल्द ही हो गा ये पराया\\
सारा ही पैसा लुटाया\\
स्टॉक्स खरीदने...\\
\\
\\
\\
सारी सारी रात तोहे नींद नहीं आये\\
डूबे हुए स्टॉकों की याद सताए\\
गुज़रे हुए वापस नहीं आते x2\\
उन के लिए क्यों तू आँसू बहाए\\
म्युनिसिपल बौंड खरीदो\\
यही है डौली ने सिखाया\\
सारा ही पैसा लुटाया\\
स्टॉक्स खरीदने वाले\\
क्या तेरे मन में समाया\\
सारा ही पैसा लुटाया\\
तू ने सारा ही पैसा लुटाया…\\
\\
\poemtitle{\texthindi{अमरीका के कुछ रंग\\
(जहां भी रहो फूल और काँटे हों ही। अमेरिका में बहुत फूल मिले। उस के साथ कुछ काँटों का वर्णन इस गीत मैं है।) \\
\\
पत्नी—\\
कहाँ ले चले हो बता दो मुसाफ़िर\\
सितारों से आगे ये कैसा जहाँ है\\
पति —\\
सितारों के सपने हो सके ना अपने\\
पहुँचे अमरीका नसीबा बुरा है\\
\\
यहाँ मुंडू माली और नौकर तुम्हीं हो\\
धोबी तुम्हीं और शौफ़र तुम्हीं हो\\
दो पल बैठना मुमकिन नहीं है\\
अब क्यों रोये क्यों पछताये\\
मिलता वही जो ऊपर लिखा है\\
\\
यहाँ महल ऊँचे मगर दिल हैं नीचे\\
यहाँ पैसा पहले मगर प्यार पीछे\\
बिज़नेस जो दे जाये सब से बड़ा है\\
यारी खरीदो प्यार खरीदो\\
सब रिश्तों से डॉलर बड़ा है\\
\\
सुनते थे पेड़ों पे डॉलर हैं खिलते\\
जेबें हैं भारी दिल नहीं मिलते\\
सोने के पिंजरे में जान फँसी है\\
हीरे खरीदें स्टॉक्स खरीदें\\
चौबीस घंटे ये ही कथा है\\
\\
\\
दिन रात देश में गोलियां बरसती\\
रंग भेद भाव की आँधी है चलती\\
स्कूल जेल बन गये बच्चे हैं कैदी\\
पड़ोसी ना जाने सभी अनजाने\\
घर मेरा मकबरा बन गया है\\
\\
सितारों के सपने हो सके ना अपने\\
पहुँचे अमरीका नसीब बुरा है\\
पहुँचे अमरीका नसीब बुरा है\\
\\
सत्तर्वाँ जन्मदिन\\
\\
कल का छोकरा बुड्ढा हुआ है\\
अपने को समझे जवान\\
भागे दो गज़ निकले जान\\
\\
मुँह पे ज़्यादा सिर पे कम हैं\\
झड़ती ज़ुल्फ़ों का मौसम है\\
कान सुने ना आँख ना देखे\\
जो अंग छुओ वो ही नरम है\\
आती जाती को ये छेड़ें\\
दिल इन का है जवान\\
भागे दो गज़ निकले जान\\
\\
देख तेरे चेहरे की लकीरें\\
हाथ शर्म से डूबा जाये\\
सर तो काला कर बैठे हो\\
बाल छाती के कै से छुपायें\\
फ़ेस लिफ्ट करा लो\\
छाती मुंडा लो\\
कोई ना सके पहचान\\
भागे दो गज़ निकले जान\\
\\
पिकी मदन को जो जन जाने\\
भाग्यशाली वो खुद को माने\\
इन के लब पे सदा तराने\\
दिल में क्या है कोई ना जाने\\
गम को छुपा के जग को हसायें\\
जगत में इन का नाम \\
भागे दो गज़ निकले जान\\
\\
\\
कल का छोकरा बुड्ढा हुआ है\\
अपने को समझे जवान\\
भागे दो गज़ निकले जान\\
भागे दो गज़ निकले जान\\
(तर्ज़ —देख तेरे संसार की हालत\\
एक मित्र के लिये लिखी थी। नाम बदल के\\
किसी के लिए गाई जा सकती है।)\\
\\
\\
\\
\\
\\
\\
\\
\\
\\
\\
\chapter{बुढ़ापा बीमारी मौत\\
\\
\\
\\
\\
\\
\\
\\
\\
\\
\poemtitle{\texthindi{बड़ती उम्र\\
\\
ज़िंदा हो शुक्र करो लाखों को नसीब ना होता \\
हर रात को सुबह का सूरज नसीब नहीं होता\\
\\
खुश रहो हर हाल में, वक्त और बुरा हो\\
सकता है\\
रब के तोहफ़े गिनो, कमियाँ हर कोई गिन\\
सकता है \\
\\
पुस्तक पढ़ो सुंदर दुनियाँ की सैर करो \\
रब ने जिस काम का बीज दिया \\
उसे जगा कर दुनियाँ में बाँटा करो \\
बहाने हर कोई बना सकता है \\
\\
मैं ने पूछा ज्योतिषी से दिन कितने बाकी बचे\\
हैं मेरे\\
जितनी लोगों की सेवा करो रब दुगने कर दे\\
गा तेरे\\
\\
ज़िंदा रहो खुल के जब तक साँसों का \\
अमन गमन है \\
किसी के काम आ सको तो वीरान बंजर भी \\
चमन है \\
\\
जीते रहे तो उम्र का नम्बर बड़ता जाए गा\\
ज़रूरी नहीं जवानी का नम्बर घटता जाए गा\\
\\
मौत तो सब को आनी है\\
बेमौत मरना ज़रूरी नहीं\\
बड़े होना जीवन की रीत है\\
बूड़े होना ज़रूरी नहीं\\
बड़े होना जीवन की रीत है\\
बूड़े होना ज़रूरी नहीं\\
\\
\\
\poemtitle{\texthindi{मेरी उम्र\\
\\
बच्चे मेरे बड़े हो गये\\
दोहते दोहतियाँ जवाँ हो गयीं\\
पर मैं वही जवान गबरू\\
दिमागी उम्र बढ़ना भूल गयी\\
\\
\\
\\
\\
\\
शीशे में परछाई ना पहचानू\\
ये बाबा कहाँ से आया\\
बिन कार्ड डिस्काउंट मिले\\
कैसा ज़माना है आया\\
\\
लोग कहें अंकल मुझे\\
शक उन की अक्ल पे होता\\
बहन भाई बूड़े बस मैं जवान\\
ऐसा गुमान दिल में होता\\
\\
अभी भी फेरूँ हाथ मैं सर पे\\
भले चट खाली मैदान वहाँ\\
बंजर कैसे ये बन गया\\
अभी हरा भरा था खेत वहाँ\\
\\
स्याही बचाने वास्ते छोटे अक्षर\\
अखबार वाले लिखने लगे\\
दर्द गले का बचाने खातिर\\
लोग धीमी आवाज़ बोलने लगे\\
\\
टी वी पे बच्चे खबर सुनायें मुझे\\
समझायें, अजब ज़माना आया\\
मेरा डॉक्टर लगे स्कूल का बच्चा\\
ज़रूर घोर कल युग है आया\\
\\
\\
झुर्रियाँ जोड़ों के दर्द फूली साँसें\\
ये औरों की वसीयत है\\
मेरे जिस्म में कैसे घुस आयीं\\
गलत पते की मिली नसीहत है\\
\\
 भाई बहन झेलें बीमारियां बूढ़ों की\\
इक हल्का ख्याल दिल में आता\\
कल मेरा भी नंबर लगे गा\\
मैं ना रहूँ गा यकीन नहीं आता\\
\\
\\
अब साथी संगी झड़ने लगे\\
शमशान के चक्कर लगने लगे\\
खुद को देखूँ चादर में लिपटा\\
सुनू लोग बातें मेरी करने लगे\\
\\
ये उम्र ना जाने कहाँ चली गयी\\
अब सोचूँ तरकीबें लंबी करने की\\
दवाइयों की लिस्ट बढ़ती गयी\\
सुनू बात ऑपरेशन और मरने की\\
अनगिनत दिन मिले थे मुझे\\
अधूरी तमन्नाएँ पूरी कर लूँ गा\\
कई ख्याल सपने थे मन में\\
जल्दी क्या है कल कर लूँ गा\\
\\
शाम ढली पर कल नहीं आया\\
इंसाफ़ कहाँ का है बतलाओ\\
मेरा सूरज डूब रहा\\
नई फसल को जा समझाओ\\
\\
\\
\\
भूल कड़वी यादें, मीठी बातों को\\
याद करो\\
खुल के जियो, दूजों का दामन\\
खुशी से भरो\\
चाहे लगे उम्र लंबी पर है छोटी\\
छोड़ शिकायत गिले पछतावा\\
इस पल को रौशन करो\\
इस पल को रौशन करो\\
\\
\\
\poemtitle{\texthindi{खोई जवानी\\
\\
खोई जवानी ढूँढे जब वो\\
हाथ छुड़ा कर चली गयी\\
आँख उठा ना देखा उस को\\
चुपके घर से निकल गयी\\
\\
बहता तेज़ झरना है जवानी\\
वापिस ना आए गिरता पानी\\
\\
जिस्म की ताकत खो गई\\
झुरियों की माला लिपट गई\\
\\
नशा किया कसरत ना की\\
खाना कई जन्मों का खाया\\
काम से फुर्सत मिली नहीं\\
पार्टियों में जा वक्त गँवाया\\
\\
अब बिन तेल की सब्ज़ी माँगे\\
मक्खन लदे पराँठे छोड़ दिये\\
खरीदने से पहले लेबल देखे\\
चीनी से रिश्ते नाते तोड़ दिये\\
\\
अंडे के पीले फेंक दिये\\
मिठाइयों से परहेज़ किये\\
पीते थे वाइन और बीयर\\
अब ईसबगोल हर शाम पिये\\
\\
देर आये दरुस्त आये\\
वरज़िश अब शुरू करें गे\\
लगन लगा लक्ष्य बना\\
बीमारियाँ अब दूर करें गे\\
\\
बिन मेहनत के ना माँ का\\
दूध मिले ना साँसें चलती\\
आलस छोड़ हिम्मत कर\\
जीवन का सुधार करें\\
\\
खोई जवानी को हिम्मत से\\
वापिस लाने की मेहनत करें\\
खोई जवानी को हिम्मत से\\
वापिस लाने की मेहनत करें\\
\\
\\
\\
\\
\\
\\
\\
\poemtitle{\texthindi{दोस्तों की नई तस्वीरें\\
\\
पुराने दोस्तों की नई तस्वीरें देखता हूँ\\
खंडहरों में रंगीन महल ढूँढता हूँ\\
\\
वो अधखिले सपने, सितारों की चाहत\\
टूटे ख्वाबों के बेजान टुकड़े देखता हूँ\\
\\
झुर्रियों पीछे जवानी की निशानियाँ ढूँढता\\
चेहरों पे ठोकरों की परेशानियाँ देखता हूँ\\
ना जाने कब चमक पे बादल छा गये\\
घने बाल उड़े जो बचे रंग बदला गये\\
\\
जीवन के थपेड़े सीधी कमर झुका गये\\
बुढ़ापे की सारी निशानियाँ देखता हूँ\\
\\
फीकी मुस्कानों में हल्का दम है\\
भविष्य के सपने ख्वाहिशें कम हैं\\
\\
कोई मिलने ना आता आँखें नम हैं\\
कमज़ोर थके हारे पिंजर देखता हूँ\\
बीमारियों ने घर बनाया किसी को\\
कुछ चेहरे काल ने बेवक्त खा लिये\\
धुंधली यादों में खोये साथी देखता हूँ\\
\\
पुराने दोस्तों की नई तस्वीरें देखता\\
टूटे ख्वाबों के बेजान टुकड़े देखता हूँ\\
\\
\\
\\
\\
\poemtitle{\texthindi{बुढ़ापे के रंग\\
\\
बच्चों ने बुढ़ापा नहीं देखा\\
ये तो औरों की बीमारी है\\
मौत या उस का साया दूर\\
मस्त जवानी की खुमारी है\\
\\
\\
\\
\\
\\
या रब इन्हें कुछ दिन बुढ़ापा\\
और फिर से जवानी लौटा दे\\
फिर ये शायद समझ पायें\\
आते दिनों की कठिन राहें\\
\\
वो क्या जाने बुढ़ापे के गम\\
वो चार दीवारी की बंद घुटन\\
सन्नाटा सुनसान अकेलापन\\
\\
\\
\\
मौत का चारों ओर मण्डराना\\
साथी का अचानक छोड़ जाना\\
इक काठी से संभलना मुश्किल\\
सफ़र ज़िंदगी का अकेले निभाना\\
\\
\\
ऊँचा सुनना अंधेपन का डर\\
रात में उठना फिर नींद ना आना\\
आ भी गयी तो ना उठने का डर\\
भूली यादों भरा गीला सिरहाना\\
\\
जन्म दिन पे मिले तोहफ़ा नया\\
जोड़ जुड़ें कैंसर ने धर लिया\\
फेफड़ों की बीमारी साँस फूलना\\
चढ़ती जवानी ने रुख मोड़ लिया\\
\\
\\
कच्ची दीवारों पीछे सुनू बातें\\
क्यों ना जाते उस पार जहाँ से\\
बेवजह बैठे हैं ना कोई मकसद\\
पार क्यों नहीं करते ये सरहद\\
\\
बने रुकावट जीवन की राह में\\
पत्थर जैसे नदी के परवाह में\\
आखरी साँस इन की ये शायद\\
सुनता हूँ उन की दबी ज़बान में\\
\\
\\
भूले इन्हीं हाथों ने चलना सिखाया\\
इसी आवाज़ ने बोलना समझाया\\
इन्हीं कन्धों पे बैठ दुनिया को देखा\\
पार की बचपन से जवानी की रेखा\\
\\
फिर सोचता मैं ने यही किया था\\
जब थी उन्हें मेरी ज़रूरत\\
मैं ने भी छोड़ा मुँह फेरा था\\
कल कर लूँ गा वो सदा रहेंगे\\
ना खत लिखा बात भी ना की\\
अपनी रंगीन राहों में खोया था\\
\\
जो बोया मिलता वही जीवन के खेत में\\
खिलते नहीं रंगीन फूल बंजर सूखी रेत में\\
\\
लाचारी बच्चों की हम समझ पायें\\
संवारे अपनी क्यारी और गायें\\
उन को मिली एक ही फुलवारी\\
जिस के खिलने की खुशी मनायें\\
\\
\\
\\
\\
पलक झपकते ये बचपन से हुए जवान\\
ना जाने कब मैं जवान से बूढ़ा हो गया\\
कमर झुकी चेहरे पे झुर्रियाँ मेरी पहचान\\
रंगे जवानी राह में जाने कहाँ खो गया\\
\\
मिलती एक बारी ज़िंदगी के खेल में\\
इक तरफ़ा रास्ता है जीवन के मेल में\\
गुज़रा वक्त ना लौट के आए कभी\\
माँ बाप का अपना, बच्चों का\\
ख्याल रखो मेल में\\
हर पल को चमक सुनहरी दो\\
इस सुंदर छोटे से खेल में\\
\\
जितना दिया है रब ने\\
वो झोली से भी ज़्यादा है\\
भरपूर बचपन जीवन बाद\\
बुढ़ापे का गम आधा है\\
भरपूर बचपन जीवन बाद\\
बुढ़ापे का गम आधा है\\
\poemtitle{\texthindi{आल्ज़ाइमर्ज़\\
\\
(बुज़ुर्ग एक जवान बच्चे से बात कर रहा है।\\
अक्षर बदल कर ये ख्याल आदमी या औरत\\
के लिए लिखा जा सकता है।)\\
\\
याद है रात भर सर्द पट्टियाँ तेरे सर पे लगाया\\
करता था\\
लिटा गोद में लगा सीने बाल सहलाया करता\\
था\\
तेरी बीमारी मुझे मेरी उम्र तुझे\\
रब से भीख माँगा करता था\\
\\
तेरी साँसों की आवाज़ सुनने\\
अपनी साँस थामा करता था\\
\\
चोट लगती तुझे दर्द होता मुझे\\
दिल रातों में रोता था\\
तेरी खुशी के खज़ाने बना मोती\\
दिल में हार पिरोता था\\
\\
छुपा दर्द, जीवन तेरा रंगीन\\
बनाने के साधन सोचता था\\
दुनियाँ की दर्दनाक खबरें\\
तेरे कानों से रोकता था\\
\\
हटा नोकीले काँटे राहों से\\
फूल बिछाया करता था\\
ना दुःख दे ना मुश्किल तुझे\\
मैं दुनियाँ से झगड़ता था\\
\\
\\
जब किसी ने दिल तोड़ा तेरा\\
दिल मेरा भी मुरझाता था\\
देख तेरा दिलदार साथी\\
मैं मन ही मन मुसकाता था\\
\\
फुलवारी खिलते देख तेरी\\
दिल मेरा भी खिल जाता था\\
सब को हँसता फलता देख\\
अंदरूनी सुकून मिल जाता था\\
\\
\\
तुम कम मिलते, व्यस्त थे\\
झलक पाने को तरसता था\\
हर पल यादों का ले सहारा\\
दिन रात गुज़ारा करता था\\
\\
ना जाने क्यों, कब, कैसे\\
यादों पे बादल छाने लगे\\
कहाँ था मैं कौन था मैं\\
संगी साथी दूर जाने लगे\\
\\
\\
गुज़री पुरानी यादें ज़िंदा हैं\\
जैसे कल की ही बात है वो\\
जो कल बीता या आज हुआ\\
भूला ज्यूँ काली रात है वो\\
\\
जिस हाल में हूँ खुश हूँ बहुत\\
शायद तुम्हें मालूम नहीं\\
मेरा सुंदर अतीत मेरी दुनिया है\\
शायद तुम्हें मालूम नहीं\\
\\
\\
ना जानू तुझे ना पहचानू तुम्हें\\
पर यादें तेरी ज़िंदा हैं दिल में\\
इस ख्याल से रहना खुश बेटी\\
बस तू ही तू रहती है दिल में\\
इस ख्याल से रहना खुश बेटी\\
बस तू ही तू रहती है दिल में\\
\\
\\
\\
\\
\poemtitle{\texthindi{हैं से थे बन गये\\
\\
न जाने कब हैं से थे बन गये\\
धूप पे काले बादल छा गये\\
कल कलियों से फूल खिले\\
कल आते सूख के मुरझा गये\\
\\
कल थे हम हसीन जवान\\
अपना नया ज़माना था\\
रंगीन सावन को ज़ालिम\\
पतझड़ खाक बना गये\\
\\
बर्फ़ करे फ़ख्र चमक का\\
नासमझ माना जाये गा\\
कड़ी गर्म धूप बेरहमी से\\
पिघला के पानी बना गये\\
\\
समंदर में लहरें उठें बलखाती\\
छूना चाहें आसमान\\
दो पल में गिरें बेनाम\\
वजूद पानी में समा जाये\\
\\
ये नाम सूरत धन ओहदा\\
चन्द दिनों के हैं मेहमान\\
समय बहुत बलवान है\\
वक्त की चादर से ढके गये\\
\\
जियो इस पल में गले लगाओ\\
आज को गम चिंता से न जलाओ\\
तेरे जैसे लाखों करोड़ों\\
इसी राह से आ कर गुज़र गये\\
\\
न जाने कब हैं से थे बन गये\\
धूप पे काले बादल छा गये\\
कल कलियों से फूल खिले थे\\
कल आते सूख के मुरझा गये\\
कल आते सूख के मुरझा गये\\
\\
\\
\\
\poemtitle{\texthindi{जीवन का खेल\\
\\
बदल जाते खिलाड़ी पर\\
खेल चलता जाता है\\
एक खिलाड़ी चला गया\\
दूजा उस जगह आ जाता है\\
\\
खिलाड़ी टीम की शक्लें नाम\\
कायदे कानून बदले\\
लाख कोशिशें करने पर भी\\
कोई बच के नहीं जा पाता है\\
\\
\\
जीतने खातिर झगड़ा लड़ा\\
कुछ जीता कुछ चुराया\\
आगे बड़ने के लालच में परेशान\\
खेल का मज़ा ले ना पाया\\
\\
झूठ बोला तरकीबें बनायीं\\
स्वार्थी बन दूजों को गिराया\\
ये खेल है, खत्म हो गा भूला\\
बस अपनी धुन में ही समाया\\
\\
\\
हर खिलाड़ी सोचे वो है पहला \\
खेल में, उसी वास्ते खेल रचाया\\
आया मैदान में खेला कूदा\\
बड़ा हुआ चिल्लाया शोर मचाया\\
बादल गरजा बिजली चमकी\\
फिर आँसुओं का झरना बरसाया\\
\\
\\
\\
\\
\\
चुपके हार गिरा पस्त हुआ\\
लोगों ने कंधों पे उठा के हटाया\\
देख जगह खाली अगले खिलाड़ी ने\\
उछलते कूदते कदम बढ़ाया\\
ना शुक्रिया उन का जो पहले आये\\
उस के लिए मैदान बनाया\\
\\
\\
\\
\\
\\
धीरे धीरे थका कदम उठाना मुश्किल\\
खेल के अंत का ध्यान आया \\
बहुत देर हो चुकी थी संभलते\\
नये खिलाड़ी ने बेरहमी से आन गिराया\\
\\
बदल जाते खिलाड़ी पर\\
खेल चलता जाता है\\
एक खिलाड़ी चला गया\\
दूजा उस की जगह आ जाता है\\
\\
\\
\poemtitle{\texthindi{बहुत देर\\
\\
आये मेरी मैयत पे गिरा आँसू\\
दो फूल चढ़ा के चले गये\\
पुल बांधे तारीफ़ों के\\
अनगिनत खामियाँ भुला गये\\
\\
\\
\\
\\
\\
ना बरसों मिले बात भी ना की\\
अब मुँह दिखलाने आ गये\\
काश वक्त साथ गुज़ारा होता\\
ज़िंदा रहते तोहफ़ा भेजा होता\\
गिरे को सहारा, पोंछता आँसू\\
रातों को रौशन किया होता\\
आँख तरस गई फिर नम हुई\\
राह देखते नैन थक सूख गये\\
बहुत देर कर दी आने में\\
आये तुम जब हम ही ना रहे\\
तेरा नाम लेते जहाँ से गुज़र गये\\
तेरा नाम लेते जहाँ से गुज़र गये\\
\\
\\
\poemtitle{\texthindi{मोम के पुतले\\
\\
जाने कब ये मोम पिघल जाये\\
ढाँचा भी आग में जल जाये\\
दो दिन के हैं मेहमान सभी\\
मुट्ठी राख पानी में मिल जाये\\
\\
कल साथ था मेरे अब साया नहीं\\
दरवाज़ा खुला पर वो आया नहीं\\
दो पल बाहर गया लौट आये गा\\
यादें छोड़ गया फिर आया नहीं\\
\\
दो पल उस के गीत गाये\\
तस्वीर पे इक दो फूल चढ़े \\
याद रखने का वादा किया\\
फिर अपनी राह पे चल पड़े\\
भूले नाम जो दो दिन पहले मरे\\
किसे है वक्त बेवा का पूछे हाल\\
अपने मकड़ी जाल में सभी घिरे\\
\\
जिस की मैयत पे आये हो\\
वो कल किसी और पे आया था\\
हम आये हैं आज अलविदा कहने\\
कल कोई हमें विदा करने आये\\
\\
\\
\\
जाने कब ये मोम पिघल जाये\\
ढाँचा भी आग में जल जाये\\
दो दिन के हैं मेहमान सभी\\
मुट्ठी राख पानी में मिल जाये\\
\\
\\
\\
\\
\poemtitle{\texthindi{जन्म मरण\\
\\
जब बर्फ़ की तितलियां पिघलती हैं\\
पेड़ से पत्ता गिरता है\\
घर छोड़ सितारा चमके कुछ पल\\
मिट्टी में जा मिलता है\\
\\
तेज़ शोरगुल पानी नदी का\\
चुप सागर में मिलता है\\
साथी संगी मित्र नाता तोड़\\
दुनिया से बिछड़ता है\\
\\
\\
जिन संग खेले बड़े हुए हँसते गाते\\
अपने सामने मरता है\\
सोच के अपना अंत\\
खामोश ख्याल दिल में उभरता है\\
\\
शमशान में लेटा जिस्म है दूजा\\
चेहरा अपना दिखता है\\
क्या यही है ज़िंदगी जिस खातिर\\
इंसान दिन रात भटकता है?\\
\\
\\
बैठ पेड़ की छाँव में, काश\\
उन संग समय बिताया होता\\
मात पिता भाई बहनों का संग\\
किस्मत वालों को मिलता है\\
\\
अधखिले सपने यादों का गुबार\\
फूट के बाहर निकलता है\\
कुछ दिन संभल जाता इंसान\\
फिर झूठी राह पे फिसलता है\\
\\
\\
काम क्रोध मद अहंकार का\\
फिर धीरे धीरे कै दी बनता है\\
अक्लमंद पहचान अस्ल को\\
जल्द सीधी राह पकड़ता है\\
\\
ना आने का वक्त है हाथों में\\
ना ही यहाँ से जाने का\\
कुदरत का जन्म मरण का खेल\\
अपने असूलों से चलता है\\
कुदरत का जन्म मरण का खेल\\
अपने असूलों से चलता है\\
\\
\\
\poemtitle{\texthindi{मौत\\
\\
मौत को इतने करीब देखा है\\
अब जानी पहचानी लगती है\\
दिखता उस का रूप आईने में\\
इक सच्ची कहानी लगती है\\
\\
जब जान लिया पहचान लिया\\
अब उस की चिंता कम लगती\\
जीवन किताब का पहला अक्षर\\
आखरी पन्ने की निशानी लगती\\
\\
शरीर नाशवान सुना पढ़ा और देखा\\
साफ़ दिखा अपनों को जलते देखा\\
हो गा यही हाल कल मेरा खेल तमाम\\
चादर ढका घर वालों को रोते देखा\\
\\
\\
मुश्किलें परेशानियां खत्म हो जायें गी\\
जीवन मौत की रेखा मिट जाये गी\\
जीवन सूर्य उगता ले उमंगें लाखों\\
शाम सूरज की तपस ढल जाये गी\\
\\
शोहरत पैसा शरीर गुमान\\
चंद दिन मेहमान घर में आये\\
पलक झपकते रुखसत हों गे\\
पंछी के पर निकले उड़ जाये\\
\\
\\
गस्सा गरुर चिंता लोभ ना रखो\\
झुलसें जिस्म संग इक पल में\\
जोड़ा खज़ाना यहीं रह जाये\\
\\
करो तमन्नायें पूरी बाँटो प्यार\\
ढलता जिस्म सदा रह ना पाये\\
ये जीवन मिला है कुछ पल\\
हँसो हसाओ खुशी से जी भर\\
बीते कल की भूलें भुला दो\\
अगला पल आये ना आये\\
साँसों का आना जाना नियामत\\
ना जाने ये ताँता कब रुक जाये\\
ना जाने ये ताँता कब रुक जाये\\
\\
\\
\poemtitle{\texthindi{यादों के खंडरात\\
\\
घर भरा हुआ लोगों से पर मेरे लिए खाली है\\
हर शै पे नाम लिखा हर पौधे का तू माली है\\
\\
बच्चों में छवि तेरी दोहते दोहती में रूप तेरा\\
उन की सोच हरकतों के पीछे देखूँ हाथ तेरा\\
\\
नानी बीवी मौम, दोस्त नाम से याद करें गे\\
बारी बारी चले गए मैं और तेरी यादें रहें गे\\
देखूँ जिधर मैं नज़र तू आये\\
बोलूँ पर जवाब ना आये\\
मौत के सामने बस नहीं\\
आँखों ने आँसू छलकाये\\
\\
\\
दूजों संग होते बहुत सुना था\\
इतना मुश्किल हो गा ये\\
ऐसा कभी ना सोचा था\\
\\
अब मालूम हुआ तो ज़िंदगी\\
वापिस लाना चाहता हूँ\\
लाखों ख्याल आते मन में\\
तेरी बातें सुनना चाहता हूँ\\
तेरी ख्वाहिशें इच्छायें\\
पूरी करना चाहता हूँ\\
काश समझ पाता मुश्किलें तेरी\\
सोच पिछले कुछ सालों की\\
वो तू नहीं तेरी बीमारी थी\\
ले ज़बान शिकायत आँसुओं की\\
\\
कौन रोके टोके गा मुझे\\
ऐसे सवालों में डूबा रहता हूँ\\
अकेले बैठ जीवन भर\\
गम में खोया उलझा रहता हूँ\\
जब भी आहट होती सोचता हूँ\\
लौट के तू आ जाए गी\\
अकेले अब चलते उम्र भर की\\
सज़ा भुगतनी पाए गी\\
\\
कम्बख़्त वक्त बहुत ही ज़ालिम\\
पंछी की तरह उड़ जाता\\
लाख करो दुआएँ घड़ी की\\
सुई का काँटा लौट ना आता\\
\\
\\
\\
यही है पल जिस में है जीवन\\
इस को अपनों पे लुटा डालो\\
जो तुम से प्यार करें\\
उन खातिर खुदी मिटा डालो\\
\\
देखो मुझे मेरे गम मजबूरी\\
छलकते आँसुओं को\\
उसकी मौत मेरे दुख से सीखना है\\
तो उस के लिये ज़िंदा रहना सीखो\\
वो चले गए तो सर पकड़ रोओ गे\\
तनहा यादों की माला पिरोओ गे\\
\\
यादों के खंडरात का कैदी रात मेरी काली है\\
घर भरा हुआ है लोगों से मेरे लिए ये खाली है\\
\\
\\
\\
\\
\\
\\
\poemtitle{\texthindi{दुआ\\
\\
\\
(अपने एक प्रिय मित्र को खुले कास्केट में\\
और परिवार को रोते देखने के बाद ये कविता\\
लिखी थी)\\
\\
साँस रुकी धड़कन दिल की\\
फिर भी नज़ारा देख सकता हूँ\\
तेरा रोना देखा ना जाता\\
ना आँसू पोंछने आ सकता हूँ\\
\\
देखूँ बच्चों को रोते बिलखते\\
हाथ पकड़ते फूल चढ़ाते\\
गर्मी बारिश में बादल बन छाता\\
सदा नहीं मैं रह सकता हूँ\\
\\
मुझे तो मुक्ति मिल गई\\
जो देखना करना बहुत किया\\
साथ तुम्हारा मिला किस्मत से\\
जीवन का रस बहुत पिया\\
\\
\\
ना सोचा ना माँगा फिर भी\\
किस्मत ने झोली भर दी\\
ज़मीन से छू लिया आसमान\\
हम को रब ने बहुत दिया\\
\\
जितनी साँसें मिलीं तुम्हें\\
हँसते खेलते पूरी करना तुम\\
खुल के जीना मेरे हमदम\\
मेरे हिस्से का भी जीना तुम\\
\\
\\
दुगनी खुशियाँ बाँटना हमेशा\\
दुगने पोंछना औरों के आँसू\\
मीठी यादों का ले के सहारा\\
दूजों की काठी बन जाना तुम\\
\\
इक रोज़ तेरा भी दिन आए गा\\
तू भी यहीं पर लेटी हो गी\\
जिन को जन्मा बड़ा किया\\
सर थामे तेरी बेटी हों गी\\
\\
\\
करता हूँ दुआ वो दिन आये\\
इक लम्बे अरसे के बाद\\
करूँ गा तेरा इंतज़ार\\
मुलाकात अगले जन्म हो गी\\
\\
साँस रुकी धड़कन दिल की\\
फिर भी नज़ारा देख सकता हूँ\\
तेरा रोना देखा ना जाता\\
ना आँसू पोंछने आ सकता हूँ\\
         \\
\poemtitle{\texthindi{जीवन और मौत\\
\\
हवा का झोंका, बुलबुला या उड़ते बादल\\
का साया\\
इक सपना है जीवन, चमकता सुनहरी मृग\\
छाया\\
भूले नशीली जवानी में, शरीर नाशवान जल\\
जाए गा\\
जिस खातिर की चोरी धोखे, दिल अपनों का\\
दुखाया\\
जन्म से अंत नामी जिल्द से\\
बंद किताब में समाया\\
जीवन गाथा खुद लिक्खी या\\
किस्मत ने लिखवाया\\
\\
अगले साँस की खबर नहीं\\
कई जन्मों की करी तैयारी\\
काम क्रोध मद लोभ माया में\\
भूले भटके जन्म बिताया\\
\\
\\
जब थे ज़िंदा लोग ना मिलते\\
पीछे पीठ बातें कोसा करते\\
आँख बंद हुई बड़ी कतार लगा\\
पुल तारीफ़ों के बांधा करते\\
\\
वो भी शरम से मिलने आते\\
जो सालों घर ना आये थे\\
दो पल मिलना मुश्किल था\\
अब दिन भर मुँह दिखाई करते\\
\\
\\
मरने बाद इतने फल फूल मिले\\
जो ज़िंदा होते ना पाए\\
जितनी बड़ाई मरने बाद हुई\\
ज़िंदा रहते सुन ना पाए\\
\\
ज़िंदा थे लोग गलतियों खामियों\\
पर तवज्जो देते बातें करते\\
जाने बाद अच्छाइयों की चर्चा करते\\
\\
\\
\\
जब था मैं बीमार अपाहिज \\
कोई सालों ना मिलने आता\\
अब नहीं हूँ मिलने वालों का\\
ताँता खत्म ना होने पाता\\
\\
दूजों से अच्छा लगने खातिर\\
कोई पक्की दोस्ती जताता\\
अब बना हमदर्द जो पहले\\
आँख मिलाने से कतराता\\
\\
\\
बीवी, शौहर या अपने छोड़\\
सब ढोंगी नज़र आते\\
कोई चुपके खून के आँसू छुपाता\\
कोई नकली पानी छलकाते\\
\\
किसी की ज़िंदगी इक पल में\\
आसमान से मिट्टी में मिल जाती\\
कोई ज़िंदा रहते उँगली ना पकड़ता\\
अब कंधों का सहारे दिलाते\\
\\
\\
इक पल ज़िंदा फिर साँस धड़कन रूकती\\
लाख यत्न करे इंसान, जीत मौत की होती\\
शव देख कर दो पल अपने अंत का सोचें\\
जल्द भूलते जीवन का सच\\
नकली कहानी शुरू फिर होती\\
\\
हवा का झोंका, बुलबुला या\\
उड़ते बादल का साया\\
इक सपना है जीवन\\
चमकता सुनहरी मृग छाया\\
चमकता सुनहरी मृग छाया\\
\\
\poemtitle{\texthindi{जीवन का अन्त\\
\\
बहुत अजब है जीवन\\
चन्द बरसों में मेरा नाम ना हो गा\\
हमारे ख्वाब किस्से ऊँचाइयाँ\\
गहराइयाँ कोई ना जाने गा \\
घर बनाया सजाया चमकाया\\
मिट्टी का ढेर बन जाए गा\\
\\
हमारी कश्मकश मेहनत काम\\
किसी को याद नहीं\\
कौन था मैं क्या क्यों करता था\\
पूछने का वक्त ना हो गा\\
\\
अपनी दुनियाँ के चर्चे\\
गर पूछें तो लोग हैरान हों गे\\
किस दुनियाँ से आए हो\\
खिलाड़ी बदले खेल वही हों गे\\
\\
गुज़रे जीवनों के क़िस्से\\
किसी को याद ना हों गे\\
\\
बहुत अजब है जीवन\\
चन्द बरसों में नाम ना हो गा\\
गनीमत समझो दो पीढ़ियों तक\\
शायद मेरा नाम तो हो गा \\
शायद मेरा नाम तो हो गा\\
\\
\\
\poemtitle{\texthindi{समय की धूल\\
समय की धूल सदियों को छुपा जाती \\
समुद्र की लहर रेतीले महल मिटा जाती \\
\\
चलती फिरती बोलती बड़ती ज़िंदगी\\
साँस रुकने से पल में खत्म हो जाती\\
\\
तस्वीरें यादें नाम काम कुछ पल ज़िंदा\\
अगली पीढ़ी जीने में मग्न, भुला जाती \\
\\
शिकवे गिले दो पल के मेहमान\\
शिकायत करने सुनने वाले ना रहे \\
इक उम्र गुज़रने पे समझ आती\\
\\
सूरत पैसा ओहदा मृग छाया\\
सदा इन्हें कोई बांध ना पाया\\
सावन के फल फूल रंगीन पत्ते \\
पतझड़ की तूफ़ानी तेज़ हवा \\
उड़ा बिखरा मिट्टी में मिला जाती\\
\\
 \\
राजे महाराजे वक्त की नदी में बह जाते    \\
 ऊँची जात के अभिमानी राख बन जाते \\
दस नामो के बादशाह बेनाम बन जाते \\
समय की भूख मिट्टी को मिट्टी से मिलाती \\
\\
हैसियत खाक में मिल जाती\\
एक दो पीढ़ी तक रहती निशानी \\
फिर समय की गोद में सो जाती\\
\\
समय की धूल सदियों को छुपा जाती \\
समंदर की लहर रेतीले महल मिटा जाती\\
\\