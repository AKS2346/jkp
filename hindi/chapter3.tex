% Page 55 of 518
%Chapter 3
\chapter{ \texthindi{परिवार} }
%\begin{multicols}[2]

\poemtitle{ \texthindi{दो नंबर मकान } }
\begin{verse}[\versewidth]\texthindi{
आओ सब मिल गायें गाथा\\
दो नंबर मकान की\\
सन पचास में बोली लगा कर\\
लगा दी बाज़ी जान की\\
मात पिता की जै बोलो\\
मात पिता की जै\\
मात पिता की जै बोलो\\
मात पिता की जै\\
लाला जी को दस एकड़\\
ज़मीं मिली ईनाम में\\
कुंदन बेटा बने गा डॉक्टर\\
करम ज़मीं के काम में\\
कुदरत के रंग किस्मत पलटी\\
करम मिले श्री राम में\\
छोड़ डॉक्टरी के सपने\\
कुंदन खेती के काम में\\
ना शिकवा ना गिला था कोई\\
चेहरे पर मुस्कान थी\\
आओ सब मिल गायें गाथा\\
दो नंबर मकान की\\
मात पिता की जै बोलो\\
मात पिता की जै\\
\\
सरगोधे से चली ये जोड़ी\\
पहुँची खानेवाले में\\
पिता बाईस के माता जी थी\\
अभी सोलहवें साल में\\
पिता जी ने मारा छक्का\\
सब से पहली बॉल में\\
क्रिकेट टीम के कैप्टेन सूरज\\
पहुँचे पहले साल में\\
रेलवेज़ का अफ़सर हो गा\\
शान हिंदुस्तान की\\
आओ सब मिल गायें गाथा\\
दो नंबर मकान की\\
मात पिता की जै बोलो\\
मात पिता की जै\\

सुदेश मोहिन्दर लाँघ ना पाये\\
बचपन की दीवार को\\
प्रेम कांता कंचन विरिंदर\\
शोभा दें संसार को\\
कृशन गिंदी शोकी ने कर दिया\\
पूरा लंबी कतार को\\
मात पिता ने सींची क्यारी\\
दे कर अपने प्यार को\\
जीवन धारा बहती जाये\\
खबर ना पाकिस्तान की\\
आओ सब मिल गायें गाथा\\
दो नंबर मकान की\\
मात पिता की जै बोलो\\
मात पिता की जै\\
\\
खानेवाले में धूप की गर्मी\\
नफ़रत की थी आग जली\\
दूर दर्शी हिंदू जनता\\
सदियों के घर से भाग चली\\
पिता जी नारंग ठक्कर भाई\\
छुपा प्यारी हर दिल की कली\\
दूर सबाथु ठंडी छाँव में\\
परिवार की नाव चली\\
मई सैंतालिस जान बचा कर\\
ढूँढी जगह विश्राम की\\
आओ सब मिल गायें गाथा\\
दो नंबर मकान की\\
मात पिता की जै बोलो\\
मात पिता की जै\\
\\
जहाँ भी देखो लाशें थीं\\
हर तरफ़ खून की होली थी\\
हा हा कार था आग और धुआँ\\
मार पीट की टोली थी\\
सदियों से जो भाई बहन थे\\
अब नफ़रत की बोली थी\\
पानीपत घर छीन लिया\\
जो जगह थी मुसलमान की\\
\\
आओ सब मिल गायें गाथा\\
दो नंबर मकान की\\
मात पिता की जै बोलो\\
मात पिता की जै\\
\\
जिधर भी देखो टैंट लगे थे\\
हर कोई घर की आस में\\
रेल लाइन के पार था प्यारा\\
इक घर खुले आकाश में\\
दो नंबर पर नज़र पड़ी\\
पिता जी की तलाश में\\
बीवी बच्चे यहीं पलें\\
हरियाली और प्रकाश में\\
मां ने ना की पैसा ना पल्ले\\
बोली दी मकान की\\
आओ सब मिल गायें गाथा\\
दो नंबर मकान की\\
मात पिता की जै बोलो\\
मात पिता की जै\\
आओ सब मिल गायें गाथा\\
दो नंबर मकान की\\
मात पिता की जै बोलो\\
मात पिता की जै\\
\\
(यह कविता मैं माता जी और पिता जी को\\
अर्पित करता हूँ। हम भारत के उस हिस्से में थे\\
जो अब पाकिस्तान में है।\\
तर्ज़—आओ बच्चो तुम्हें दिखायें झांकी हिंदुस्तान)
}
\end{verse}

\poemtitle{ \texthindi{पहला मिलन}}
\begin{verse}[\versewidth]\texthindi{
याद है जब हम पहली बार मिले थे\\
थामा पहली बार नर्म काँपता हाथ\\
उँगलियों ने चेहरे से बाल हटाए\\
आँख झुकी “आप बोलती बहुत\\
अच्छा हैं।”\\
\\
बिजली जिस्म में फैली होंठ थर्राए\\
छोटी छोटी बातों पे रूठ जाया करते\\
रात की नींद दिन का चैन गवाया करते\\
कई सुंदर सपने दिन में बनाया करते\\
हवा में रंग बिरंगे महल सजाया करते\\
हाथों पे तरह तरह तेरा नाम लिख\\
अपने नाम से जोड़ा करते\\
ना रिश्तों का बोझ ना पीछे का गम\\
खुश आपस में गिले ना करते\\
\\
उत्सुकता थी दिल में घबराहट काफ़ी थी\\
याद है जब हम पहली बार मिले थे\\
याद है जब हम पहली बार मिले थे
}
\end{verse}

\poemtitle{\texthindi{मुहब्बत}}
\begin{verse}[\versewidth]\texthindi{
मुहब्बत मानो शब्दों में लायी नहीं जाती\\
हकीकत जो ज़बान से समझाई नहीं जाती\\
फूल की खुशबू हवाओं में रम जाती\\
हल्की मुस्कान दिल को है भाती\\
दिलों की बात चेहरे पे लाई नहीं जाती\\
मुहब्बत…\\
\\
झुकी आँखें होंठ काँपते दिल का राज़ बताते\\
गाल गुलाबी माथे पसीना खुद से वो शर्माते\\
अपनो से क्या परदा बात छुपाई नहीं जाती\\
मुहब्बत…\\
\\
ना पैसे की इसे चाहत ना ढूँढे कोई बड़ा नाम\\
बंगला ना शोहरत इसे बस दिल से है काम\\
ये रब की मेहर, दौलत से कमायी नहीं जाती\\
मुहब्बत…\\
\\
दिल की बात दिल जाने कहने से क्या लेना\\
नज़र नीची ने कह डाला होंठों को सी लेना\\
रूह बात करे रूह से मुँह से बताई नहीं जाती \\
मुहब्बत मानो शब्दों में लायी नहीं जाती\\
हकीकत ऐसी ज़ुबान से समझाई नहीं जाती
}
\end{verse}


\poemtitle{\texthindi{नाचूँ खुशियों से}}
\begin{verse}[\versewidth]\texthindi{
नाचूँ मैं खुशियों से रात दिन\\
मुझे मेरा प्यार मिला\\
यार मिला दिल दार मिला\\
नाचूँ मैं खुशियों से रात दिन\\
\\
जब से मैं ने होश सम्भाला\\
तुम्हें ही चाहा तुम्हें ही माँगा\\
राह में ठोकर जब मोहे लागी\\
बाँह पकड़ कर तू ने सम्भाला\\
जो भी मैं ने माँगा रब से\\
उस से ज़्यादा मिला\\
नाचूँ मैं …\\
\\
दिल में उमंगें लब पे तराने\\
सपनों ने ली है अंगड़ाई\\
फूल खिले हैं इस बगिया में\\
देखूँ जिधर बहार है छाई\\
कलियों के अब दिन आये हैं\\
करूँ क्या रुत से गिला\\
नाचूँ मैं खुशियाँ से रात दिन\\
मुझे मेरा प्यार मिला\\
यार मिला दिल दार मिला\\
नाचूँ मैं खुशियों से रात दिन
}
\end{verse}

% page 79 of working version.docx
\poemtitle{\texthindi{लगता है मैं घर आ गया हूँ
}}\begin{verse}[\versewidth]\texthindi{
ये कविता मैं भारत को, अपनी मातृ भूमि को\\
अर्पण करता हूँ। जो भारत छोड़ आये हैं,\\
आओ घर चलते हैं।\\
\\
खुदगर्ज़ी से खुशहाली पाने देश था छोड़ा\\
मात पिता भाई बहनों से नाता था तोड़ा\\
उन सुनहरी यादों को ताज़ा कर लेता हूँ\\
कदम जहाज़ से बाहर जब रखता हूँ\\
लगता है मैं घर आ गया हूँ\\
\\
इमीग्रेशन क्लर्क में देखूं बाप की परछाई\\
सर ओढ़े आँचल में माँ लौट के वापस आई\\
सड़क पे खेलते बच्चों बीच खुद को ढूंढूँ\\
शोर गुल में बचपन के खोये यारों को ढूंढूँ\\
बाहर निकलते ऐसे नज़ारे देखता हूँ\\
लगता है मैं घर आ गया हूँ\\
\\
पड़ोसी को मिलने का न्योता ना चाहिए\\
हमारे घर आये हो, चाय तो पी के जाइये\\
दो रोटी और बना लें गे, खाना यहीं खाइये\\
ऐसी प्यार भरी बातें सुनता हूँ\\
लगता है मैं घर आ गया हूँ\\
\\
जहां बड़ों कि इज़्ज़त अभी भी होती\\
अंत समय अकेले रहने नहीं देती\\
जहां बच्चे बढ़े बूढ़ों को कंधा देते\\
सर झुका पैरों को छू दुआ हैं लेते\\
ऐसी पुरानी रीतें देखता हूँ\\
लगता है मैं घर आ गया हूँ\\
\\
पड़ोसी जब चाहे दरवाज़ा खटका सकता है\\
मिलने को खास वक्त ज़रूरी ना समझता है\\
फ़ासला उन के अपने घर का मिट जाता है\\
ऐसे बड़े परिवार को साथ साथ देखता हूँ\\
लगता है मैं घर आ गया हूँ\\
\\
बड़े आदमी ताऊ और चाचा औरत\\
 मासी कहलाती है\\
हर बच्चा बच्ची अपनी ही बेटा \\
बेटी कहलाती है\\
जहां रिश्तों का मिट जाता है फ़र्क \\
अपने पराये में\\
 घुल मिल प्यार से सब का रहना देखता हूँ\\
लगता है मैं घर आ गया हूँ\\
\\
रेल गाड़ी में तेज़ “चाय गरम” की पुकार\\
पोटली से निकलें परांठे आम का अचार\\
मुँह में पानी आ जाता है, माँग लूँ?\\
दिल में आये विचार\\
“आप भी दो बुर्की ले लो”\\
अनजान हमसफ़र कहता है\\
दो रोटी सफ़र में खाता हूँ\\
लगता है मैं घर आ गया हूँ\\
\\
लाउड स्पीकर सुबह सुबह रब के गीत सुनाये\\
राम, वाहे गुरु, अल्लाह की ऊँची महिमा गाये\\
कोयल की मधुर आवाज़ सोये सपनों से जगाये\\
सुरीली आवाज़ों में माँ बाप से जफ्फी लेता हूँ\\
लगता है मैं घर आ गया हूँ\\
\\
दीवाली में जगमग देश हुआ होली में बना सतरंग\\
लोहड़ी में सुंदर मुंद्रिये राखी भाई बहिन के\\
संग\\
नवरात्रे, कंजकें, दसहरा हर मौके पे होता\\
सत्संग\\
जब ऐसे अपने अनेक त्यौहार देखता हूँ\\
लगता है मैं घर आ गया हूँ\\
\\
वो उड़ती पतंगें, पैंचे लड़ाना\\
दीवारों छतों पर दीयों का सजाना\\
गुल्ली डंडा, पिट्ठू, कंचों की आवाज़\\
कौओं की कैं कैं का शोर मचाना\\
नल से खींच ठंडे पानी में\\
ठिठुर के नहाना\\
राह चलते ऐसे भूले नज़ारे देखता हूँ\\
लगता है मैं घर आ गाया हूँ\\
\\
हवा महके जगाये भूली बिसरी यादें\\
धूल में अपनी मिट्टी की खुशबू\\
माँ बाप की फरियादें\\
खस खस सी सुगंध पहली बारिश की\\
ठंडी हवा\\
पानी से पेड़ घर की दीवारें धुलते देखता हूँ\\
लगता है मैं घर आ गया हूँ\\
\\
माँ बाप दादा नानी की ज़िंदगी दोहरायी जाती\\
भाई बहनों की दौड़ धूप कहानी सुनायी जाती\\
ज़िंदगी की ऊँच नीच, हालातें बताई जाती\\
बचपन से आज की खुली किताब देखता हूँ\\
लगता है मैं घर आ गया हूँ\\
\\
और फिर \\
बिछड़ते वक्त कैदी आंसुओं का छुपाना\\
अनकहे फिर ना मिलने के ख्यालों का आना\\
वो हाथों का पकड़ना फिर न छुड़ाना\\
लम्बी प्यार भारी जुदाई, कंधे सहलाना\\
टपकते सुर्ख आंसुओं में सोचता हूँ\\
लगता है मैं घर छोड़े जा रहा हूँ\\
\\
करता हूँ खुद से वादा, जल्द दोहराऊँ गा\\
लगता है मैं घर आ गया हूँ\\
\\
सुनसान हैं गलियाँ ना बन्दों की आवाज़\\
अजनबी चेहरे भाषा अलग यहाँ के साज़\\
पड़ोसी पड़ोसी को ना जाने\\
अपनों को भी न पहचाने\\
सालों से साथ है इन का\\
फिर भी लगते हैं अनजाने\\
\\
बिन वजह रोज़ गोलियों का चलना\\
मासूम बच्चों बड़ों का बन्दूकों से मरना\\
ऐसी जगह से हर साल वापिस लौटता हूँ\\
तो लगता है मैं घर आ गया हूँ\\
मातृ भूमि में आ गया हूँ\\
पितृ भूमि में समा गया हूँ\\
मैं अपने ही घर वापस आ गया हूँ\\
लगता है मैं घर आ गया हूँ\\
लगता है मैं घर आ गया हूँ
}
\end{verse}

\poemtitle{\texthindi{मैं कहाँ फँस गया हूँ 
}}\begin{verse}[\versewidth]\texthindi{
जब में ने कविता-“लगता है मैं घर आ गया\\
हूँ” लिखी तो मेरे भाई, प्रेम, ने कहा “भाई, \\
पढ़ के मज़ा आ गया और बहुत अच्छा लगा। \\
तुम ने ऐसी चीज़ें भी देखी, झेली हों गी जो\\
तुम्हें परेशान करती हैं। उस को मध्य नज़र\\
रखते हुए कुछ लिखो”\\
\\
लगभग बीस साल पहले ये कविता लिखी\\
थी। कई चीज़ें अभी भी लागू हैं। अब तो\\
भारत कई तरह से इतना बदल गया है की ये\\
शायद अब नहीं लिख पाता। ये परिवर्तन देख\\
कर बहुत खुशी और गर्व होता है।\\
\\
डेंगू टाइफाइड और मलेरिया\\
मखी मछरों का है राज\\
दूध में ज़्यादा नल में कम पानी\\
कूड़ा गलियों का सरताज\\
खुली नालियां, हवा में बदबू\\
पुरानी गलियाँ वैसी ही आज\\
बचपन की ऐसी निशानियाँ\\
देखता हूँ\\
सोचता हूँ मैं कहाँ फँस गया हूँ\\
\\
सड़कों पर वही भीड़ भड़का\\
ट्रकों का जहां राज है पक्का\\
सब को शिक्षा देता पिछवाड़ा\\
बुरी नज़र वाले तेरा मुँह काला\\
माँ का आशीर्वाद जय जय माता\\
डिप्पर एट नाईट ओ के टाटा\\
ट्रकों से लटकते गन्ने खींचें बच्चे\\
लगता है मैं फिर बचपन में आ गया हूँ\\
\\
मेट्रो मैं चड़ूँ जेबों को बचाऊं\\
खाने को देखूं, खाऊं ना खाऊं\\
पेट खराब हो गा ज़रूर\\
डॉक्टरों के चक्कर में न फँस जाऊं\\
हस्पताल बने पैसे की मशीने\\
बैंक बैलेंस खाली ना कर जाऊं\\
रोज़ बचाव के तरीके ढूंढता हूँ\\
सोचता हूँ मैं कहाँ फँस गया हूँ\\
\\
बदन कांपता है यहाँ की सर्दी से\\
फेफड़े बंद हुए धूएँ और गर्दी से\\
चोरी डकैती बलात्कारी\\
दिल दहल गया आवारागर्दी से\\
अरे यारों, शिकायत करें किस से\\
डर लगता यहाँ खाकी वर्दी से\\
ऎसे दुःख भरे हालात देखता हूँ\\
सोचता हूँ मैं कहाँ फँस गया हूँ\\
\\
कुर्सी खातिर आया राम गया राम हैं अभी\\
नाम बदले उन के पर काली हरकतें हैं अभी\\
देश में छाया अन्धकार रौशन इन का घर\\
कानून आम आदमी पर ना इन्हें कोई डर\\
नये चेहरों पे राजनीति पुरानी देखता हूँ\\
सोचता हूँ मैं कहाँ फँस गया हूँ\\
\\
हर काम के लिए जानकारी या रिश्वत लाओ\\
यहाँ का दस्तूर खुद खाओ दूजों को खिलाओ \\
मंत्री से ले चपरासी का रास्ता पैसे का\\
पर नारे ज़ोर से लगाते दुराचार हटाओ!\\
ये सालों पुरानी तरकीबें देखता हूँ\\
सोचता हूँ मैं कहाँ फँस गया हूँ\\
\\
चलते हुए देखता हूँ सामने तो\\
थूक कुत्तों की देन पे फिसलता\\
देखूं जो नीचे कार से जा टकरता\\
घर से बाहर जब मैं निकलता\\
बचाऊं खुद को सामने या नीचे से\\
सोचता हूँ कहाँ फँस गया हूँ\\
\\
लिखा है 'गधा पेशाब कर रहा है' \\
मगर आदमी खड़ा है\\
बिन वजह भौंकता कुत्तों का झुण्ड \\
निकल पड़ा है\\
सड़क पे उलटी तरफ कार स्कूटर चल पड़ा है\\
लाल बत्ती में ड्राईवर बेधड़क निकल पड़ा है\\
ऎसे अजब नज़ारे देखता हूँ\\
सोचता हूँ कहाँ फँस गया हूँ\\
\\
पक्की टिकट थी ट्रेन और प्लेन की\\
उसे भी उन्हों ने रद्द कर डाला\\
बहाने बनायें पर सच तो ये था\\
मिनिस्टर या वी आई पी आने वाला\\
सफ़र बन जाता है सफ्फ़र जहां\\
डाँटना दुतकारना झेलता हूँ\\
सोचता हूँ कहाँ फँस गया हूँ\\
\\
इंडिया का सफ़र बहुत लम्बा लगता\\
टी एस ए, थ्रोमबोसिस से डर लगता\\
जेट लैग सात दिन इधर भी उधर भी\\
दो हफ़्ते का सफ़र चार का बनता\\
ऎसे कष्ट भरे दिन देखता हूँ\\
सोचता है कहाँ फँस गया हूँ\\
\\
अब गिनता दिन घर वापस जाने के\\
बिन सोचे हरी सलाद खाने के\\
दोहते दोहतियों को गोद बिठाने के\\
उत्तर हो या दक्षिण घर अपना मन भाये\\
परिंदे छोड़ पुराना घोंसला नया बसाये\\
हर जगह फूल और कांटे अपने अपने\\
राह जो चुनी वहीं अपने सपने सजाये\\
ऎसे ख्यालों में डूबा जहाज़ में बैठता हूँ\\
इक घर छोड़ मैं दूजे घर जा रहा हूँ\\
\\
वो भी मेरा ये भी मेरा जहाँ भी जाता हूँ\\
लगता है मैं घर आ गया हूँ
}
\end{verse}

\poemtitle{\texthindi{अपनी मिट्टी }}
\begin{verse}[\versewidth]\texthindi{
गुज़रे साल पचास छोड़े अपना देश\\
मिट्टी उस की अभी भी अपनी लगती है\\
मिट्टी उस की अभी भी अपनी लगती है\\
\\
पहले लोग कहते बेटा भाई अब अंकल\\
जिस नाम से मुझे पुकारें\\
ज़ुबान उन की मीठी लगती है\\
\\
हवा नज़ारे रस्में लोग लगें अपने\\
जैसे कभी ना बिछड़े थे\\
कोयल की धुन मीठी\\
कुत्ते की भौं भौं भी अच्छी लगती है\\
\\
छुपी यादें खोल आँखें लें अंगड़ाई\\
पेड़ की छाया गर्म लू से बचाती\\
पैसा एक ना पल्ले न थे हम गरीब\\
प्यार भरी भरपूर ज़िंदगी\\
हर कमी को पूरा करती है\\
गुज़रे साल पचास छोड़े अपना देश\\
मिट्टी उस की अभी भी अपनी लगती है\\
मिट्टी उस की अभी भी अपनी लगती है
}
\end{verse}

\poemtitle{\texthindi{हमारा बचपन
}}\begin{verse}[\versewidth]\texthindi{
हम आठ, साइकल एक\\
भरपूर थी हमारी खुशी\\
इक निक्कर कमीज़\\
चप्पल का जोड़ा खज़ाना था\\
हर जश्न मनाते धूम धाम से\\
प्यार भर देता था खुशी\\
\\
माँ बाप मुसकाते चुपके पीते ज़हर\\
शहद हमें पिलाते थे\\
खुद रह कर भूखा\\
मक्खन लदे पराँठे हमें खिलाते थे\\
इक बादशाही ज़िंदगी से\\
इक दिन में बने खानाबदोश\\
ना जाने कै से हंस के\\
राज गद्दी पे हमें बिठाते थे\\
\\
पेड़ों पे आम अमरूद नहीं\\
मीठा अमृत मिलता था\\
तंदूर से आग नहीं, नर्म सेक\\
दिल को सुकून मिलता था\\
\\
पैसा एक ना पल्ले\\
घर शीश महल दिखता था\\
खुशियों के फव्वारे गूंजते\\
बेफ़िक्र सुख चैन मिलता था\\
\\
नाम पानीपत पर अक्सर नल में पानी नहीं था\\
दो हाथ पम्प थे कसरत कोई गिला नहीं था\\
कभी आयी कभी गई \\
बिजली खेले आँख मिचौली\\
हाथ के पंखे, मोम बत्ती\\
कमियों का पता नहीं था\\
\\
कभी गुल्ली डंडा पिठ्ठु\\
कभी क्रिकेट की थी बारी\\
कँचे लुक्कन छुप्पी झूला\\
गुलेल से पथरी मारी\\
\\
पढ़ाई क्या चीज़ है\\
उस बारे कम सोचा था\\
अभी है बचपन खेलो कूदो\\
पढ़ने लिखने को उम्र है सारी\\
\\
हवा में पतंगें फल फूल ज़मीं पे\\
भर देते रंगीन नज़ारा\\
ना परवाह दूजों पास है क्या\\
घड़ा रहता भरा हमारा\\
आँगन दिन में खेल मैदान\\
मच्छरदानी में तारों नीचे\\
सोने का कमरा हमारा\\
\\
बचपन के अनमोल दौर की\\
तस्वीरें जब मन में खोलूँ\\
न गम न ज़्यादा सपने\\
बस वर्तमान ही काफ़ी था\\
\\
स्वामी जी शकुन्तला दर्शी\\
माँ का आशीर्वाद बरसता है\\
ऐसा सुंदर सुहाना बचपन\\
किस्मत वालों को मिलता है\\
\\
जैसे हवा में खुशबू, तालाब में\\
रंगीन कमल खिलता है\\
ऐसा सुंदर बचपन\\
किस्मत वालों को मिलता है\\
ऐसा सुंदर बचपन\\
किस्मत वालों को मिलता है
}
\end{verse}

\poemtitle{\texthindi{दिल करता है}}
\begin{verse}[\versewidth]\texthindi{
दिल करता है उड़ कर आऊँ\\
चंद घड़ियाँ तेरे साथ बिताऊँ\\
प्यार की गंगा निस दिन बहती\\
आ के अपनी प्यास बुझाऊँ\\
दिल करता है…\\
\\
जो बिछड़े हैं काश वो होते\\
प्यार के फूलों की माला पिरोते\\
मात पिता भाई बहनों को\\
दिल से कैसे भुलाऊँ मैं\\
दिल करता है…\\
\\
बचपन की यादें दोहरायें\\
भूले बिसरे गीत सुनायें\\
उन यादों को दिल से लगा कर\\
सालों साल बिताऊँ मैं\\
दिल करता है…\\
\\
खुशी की चादर में गम छुपाये\\
हर कोई अपना बोझ उठाये\\
एक अकेला थक जाये गा\\
आ के हाथ बटाऊँ मैं\\
दिल करता है…\\
\\
जन्म मरण तक साथ है अपना\\
चार दिनों का है ये सपना\\
सपनों को रंगों से भर दूँ\\
खुशियों के फूल चढ़ाऊँ मैं\\
दिल करता है…\\
\\
दिल करता है उड़ कर आऊँ\\
चंद घड़ियाँ तेरे साथ बिताऊँ\\
प्यार की गंगा निस दिन बहती\\
आ के अपनी प्यास बुझाऊँ\\
दिल करता है
}
\end{verse}

\poemtitle{\texthindi{व्यापारी की इज़्ज़त
}}\begin{verse}[\versewidth]\texthindi{
सामान बेच रही हूँ मैं इज़्ज़त तो नहीं\\
झुक रही हूँ ना समझना गैरत ही नहीं\\
\\
गरीब शायद पैसे से शराफ़त से नहीं\\
करती मेहनत पैसे खातिर चोरी नहीं\\
\\
घर बच्चों की परवरिश करना है धर्म\\
लोगों की गाली सुनने का शौक नहीं\\
प्यार से बात करो पैसे से ना तोलो मुझे\\
मुस्कान देने से दौलत घटती तो नहीं\\
\\
आवाज़ ऊँची कर इंसान बड़ा नहीं बनता\\
हल्का सर्द झोंका देता सुकून, तूफ़ान नहीं\\
\\
ना देखो मुझे शक की निगाह से\\
थमाओ हाथ में, फेंको नहीं पैसे\\
काम करती हूँ भिखारी तो नहीं\\
\\
सामान बेच रही हूँ मैं इज़्ज़त तो नहीं\\
झुक रही हूँ ना समझना गैरत ही नहीं\\
\\
माँ\\
सन्देश आया तेरे घर से\\
माँ की आँखें तेरी राह को तरसे\\
पिछले सावन वो बोली थी\\
अर्थी निकले गी अब इस घर से\\
सन्देश आया…\\
\\
तन से अपना दूध पिलाया\\
भूखे रह कर तुझे खिलाया\\
अपने मन की चाह मिटा कर\\
तेरा सपना सार कराया\\
शिकवा ज़बान पर कभी ना लाई\\
प्यार सदा आँखों से बरसे\\
सन्देश आया...\\
\\
मन ही मन वो घबराती थी\\
जल्द बुढ़ापा आये गा\\
बेटा डॉक्टर बन जाने पर\\
वक्त पे काम वो आये गा\\
उस के सपने टूट गये जब\\
पाँव निकाला तू ने घर से\\
सन्देश…\\
\\
माया खातिर जाल बिछाया\\
जाल में अपना आप गंवाया\\
मात पिता को छोड़ा तू ने\\
यादों पर भी पड़ गया साया\\
यादों के वो महल हैं खाली\\
महल निवासी निकले घर से\\
\\
सन्देश आया तेरे घर से\\
माँ की आँखें तेरी राह को तरसे\\
पिछले सावन वो बोली थी\\
अर्थी निकले गी अब इस घर से\\
सन्देश आया…\\
\\
उर्मिल की कहानी\\
\\
सुनो छोटी सी लड़की की लम्बी कहानी\\
सारी दुनिया से न्यारी प्यारी सी नानी\\
सुनो छोटी सी लड़की की कहानी\\
\\
राम पिता थे और सरला थी माता\\
छोटी सी गुड़िया के नंदी हैं भापा\\
नाना नानी से सुख प्यार बहुत पाया\\
आँख जब खुली न देखा बाप का साया\\
दिन सात बाद मिला उन्हें गंगा का पानी\\
सुनो…\\
\\
गुड़ियों से खेला और वायलिन बजाया\\
छोड़ पाकिस्तान लुधियाना घर बनाया\\
चीनी घर में थोड़ी पर बाँट इस ने खाई\\
यौवन में आई तो सूरज से की सगाई\\
नौ साल बाद पिया पानीपत का पानी\\
सुनो…\\
\\
चाँद सा चेहरा और आँखें हैं तारों सी\\
लौ से चमके डार्लिंग सूरज प्यारे की\\
पहले पहुंची निशी फिर आरती घर आई\\
साथ साथ करती थी बी एड की पढ़ाई\\
सिर पे ना ताज था सूरज बुलाये रानी\\
सुनो…\\
\\
मोम जैसा दिल चट्टान जैसा सर है\\
बाल धो के निकली तेल से वो तर है\\
आम चूसे निम्बू का आचार मन भाये\\
तीस नंबर घर में मेहमान सदा आये\\
इस शोर गुल में इनकी बीती जवानी\\
सुनो…\\
\\
छोड़ जोधपुर को दिल्ली घर बनाया\\
रेलवे कालोनी फिर आनंद विहार बसाया\\
ब्रिज इन की सौतन बैडमिंटन से प्यार था\\
बच्चों के ऊंचे नंबर दिलाने का विचार था\\
मॉडर्न स्कूल में वो बन गई मास्टरानी\\
सुनो…\\
\\
फिर क्या हुआ आरती?\\
\\
पार्किन्सन सिर सढ़िया का दिल इन पे आया\\
सुरमिल की ताकत ने दूर तक भगाया\\
जिस्म थक जाये अन्दर से ताकतवर है\\
परिवार का प्यार सेवा दवा का असर है\\
अंत हुए कष्ट बीमारी करे खत्म ज़िन्दगानी\\
सुनो छोटी सी लड़की की लम्बी कहानी\\
सारी दुनिया से न्यारी प्यारी सी नानी\\
सुनो छोटी सी लड़की की कहानी\\
(उर्मिल की श्रद्धांजली आरती की सहायता से
लिखी गयी है)
}
\end{verse}

\poemtitle{\texthindi{प्रेम लूथरा श्रद्धांजलि
}}\begin{verse}[\versewidth]\texthindi{
बूढ़ी हड्डियां कमज़ोर हुईं\\
जोड़ भी अब जुड़ने लगे\\
थक गया दिल धड़कते धड़कते\\
सांस भी अब रुकने लगे\\
\\
दौड़ धूप से झुलसा कोमल बदन\\
जिस्म कमज़ोर उठते नहीं कदम\\
जीने का कोई मकसद ना देखूँ\\
अंदर बाहर मैं थक गया हूँ\\
\\
कैंसर ने घर मुझ में बनाया\\
चोर के माफ़िक वो घुस आया\\
लाख दवा और दुआएँ करवाईं\\
फिर भी उस से जीत ना पाया\\
\\
एक म्यान में दो तलवारें रह ना पाएं\\
दुश्मन इक दूजे को सह ना पाएं\\
मैदाने जंग में हम ने की बहुत लड़ाई\\
उस कातिल से हम जीत ना पाए\\
\\
अब दिल करता है मैं सो जाऊँ\\
ऐसी नींद कि उठ नहीं पाऊँ\\
अपने मरने का डर नहीं लगता\\
बोझा अपनों पे ना बन जाऊँ\\
\\
अपने जाने का गम नहीं मुझे\\
डरता हूँ सोच तेरा चुप के रोना\\
बुरा शगुन तुझे कहेगी दुनिया\\
अकेले ज़िंदगी के बोझ का ढोना\\
\\
थोड़ा और जीने को मन करता\\
अपनों से दिल कभी नहीं भरता\\
कुछ और पल तेरा दामन ना छूटे\\
घुट गले लगाने को दिल करता\\
\\
काश उन संग वक्त बिताया होता\\
कल मिल लें गे जल्दी क्या है ?\\
काश ऎसा ख्याल आया ना होता\\
ज़िंदगी को और गले लगाया होता\\
\\
काश जिन्हें दुःख दिया उन्हें सताया ना होता\\
ना चिंता ना मुसीबत में घबराया होता\\
अपनों को इज़हारे मोहब्बत कराया होता\\
दुनिया को प्यार से और सहलाया होता\\
\\
प्यार लेन देन थी मेरी पहचान\\
दो चार दिन के संगी साथी\\
जैसे आये और गये मेहमान\\
अब फूलों की सेज बना हूँ\\
कल तस्वीर दीवारों की पहचान\\
\\
दिन होते लम्बे पर ज़िंदगी छोटी\\
इक इक पल गिनते बीता\\
उम्र दो पल में है खोती\\
\\
कल की है बात बचपन था जवानी थी\\
आज मैं ने दुनिया से रवानी की\\
\\
कुछ मीठी यादें कुछ शिकवे गिले\\
चंद दिन मेरी बातें हों गी\\
फिर इक कागज़ पे या किसी के दिल में\\
मेरे नाम की यादें हों गी\\
\\
ऐसे ख्यालों में डूबा खोया रहता\\
पूछने पर ज़बान से कह नहीं पाता\\
मेरे ख्याल मेरे संग ही जाने दो\\
जो पराये हैं वो ना समझें मेरी बात\\
जो समझें उन्हें आँखों से कह जाता\\
\\
कुछ तो अच्छा किया हो गा\\
जब प्यार हर तरफ़ देखता हूँ\\
जो दिया था दूजों को\\
अब वापिस लौटते देखता हूँ\\
कुछ आते करने आखरी नमस्ते\\
कुछ को अपने संग मरते देखता हूँ\\
\\
आंसुओं से आटा गूंधना\\
फिर रोटी का जल जाना\\
एक के लिए बनाना\\
फिर चुप बैठ अकेले खाना\\
\\
आँखें नम हो जाती हैं\\
तेरी अकेली ज़िंदगी सोचता हूँ\\
इस सोच से दरवाज़ा मौत का\\
बंद होने से रोकता हूँ\\
\\
पर करूं क्या आज तक\\
कोई जीत ना पाया काल से\\
पचपन साल मिलीं थी खुशियां\\
मुसकाना उसी ख्याल से\\
\\
मेरे हिस्से का खाना मेरा प्यार भी दुगने लुटाना\\
कल हमसफ़र, अब तुम में मैं हूँ\\
गुज़रे तू जिस भी हाल से\\
\\
जिस्म ना सही, रूह सदा तेरे साथ है\\
रहना सुखी हर हाल में तेरे सर पे मेरा हाथ है\\
जितने दिन मिले तुम्हें हँसते हुए बिताना तुम\\
यही दुआ मेरी राम से यही मेरी फ़रियाद है\\
\\
अपनी इनिंग अब पूरी कर ली मैं ने\\
जितनी लिखी उतनी रन बना लीं मैं ने\\
खुशियां बांटने से स्कोर की गिनती होती\\
फिर तो कई सेंचरी लगा दी मैं ने\\
\\
कोई तो क्लीन बाउल्ड या कौट आउट हुआ\\
\\
शुक्र है माँ बाप राम और राम शरणम् का\\
शशि अशित ज्योती दिशा तनुज सब का\\
मेरे संगी थे साथी थे इस खूबसूरत मेल में\\
सब को आउट होना है ज़िंदगी के खेल में\\
\\
अलविदा मैं सब को करता हूँ प्रेम से\\
ये मेरा आखरी खत है तुम्हारे प्रेम से\\
\\
फ़ॉण्डली प्रेम
}
\end{verse}

\poemtitle{\texthindi{एफ़ जी टी मर गई
}}\begin{verse}[\versewidth]\texthindi{
कोई खबर ना ज़िक्र है उस का\\
लगता है वो शायद मर गयी है\\
चर्चे सुनते चार दिन उस के\\
ज़िंदाबाद ज़ोर शोर के नारे\\
लूथरा खानदान इतिहास पन्नों में\\
शायद उस का ज़िक्र तो हो गा\\
एक या दूजी पीढ़ी पढ़ेगी मुस्कुराते\\
मीठी यादें खोने का फ़िक्र तो हो गा\\
अब मुलाकात होती चमकते फ़ोन पे\\
झुकी आँखें उँगलियों से\\
कौन करे तकलीफ़\\
घर काम छोड़ सफ़र करने से\\
अब वट्स ऐप रहे ज़िंदाबाद\\
हिंग लगे ना फटकड़ी\\
मुलाकात हो जाए\\
अब तो शब्द भी सिम्बल बने \\
करनी पड़ती कम बात \\
\\
भूले सुख जफ्फी हाथ सहलाने में\\
मुस्करा आँख से आँख मिलाने में\\
भूल गए मिल खेलना हँसना हँसाना\\
खुशियाँ बाँटना कंधे सहलाना\\
सुख दुख में आँ सुओं का बहाना\\
 \\
वक्त रुकता नहीं ज़माना बदल जाता \\
कुछ अच्छा बचा ज़्यादा खो जाता\\
अपने अपने में सब मग्न\\
परिवार का टीला धूल हो जाता\\
\\
डूब गई एफ़ जी टी वक्त के अंधेरे में\\
भूल गए गीत दिल करता है उड़ कर आऊँ\\
झूले पे चंद घड़ियाँ तेरे साथ बिताऊँ\\
दो नम्बर गाथा बोसा राम की मिठाई लाऊँ\\
\\
हम बेखुदी में तुम को पुकारे का गाना\\
दिन रात खेलना मिल जुल खाना\\
आम का पेड़ क्रिकेट ताश खिलाना\\
बिन वजह हँसी के फव्वारे लुटाना\\
\\
देखूँ उन यादों की खान में\\
सुनहरी मंच के कई खिलाड़ी\\
अलविदा कह रुला कर चले गए\\
कुछ जीवन की दौड़ धूप सह रहे\\
\\
जिन्हें ये तोहफ़ा मिला था मंच का\\
चार पीढ़ी कायम हैं मिलो मिलाओ\\
उन को इस अमृत का रस चखवाओ\\
\\
वक्त ने तब्दील किया जीवन हमारा\\
मंज़ूर है स्वीकार करना धर्म हमारा\\
\\
मौत किसी की हो खासकर अपनों की\\
इक आँसू तो आ जाता है\\
जब एफ़ जी टी का ख्याल आता है\\
चलचित्र का नज़ारा सामने आता है\\
पलकों से टपकता इक आँसू तो आ जाता है\\
पलकों से टपकता इक आँसू तो आ जाता है
}
\end{verse}
%\end{multicols}